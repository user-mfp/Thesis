\chapter{Evaluation}
\label{evaluation}

The \ac{IMI}-system in its final configuration was developed to be a reliable for every day use and easy to maintain. Although tests in the controlled environment of the lab were promising, it had to proof itself in a realistic scenario. Therefore, the \ac{IMI}-system was installed inside the Haßleben-showcase. Afterwards, the \ac{IMI}-exhibition about the showcase was defined by the museum staff and me. The staff was responsible for descriptive texts, detailed images and the overview sketch for proper feedback. Meanwhile, I assisted during the definition of the exhibition plane and the exhibits' positions.
\\
However, in order to determine, whether or not the \ac{IMI}-system raised awareness for the topic of the showcase a pre-study had to be made. Therefore, the behavior of visitors around the un-augmented showcase had to be observed. In addition, their awareness of the showcase's contents had to be found out as well. Upon this baseline, behavior of visitors with the \ac{IMI}-system present can be evaluated.
\\
A true insight can only be gained by examining how the \ac{IMI}-system is accepted by visitors over a longer period of time. Furthermore, they should not be influenced by as less unusual circumstances as possible.

The observation and questioning of visitors mostly leads to qualitative data. This data has to be evaluated differently than quantitative data gained from experimental research. In experimental research, a hypothesis leads to a study that produces data, which is then used to either accept or reject the hypothesis. Methods of grounded theory are more applicable to qualitative data, which is collected before a conclusive theory is shaped~\cite{GroundedTheory}. A comparison of the two different evaluation methods can be seen in Figure \ref{fig:experiment_vs_grounded}.
\begin{figure}[H]%
\includegraphics[width=\columnwidth]{../pics/experimental-grounded.eps}%
\caption{Comparison of experimental research (top) and grounded theory (bottom).~\cite{GroundedTheory}}
\label{fig:new_target} %Image from the book! 
\end{figure}
 
The qualitative data of the studies conducted at the Haßleben-showcase consists of descriptive notes of the behavior of visitors and transcripts of their answers during the subsequent interview. The grounded theory method consists of four general stages~\cite{GroundedTheory}. The first stage is \textit{open coding}. Here, the texts are analyzed and interesting characteristics are identified. Each characteristic is then coded by a distinctive term. In the next stage, related terms are combined as a concept. After the \textit{development of concepts}, those concepts can be further grouped during the  \textit{categorization}-stage. The fourth and final stage is the \textit{formation of a theory}. In our case, however, a weak theory was already available. Nevertheless, this theory could be improved after the pre-study.
\\
In retrospect, we followed a hybrid approach of experimental research and the grounded theory method, where qualitative data is evaluated to specify and improve an existing theory. Furthermore, the pre-study did have an influence on the implementation of the \ac{IMI}-system and the definition of the \ac{IMI}-exhibition of the Haßleben-showcase itself.

The observations of visitors around the Haßleben-showcase already had the purpose to gain a basic understanding of visitors' interaction. In contrast to grounded theory, the basic theory that \textbf{interaction with exhibits of a showcase raises the awareness about it and its contents} had already been assumed. This means, that casual engagement with a topic increases the knowledge about it.
\\
Related observations have been made with an interactive installation at the Science Museum in London, where visitors, mostly children, were invited by the installation to interact with it. Therefore, a user had to perform different gestures to trigger an animation. Feedback was given in textual and verbal form~\cite{Engagement}. After analyzing their observations with the methods of grounded theory, Haywood and Cairns among other things concluded that
\textit{
\begin{quote}
''engagement with the exhibit does have parallels with what is needed for successful learning, and this was not previously known.''\textnormal{~\cite{Engagement}}
\end{quote}}
Since, the Museum für Ur- und Frühgeschichte Thüringens is regularly visited by classes learning about the Roman age, a similar learning effect was striven for.

During the observations of the studies, the interaction of visitors with the showcase itself and amongst each other was observed and noted. Further, the size of a group of visitors along with their age were noted as well. In some cases the age of visitors had to be estimated. The time visitors spend around the Haßleben-showcase was also measured to be later taken as an indicator for the engagement of the visitors. A correlation of time spend with the showcase as a measure for engagement and the elaborateness of the answers could reveal supportive conclusions, later. When visitors left the Haßleben-room, they were asked about the showcase. The intention was to gain information about their grade of awareness considering the Haßleben-showcase. Therefore, a \textit{semi-structured interview} with each group of visitors leaving the Haßleben-room was conducted.
\\
I further stated that the awareness considering a showcase can be graded into the following three stages:
\begin{enumerate}
	\item Awareness of its \textbf{mere existence}.
	\item Awareness of its \textbf{general composition}.
	\item Awareness of its \textbf{specific composition}.
\end{enumerate}  

Hence, my questions were intended to grade each group of visitors with respect to those stages. In the end of the interview visitors were asked about their visiting habits concerning museum. 

The questions were:
\begin{itemize}
	\item ''Can you remember the grave of the princess of Haßleben?''
	\item ''What can you remember? -- What objects were on display?''
	\item ''What is, in your opinion, shown in the image?''\footnote{An image of a \textit{jewel box} positioned by the feet of the princess from the Haßleben-showcase was shown to the visitors.}
	\item ''What would you change (positive or negative)?''
	\item ''What were you especially interested in? What would you like to know more about?''
	\item ''Did you read the grave's explanatory text?''
	\item ''On what occasions do you usually visit museums and how often?''
\end{itemize}
All observations and answers were noted in a protocol-sheet. A summarizing table of all the answers can be seen in the Appendix of this work. 

%-----------------------------------------------------------------------------

\section{Pre-Study}
\label{evaluation_pre}

The Haßleben-showcase is at the beginning of the second room on the second floor of the museum. Visitors approach it from the long side when they walk through the door. There are several related showcases in the room; among them a coffin right in the middle. In the following room, the topic changes. There is a pottery oven in the corner and a bench on front of it. From the bench, I observed visitors in the previous room with the Haßleben-showcase in it. To disguise myself, I had one of the museum's audio guides\footnote{The Museum für Ur- und Frühgeschichte Thüringens offers audio guides for free. Visitors only have to leave a deposit. The audio guide is an app installed on an iPod Touch. It provides brief information about certain showcases and exhibits. They are identified by a sticker with the number of the corresponding audio track on it. The audio guide has German and English versions of each track.} with me.  
%\begin{figure}[H]%
%\includegraphics[width=\columnwidth]{../pics/blank.eps}%
%\caption{[UNFINISHED] Groundplan of the rooms around the Haßleben-showcase.}
%\label{fig:hassleben_pottery_layout} %Grundriss der drei Räume im 2. OG 
%\end{figure}The layout of the rooms can be seen in Figure \ref{fig:hassleben_pottery_layout}.
 
\paragraph{Observations} The pre-study took place from December 18th to 20th 2013 and on January 2nd and 3rd 2014. As museum staff explained, according to their experience the museum is usually well frequented during these dates. During the time of the pre-study, there were two school classes visiting the museum on field trips. A fifth grade could only be observed while walking through the museum. A sixth grade on the other hand, used a workbook provided by the museum. At the time, their topic in class was the Roman age. After their visit of the second floor, they were divided into groups of three or four pupils and given the protocol-sheets as a questionnaire. However, their answers are not taken into consideration here, because the class spent 45 minutes on the 2nd floor and the were not interviewed as the remaining subjects. Nevertheless, both classes' feedback is still noted in the summarizing table, but will not be included in the following analysis.
\\
In total, 53 visitors were encountered during the observations. The oldest visitor was 70 and the youngest 4 years of age. All visitors together had an average age of 31 and a half years. The distribution of the whole sample is depicted by the scatter plot in Figure \ref{fig:pre-study_ages}. 
\begin{figure}[H]%
\includegraphics[width=\columnwidth]{../pics/pre-age.eps}%
\caption{Age distribution of visitors and participants of the pre-study.}
\label{fig:pre-study_ages} %Scatter plot mit allen Altern und dem Durchscnitt
\end{figure}

Those 53 visitors were distributed over 19 groups. Here, the smallest group was a single person, while the largest group included six visitors. The average group size lay between two an three members. 
\\
Visitors were observed and interviewed as a group and not individually. Thus, the following answers were given by all individuals of a group and evaluated as the answer of the group. In some cases multiple answers were allowed.
\\
An accurate measurement of the \ac{LOS} for each group was not feasible. Hence, the groups' time with the Haßleben-showcase were categorized into intervals. Table \ref{tab:pre-study_los} shows the distribution of \ac{LOS} of all observed groups. For one group, a \ac{LOS} was \ac{n/s}.
\begin{table}[H]
	\centering
	\begin{tabular}{ br !{\vrule width 1pt} nr }
		\rowstyle{\bfseries}
		Length of stay	& Occurrence \\
		\toprule
		0s							& 1 		 		 \\ 
		0-30s						& 4 		 		 \\ 
		30-60s					& 7 		 		 \\ 
		1-2min 					& 6 		 		 \\  
		\hline
		\ac{n/s} 				& 1 		 		 \\ 
	\end{tabular}
	\caption{LOS of groups during the pre-study.}
	\label{tab:pre-study_los}
\end{table}
Overall, the observed groups tended to spent less than a minute at the Haßleben-showcase.
\\
The interaction of visitors with the showcase can be put into three categories. They imply ascending engagement with the exhibits and hence the topic. The first classified observations can be categorized as \textit{indifferent engagement}, where visitors appeared tired or uninterested. The second category includes \textit{appropriate engagement} in the showcase. Here, visitors were looking at the exhibits, but did nothing further. Lastly, \textit{increased engagement} was also observed. Different members of a group act differently around the exhibition. Hence, there are more values than groups in this section of the evaluations. 
\begin{table}[H]
	\centering
	\begin{tabular}{ bl !{\vrule width 1pt} nr }
		\rowstyle{\bfseries}
		Interaction with showcase				& Occurrence \\
		\toprule
		\textit{Indifferent Engagement}	& 					 \\ 
		Brief glance										& 1 				 \\ 
		Hands in pockets								& 1 				 \\ 
		\hline
		\textit{Appropriate Engagement}	& 					 \\ 
		Looking from broad side (front)	& 16				 \\ 
		Looking from narrow side (feet)	& 4 				 \\ 
		\hline
		\textit{Increased Engagement}		& 					 \\ 
		Reading the explanatory text		& 7 				 \\ 
		Thinking												& 1 				 \\ 
	\end{tabular}
	\caption{Interaction of visitors with the Haßleben-showcase during the pre-study.}
	\label{tab:pre-study_interaction_exhib}
\end{table}
Visitors interact with each other more than with a static showcase. Again, there is more than one way of interacting within a group of visitors. There might also be no observable interaction at all. Moreover, there were two lone visitors among the observed groups. Hence, there was no possible interaction with any one else and only the interaction between 51 visitors from 17 groups is listed in Table \ref{tab:pre-study_interaction_group}.
\begin{table}[H]
	\centering
	\begin{tabular}{ bl !{\vrule width 1pt} nr }
		\rowstyle{\bfseries}
		Interaction within groups				& Occurrence \\
		\toprule
		\textit{Movement}								& 					 \\ 
		Closed group 										& 31 				 \\ 
		Clustering in particular places	& 13				 \\
		Individually										& 7 				 \\  
		\hline
		\textit{Verbal Interaction}			& 					 \\ 
		Explaining											& 7					 \\ 
		Asking Question									& 4					 \\ 
		Discussing											& 3 				 \\ 
		Read aloud											& 1					 \\ 
		\hline
		\textit{Gestural Interaction}		& 					 \\
		Pointing at Exhibits						& 9					 \\
		Others													& 2 				 \\ 
	\end{tabular}
	\caption{Interaction within groups of visitors during the pre-study.}
	\label{tab:pre-study_interaction_group}
\end{table}
Like the interaction with the showcase, classified behavior has been categorized into two main categories. The first regards the \textit{movement of groups}. They can move as a closed group or break up into sub-groups and then reunite somewhere. Moreover, lose connections between visitors were also observed, where visitors mostly moved as individuals.
\\
The second category is \textit{interaction} and is divided into two sub-categories. First, \textit{verbal interaction} includes all spoken forms of interaction. The other form of observed interaction is \textit{gestural interaction}. Visitors were mostly pointing to emphasize what they were talking about. Also other gestures like describing the shape of an exhibit were observed.

%-----------------------------------------------------------------------------

\paragraph{Interviews} After a group left the area of the Haßleben-showcase, they were approached and asked, whether they would answer a number of questions. Since, the majority of visitors was German, all questions were asked in German. Five groups did not participate in the interview. Thus, there are only evaluable answers from 14 out of the 19 groups that were observed. 

The first question was if they would remember the gravesite of Haßleben. Twelve groups answered that they would remember it and only two admitted that they did not. When I followed up on the positive answers to check if they were correct, it turned out that three groups referred to another showcase on the first floor. Six of the groups were not quite sure and needed a hint.
\begin{table}[H]
	\centering
	\begin{tabular}{ bl !{\vrule width 1pt} nr }
		\rowstyle{\bfseries}
		Remembrance of the showcase	& Occurrence \\
		\toprule
		Yes													& 3					 \\
		Yes	(uncertain)							& 6					 \\
		\hline
		Yes	(incorrect)							& 3					 \\
		No													& 3					 \\
	\end{tabular}
	\caption{Visitors of the pre-study that remembered the Haßleben-showcase.}
	\label{tab:pre-study_question_1}  
\end{table}
All visitors of groups that did not remember the Haßleben-showcase were shown the correct showcase, so every interviewee knew what the actual topic of the questionnaire was.
\\
After the topic was clear, the groups were asked to name as many objects from the Haßleben-showcase as the could remember. In two cases the visitors could not remember anything. Two main categories emerged from the classified answers. First, \textit{jewelry} and second \textit{everyday objects} were identified. 
\begin{table}[H]
	\centering
	\begin{tabular}{ bl !{\vrule width 1pt} nr }
		\rowstyle{\bfseries}
		Recalled objects								& Occurrence \\
		\toprule
		\textit{Jewelry}								& 					 \\
		Jewelry													& 8					 \\
		Necklace (pearls/gems)					& 7					 \\
		Ring														& 5					 \\
		Torc														& 2					 \\
		Jewel box												& 2					 \\
		(Hair-)pins											& 1					 \\
		Fibula													& 1					 \\
		\hline
		\textit{Everyday Objects}				& 					 \\
		Bowls and pottery				  			& 7					 \\
		Coin(s)													& 3					 \\
		Bones on a plate		  	  			& 3					 \\
		Comb										  			& 2					 \\
		Belt buckle							  			& 2					 \\
		Metal objects by the leg  			& 1					 \\
		Bucket													& 1					 \\
		\hline
		Skeleton												& 4					 \\
	\end{tabular}
	\caption{Objects from the Haßleben-showcase visitors of the pre-study's recalled.}
	\label{tab:pre-study_question_2}  
\end{table}
Noteworthy about the referred objects was the fact that many of them were described and, hence, had to be interpreted to be classified.
\\
The visitors imagination and expertise was tested with the third question. The groups were shown an image of the jewel box placed at the feet of the skeleton. The wood of the jewel box had deterred and only its metal fittings and content, a necklace and fibula, were left.  
\begin{table}[H]
	\centering
	\begin{tabular}{ bl !{\vrule width 1pt} nr }
		\rowstyle{\bfseries}
		Interpretations of the image	& Occurrence \\
		\toprule
		Jewel box (and content)				& 7					 \\
		Necklace and fibula						& 2					 \\
		\hline
		Bracelet and pendant					& 2					 \\
		Jewels												& 2					 \\
		\hline
		I don't know									& 1					 \\
	\end{tabular}
	\caption{Interpretations of an image of the jewel box by visitors of the pre-study.}
	\label{tab:pre-study_question_3}  
\end{table}
Half of the groups gave the correct answer and two more groups described the correct content, but did not mention the jewel box itself. Four more groups also tried to describe the content and were less precise. One group could not tell what was depicted on the image.
\\
When asked about the improvements one third of the visitors answered that everything was well end they would not change anything. The remaining recommendations were diverse.
\begin{table}[H]
	\centering
	\begin{tabular}{ bl !{\vrule width 1pt} nr }
		\rowstyle{\bfseries}
		Improvement suggestions										& Occurrence \\
		\toprule
		Maps of the site of the find							& 3					 \\
		More suitable for small children					& 2					 \\
		Informational video												& 2					 \\
		Better illumination of the showcase 			& 1 				 \\
		Difference between original and replicas	& 1					 \\
		Temporal classification										& 1					 \\
		Being able to touch things								& 1					 \\
		More data																	& 1					 \\
		\hline
		Nothing / good as it is										& 6					 \\
	\end{tabular}
	\caption{Improvement suggestions for the Haßleben-showcase by visitors of the pre-study.}
	\label{tab:pre-study_question_4}  
\end{table}

\begin{table}[H]
	\centering
	\begin{tabular}{ bl !{\vrule width 1pt} nr }
		\rowstyle{\bfseries}
		Interests in contents from the showcase			& Occurrence 	\\
		\toprule
		\textit{Epoch}															& 					 	\\
		Procedures (daily routines, crafts, rites)	& 5					 	\\
		Romans and Germans (Tacitus)								& 1					 	\\
		Historic relevance (3rd - 5th century A.D.)	& 1					 	\\
		\hline
		\textit{Haßleben}														& 					 	\\
		Date of the excavation											& 6				 		\\
		Site of the find														& 5				 		\\
		Princess of Haßleben												& 4				 		\\
		Further remains (DNA samples)								& 3				 		\\
		\hline
		\textit{Others}															& 					 	\\
		Pottery																			& 6					 	\\
		History of humankind												& 6					 	\\
		Reproductions and originals									& 3					 	\\
		Curb information flood											& 2					 	\\
		\hline
		Nothing																			& 11					\\
	\end{tabular}
	\caption{Further interests in contents from the Haßleben-showcase by visitors of the pre-study.}
	\label{tab:pre-study_question_6}  
\end{table}
To get further insight of the engagement of the visitors, they were asked if the have read the explanatory text, which is positioned to the left of the Haßleben-showcase. Eight groups admitted to not having read the description. One lone visitor did not read the text, but knew it from prior visits. The remaining five groups had read the text. Two of those skimmed it. 
\begin{table}[H]
	\centering
	\begin{tabular}{ bl !{\vrule width 1pt} nr }
		\rowstyle{\bfseries}
		Read explanatory texts			& Occurrence \\
		\toprule
		Yes													& 3					 \\
		Yes, skimmed it							& 2					 \\
		\hline
		No, but known								& 1					 \\
		No													& 8					 \\
	\end{tabular}
	\caption{Visitors of the pre-study that have read the explanatory text of the Haßleben-showcase.}
	\label{tab:pre-study_question_6}  
\end{table}
At the end of the interview, visitors were asked on why and how often the visit museums. The reasons were sorted into two categories. The visitors go to see museums on \textit{special occasions} and out of certain \textit{interests}.
\begin{table}[H]
	\centering
	\begin{tabular}{ bl !{\vrule width 1pt} nr }
		\rowstyle{\bfseries}
		Answers	to first part of the seventh question	& Occurrence \\
		\toprule
		\textit{Special occasions}										& 					 \\
		Trip with family and friends									& 16				 \\
		Weather																				& 8					 \\
		Special exhibitions														& 2					 \\
		\hline
		\textit{Interests}														& 					 \\
		History of humankind													& 8					 \\
		General interest															& 4					 \\
		History																				& 2					 \\
		Archeology																		& 1					 \\
		Museums																				& 1					 \\
	\end{tabular}
	\caption{Answers to the seventh question of the pre-study's interview.}
	\label{tab:pre-study_question_7}  
\end{table}
The frequency at which the groups visit museums varies and the answers to this final question were vague. The two groups that gave precise descriptions also visit museums out of special interests.  
\begin{table}[H]
	\centering
	\begin{tabular}{ bl !{\vrule width 1pt} nr }
		\rowstyle{\bfseries}
		Answers	to second part of the eighth question	& Occurrence \\
		\toprule
		3 to 4 times a year														& 1					 \\
		Biannually																		& 1					 \\
		Once a year																		& 1					 \\
		Seldom																				& 7					 \\
		\hline
		\ac{n/s}																			& 4					 \\
	\end{tabular}
	\caption{Answers to the second part of the seventh question of the pre-study's interview.}
	\label{tab:pre-study_question_8}  
\end{table}

In succession of the pre-study, the \ac{IMI}-system was implemented and its technical principles were tested under controlled conditions of a lab (see Chapter \ref{installation_testing}). The subsequent \textit{main study} at the Haßleben-showcase was due when the final installation was ready to work in its real world environment.

%-----------------------------------------------------------------------------

\section{Main Study}
\label{evaluation_study}

The main study was conducted from July 23rd to 27th 2014. The basic procedure of the observation was the same. The behavior of visitors was noted. After their interaction with the \ac{IMI}-system, they were interviewed under similar conditions as visitors from the pre-study. Interviews were conducted with all members of a group at once. Thus, answers were given by a group as a unit and hence evaluated as such. 
\\
In addition to the questionnaire, the usability of the presentation software of \ac{IMI}-system was evaluated. Therefore, the \ac{SUS} was determined. The \ac{SUS}-test is used to gain insight into the subjective usability of a system~\cite{SUStest}. It consists of ten statements and users have to rate to which extend they do agree or disagree. The rating is based on agreement of each statement. The scores range from 0$\%$ (strongly disagree) to 100$\%$ (strongly agree). The average of all ratings combined yields the score of the \ac{SUS}. A score below 50$\%$ indicates problems with a systems usability, whereas a score above 70$\%$ is seen as good. Excellent usability of a system begins around 85$\%$. For the main study, the scoring was based on a Likert-scale from 0 (strongly disagree) to 10 (strongly agree) for better orientation.
\\
After the interviews, each visitor was given a form of the \ac{SUS}-test to fill out. Because most of the participants of the main study were German, the statements of the \ac{SUS} were in German~\cite{SUSdeu}.
 
\paragraph{Observations} In contrast to the pre-study, I did not rely on casual visitors alone. Thus, people were invited to participate in the evaluation of the \ac{IMI}-system of the Haßleben-showcase. In total, 58 participants took part in the main study. 36 of them were invited participants, while the remaining 22 were casual visitors. The average age of all participants was 31 and 5 month. The youngest visitor was 6 and the oldest 61 years of age. The age distribution of the whole sample is depicted by the scatter plot in Figure \ref{fig:main_study_ages}. 
\begin{figure}[H]%
\includegraphics[width=\columnwidth]{../pics/main-age.eps}%
\caption{Age distribution of visitors and participants of the main study.}%
\label{fig:main_study_ages} %Scatter plot mit allen Altern und dem Durchscnitt
\end{figure}

Participants and Visitors were distributed over 32 groups. The biggest group had nine members. This time there were 15 lone participants.

The \ac{LOS} was logged by the presentation software of the \ac{IMI}-system and evaluated by the statistics tool. The tool revealed an average \ac{LOS} of 5:25 minutes. The shortest session was over after 48 seconds. However, the longest session took 13:11 minutes. The range of the time participants spent with the system was wide, but half of the groups stayed between two and six minutes.
\begin{table}[H]
	\centering
	\begin{tabular}{ br !{\vrule width 1pt} nr }
		\rowstyle{\bfseries}
		Length of stay	& Occurrence 	\\
		\toprule
		0-2min					& 3		 				\\ 
		2-4min					& 9 	 				\\ 
		4-6min					& 7 	 				\\ 
		6-8min					& 6 	 				\\ 
		8-10min					& 3 	 				\\ 
		10-12min				& 2 	 				\\ 
		12-14min				& 1 	 				\\ 
		\ac{n/s}				& 1 		 			\\ 
	\end{tabular}
	\caption{LOS of groups during the main study.}
	\label{tab:main_study_los}
\end{table}
All observed groups pointed at the \ac{IMI}-exhibits as it was intended. Nevertheless, there were some unintended ways of pointing as well. Participants attempted \textit{other interaction} in addition to pointing gestures alone. A category of \textit{readability issues} combines observed and reported difficulties that arose during the interaction with the presentation software. The final category of observed behavior of the participants includes \textit{unintended behavior}. The visual feedback was intended for fine adjustments and a general overview. However, many users were predominantly using the visual feedback to hit a target and relied less on their own perception. 
\begin{table}[H]
	\centering
	\begin{tabular}{ bl !{\vrule width 1pt} nr }
		\rowstyle{\bfseries}
		Interaction with \ac{IMI}-system 			& Occurrence 	\\
		\toprule
		\textit{Pointing Issues}							& 					 	\\ 
		Pointing with left arm								& 3					 	\\ 
		Trembling															& 2						\\ 
		Pointing at display										& 1					 	\\
		\hline
		\textit{Other Interaction}						&							\\
		Think-aloud														& 3						\\
		Other gestures												& 1						\\
		\hline
		\textit{Readability Issues}						&							\\
		Insufficient time for reading					& 3						\\
		Text too small to read								& 2						\\
		\hline
		\textit{Unintended Behavior}					&							\\
		Over-fixated on visual feedback 			& 16					\\
		Indistinct affordance / rash actions	& 4						\\
	\end{tabular}
	\caption{Interaction of visitors with the \ac{IMI}-system during the main study.}
	\label{tab:main_study_interaction_exhib}
\end{table}
Interaction within groups can be ordered into two categories. \textit{Verbal interaction} has been observed in several variations throughout all groups of visitors. A form of non-verbal interaction was categorized as \textit{movement}. 
\begin{table}[H]
	\centering
	\begin{tabular}{ bl !{\vrule width 1pt} nr }
		\rowstyle{\bfseries}
		Interaction within groups			& Occurrence 	\\
		\toprule
		\textit{Verbal Interaction}		& 					 	\\ 
		Talking												& 9					 	\\ 
		Explaining										& 4					 	\\ 
		Discussing										& 3					 	\\ 
		Whispering										& 1					 	\\ 
		Reading out										& 1					 	\\ 
		\hline
		\textit{Movement}							& 					 	\\ 
		Individually									& 2						\\ 
		Pointing participant changed	& 2			 			\\ 
		\hline
		Crossed Arms									& 1						\\
		\ac{n/s}											& 16 					\\ 
	\end{tabular}
	\caption{Interaction within groups of visitors during the main study.}
	\label{tab:main_study_interaction_group}
\end{table}
Additional interaction within the group was not observable for lone participants.

%-----------------------------------------------------------------------------

\paragraph{Interviews} All groups of visitors were interviewed after their interaction with the \ac{IMI}-system. They were asked the same questions like the groups of the pre-study. Thus, again the first question was, whether they would remember the gravesite of Haßleben. All groups except one did remember the Haßleben-showcase.
\begin{table}[H]
	\centering
	\begin{tabular}{ bl !{\vrule width 1pt} nr }
		\rowstyle{\bfseries}
		Answers	to first question	& Occurrence 	\\
		\toprule
		Yes												& 31					\\
		No												& 1						\\
	\end{tabular}
	\caption{Answers to the first question of the main study's interview.}
	\label{tab:main_study_question_1}  
\end{table}
The following question targeted the memory of the groups. Hence, visitors were asked to name as many things inside the showcase as they could remember. The answers can be assigned to one of two main categories. Participants recalled \textit{jewelry} and \textit{everyday objects}. Additionally, they named a few things that were not inside the showcase.
\begin{table}[H]
	\centering
	\begin{tabular}{ bl !{\vrule width 1pt} nr }
		\rowstyle{\bfseries}
		Answers	to second question			& Occurrence 	\\
		\toprule
		\textit{Jewelry}								& 					 	\\
		Jewelry													& 12					\\
		Hairpins												& 12					\\
		Golden ring											& 12					\\
		Necklace												& 12					\\
		Box brooch (''Dosenfibel'')			& 11					\\
		Jewel box												& 5						\\
		Torc														& 4						\\
		Fibula													& 4						\\
		Earrings												& 3						\\		
		\hline
		\textit{Everyday Objects}				& 					 	\\
		Bowls and pottery								& 22					\\
		Coin(s)													& 13					\\
		Silver plate										& 12					\\
		Key (to the jewel box)					& 7						\\
		Belt buckle											& 4						\\
		Personal and household items		&	2						\\
		Comb														& 1						\\
		\hline
		Skeleton of the princess				& 16					\\
		Skeleton of the dog							& 14					\\
		Dirt and stones									& 2						\\
		\hline
		\textit{Not in the showcase}		& 					 	\\
		Silver box											& 1						\\
		Bag															& 1						\\
		Spear head											& 1						\\
		\hline
	\end{tabular}
	\caption{Answers to the second question of the main study's interview.}
	\label{tab:main_study_question_2}  
\end{table}
Re-organizing the categories by other properties of the exhibits that were remembered reveals, that more than half of the objects mentioned by the visitors were \ac{IMI}-exhibits. Table \ref{tab:main_study_question_3_relations} shows the exact relation of normal and interactive exhibits.
\begin{table}[H]
	\centering
	\begin{tabular}{ bl !{\vrule width 1pt} nr }
		\rowstyle{\bfseries}
		Recalled object		& Occurrence 	\\
		\toprule
		\ac{IMI}-exhibit	& 80				 	\\
		Other exhibit			& 67					\\
	\end{tabular}
	\caption{Relation of interactive to non-interactive objects from the Haßleben-showcase recalled by the participants.}
	\label{tab:main_study_question_3_relations}  
\end{table}
When presented with an image of the jewel box placed by the feet of the princess of Haßleben, 12 out of the 32 groups recognized the jewel box. 11 others groups described the correct contents of the jewel box, while the remaining nine groups' answers were not correct.  
\begin{table}[H]
	\centering
	\begin{tabular}{ bl !{\vrule width 1pt} nr }
		\rowstyle{\bfseries}
		Answers	to third question			& Occurrence 	\\
		\toprule
		Jewel box											& 12				 	\\
		Remains of a necklace					& 11					\\
		\hline
		Pearls												& 1						\\
		Bracelet											& 1						\\
		\hline
		Ring													& 3						\\
		I don't know									& 2						\\
		Key														& 1						\\
		Earring												& 1						\\
	\end{tabular}
	\caption{Answers to the third question of the main study's interview.}
	\label{tab:main_study_question_3}  
\end{table}
In succession of questions concerning the memorability of the exhibits, participants were asked what they would change about the showcase. Their feedback could be of a positive or negative nature. The participants identified \textit{general aspects} about the showcase and its contents that needed improvement. Further, participants gave feedback concerning the \ac{IMI}-system. It is categorized as \textit{feedback}-, \textit{readability}-, and \textit{interaction}-relevant. 
\begin{table}[H]
	\centering
	\begin{tabular}{ bl !{\vrule width 1pt} nr }
		\rowstyle{\bfseries}
		Answers	to fourth question											& Occurrence 	\\
		\toprule
		\textit{General Aspects}												& 					 	\\
		Visibility of exhibits (occlusion, reflections) & 13					\\
		Less trivia and more background information			& 2						\\
		Reconstruction of the jewel box									& 1						\\
		\hline
		\textit{Feedback}																& 					 	\\
		Improve feedback on exhibition plane itself			& 6						\\
		Reduce trembling																& 3						\\
		Additional instructions (sign or note)					& 3						\\	
		\hline
		\textit{Readability}														& 					 	\\
		More time to read / general Readability					& 6						\\
		Bigger display in another position							& 4						\\
		Optimize layout (text and images are too much)	& 1						\\
		\hline
		\textit{Interaction}														& 					 	\\
		Free movement																		& 4						\\
		Further gestures																& 3						\\
		Pointing for multiple users											& 1						\\
		Improve recognition for children								& 1						\\
		Pointing for left-handed users									& 1 					\\
		\hline
		\textit{Others}																	& 					 	\\
		More interactive exhibits												& 4						\\
		Clear separation of exhibits										& 2						\\
		Music or sounds																	& 1						\\
	\end{tabular}
	\caption{Answers to the fourth question of the main study's interview.}
	\label{tab:main_study_question_4}  
\end{table}
Participants were also asked what they would like to know more about after having seen the Haßleben-showcase. The categrories for interest of the visitors are about the \textit{general topic} of the exhibition and about the \textit{showcase} and its particular \textit{exhibits} as well. Some groups of visitors requested more interactive exhibits.
\begin{table}[H]
	\centering
	\begin{tabular}{ bl !{\vrule width 1pt} nr }
		\rowstyle{\bfseries}
		Answers	to fifth question								& Occurrence	\\
		\toprule
		\textit{Showcase}												&		 				 	\\
		Historical significance of the findings	&	8						\\
		Information about the princess					& 4						\\
		Information about Haßleben							& 1						\\
		\hline
		\textit{Exhibits}												& 						\\
		Skeleton of the dog											& 4						\\
		Box brooch (Dosenfibel)									& 2						\\
		Information about grave furnishings			& 2						\\
		Skeleton of the princess								& 2						\\
		Information about the jewelry						& 1						\\
		Information about the coins							& 1						\\
		Information about the comb							& 1						\\
		\hline
		\textit{General Topic}									& 						\\
		General information about the topic			& 4						\\
		Live back then (Roman age)							& 3						\\
		\hline
		More \ac{IMI}-exhibits									& 3						\\
		\hline
	\end{tabular}
	\caption{Answers to the fifth question of the main study's interview.}
	\label{tab:main_study_question_5}  
\end{table}
During the main study, three groups have read the explanatory text of the Haßleben-showcase, while the remaining 29 groups did not do that. Three groups however already knew the text and thus did not read it again.
\begin{table}[H]
	\centering
	\begin{tabular}{ bl !{\vrule width 1pt} nr }
		\rowstyle{\bfseries}
		Answers	to sixth question		& Occurrence	\\
		\toprule
		Yes													& 2					 	\\
		Yes, skimmed it							& 1					 	\\
		\hline
		No, but known								& 3						\\
		No, because of audio guide	& 2						\\
		No													& 24					\\
	\end{tabular}
	\caption{Answers to the sixth question of the main study's interview.}
	\label{tab:main_study_question_6}  
\end{table}
Toward the end of the interview, participants were asked under what circumstances they usually visit a museum. The majority replied that \textit{spacial occasions} and certain \textit{interests} were their motivation, but also \textit{fees} were identified as one of the decisive categories.
\begin{table}[H]
	\centering
	\begin{tabular}{ bl !{\vrule width 1pt} nr }
		\rowstyle{\bfseries}
		Answers	to first part of the seventh question	& Occurrence 	\\
		\toprule
		\textit{Special occasions}										& 					  \\
		Vacation (city trips)													& 21					\\
		Special exhibitions														& 8						\\
		Trip with family and friends									& 4						\\
		Open museums' night														& 3						\\
		Weather																				& 2						\\
		\hline
		\textit{Interests}														& 					  \\
		Topical relevance															& 6						\\
		Art exhibitions																& 3						\\
		Work																					& 3						\\
		\hline
		\textit{Fees}																	& 					  \\
		Free admission																& 3						\\
		Special prices																& 3						\\
		\hline
		\textit{Others}																& 					  \\
		Boredom																				& 3						\\
		Novelties																			& 1						\\
		\hline
	\end{tabular}
	\caption{Answers to the first part of the seventh question of the main study's interview.}
	\label{tab:main_study_question_7}  
\end{table}
The final question was how often participants do frequent a museum. The answers were varied from monthly over biannually and up to once in a decade.
\begin{table}[H]
	\centering
	\begin{tabular}{ bl !{\vrule width 1pt} nr }
		\rowstyle{\bfseries}
		Answers	to second part of the seventh question	& Occurrence \\
		\toprule
		Once a decade																		& 1					 \\
		Seldom																					& 2					 \\
		Once a year																			& 2					 \\
		Two or three times a year												& 10				 \\
		Biannually																			& 4					 \\
		Once a quarter																	& 3					 \\
		Two or three times a quarter										& 4					 \\
		Monthly																					& 1					 \\
	\end{tabular}
	\caption{Answers to the second part of the seventh question of the main study's interview.}
	\label{tab:main_study_question_8}  
\end{table}

%\paragraph{Annotations}
%
%\begin{itemize}
	%\item Pre- and postcondition of exhibition
	%\item Survey of visitors' behavior prior to system's installation and afterwards
	%\begin{itemize}
		%\item Interaction between visitors
		%\item Interaction with display
		%\item \ac{LOS} SUS-test
		%\item Interviews
		%\item Evaluation-Forms
	%\end{itemize}
%\end{itemize}

%-----------------------------------------------------------------------------

\paragraph{Usability Test} The evaluation of the presentation software was done after the interview. Therefore, each participant was given a copy of the \ac{SUS}-questionnaire. They filled it out and gave it back after finishing it. I stood a few meters away in order to not influence the participants and being available in case of uncertainty about one of the statements. 36 \ac{SUS}-questionnaires were handed in at the end of the main study.
\begin{figure}[H]%
\includegraphics[width=\columnwidth]{../pics/sus.eps}%
\caption{Results of the SUS-questionnaire.}%
\label{fig:main_study_sus} %Balkendiagramm der einzelnen Antworten mit dem Schnitt
\end{figure}

Figure \ref{fig:main_study_sus} depicts the separate scores for each statement of the \ac{SUS}-questionnaire alongside the final score. The overall score is 77,53$\%$, which is a good score. The scores for the separate statements ranged from a satisfactory 68.06$\%$ to an excellent 90$\%$.

%-----------------------------------------------------------------------------

\section{Post-Study}
\label{evaluation_post}

Since the time of the main study, the final \ac{IMI}-system has been installed at the Haßleben-showcase. It runs daily and is used by visitors of the museum. Meanwhile, log-files of each session are created and stored.

From July 29th to September 1st, the data of visitors using the \ac{IMI}-system was evaluated as a long term post-study under real life conditions. After this period of about two months, the log-files were evaluated with the statistics tool of the \ac{IMI}-system to gain information about the acceptance and usage of the system.
\\
Overall 410 sessions were recorded during this period. 206 of them were \textit{empty sessions}, which means that no interaction was recorded\footnote{Empty sessions occur when visitors are trackable by the depth sensor but did not interact with presentation software. Moreover, groups of visitors do count as well as staff that is passing by the showcase while doing their daily work.}. However, the remaining 204 sessions included the recordings of events during \textit{active sessions}. Hence, probably more than half of the visiting groups interacted the system to acquire information about the exhibits inside the Haßleben-showcase.
\\
On average the interaction lasted for 1:34 minutes. Meanwhile, the variation of the time visitors used the \ac{IMI}-system varied between 11 seconds and 11:43 minutes.
\\
During those 204 active sessions a total of 392 \ac{IMI}-exhibits were selected by users and presented by the presentation software. The share of each \ac{IMI}-exhibit is depicted by Figure \ref{fig:post-study_selections}. 

%Selected targets: ranking see Figure \ref{fig:post-study_selections}
%\\1. Halsring: 				66 (hard: 288) $\to 22.92\%$
%\\2. Dosenfiebel: 		53 (hard: 442) $\to 11.99\%$
%\\3. Silberteller:		51 (hard: 489) $\to 10.43\%$
%\\4. Ring: 						50 (hard: 484) $\to 10.33\%$
%\\5. Schlüssel: 			46 (hard: 368) $\to 12.5\%$
%\\6. Hundeskelett: 		45 (hard: 305) $\to 14.75\%$
%\\7. Haarnadeln: 			39 (hard: 252) $\to 15.48\%$
%\\8. Münzen: 					24 (hard: 277) $\to 8.66\%$
%\\9. Schmuckkästchen:	18 (hard: 386) $\to 4.66\%$
%\\Total of 392 selections $\to$ 196 requests per month

\begin{figure}[H]%
\includegraphics[width=\columnwidth]{../pics/selections.eps}%
\caption{Overview of selected \ac{IMI}-exhibits during the post-study.}%
\label{fig:post-study_selections} %Balkendiagramm der selected targets
\end{figure}

The difficulty of selecting a target can be described by a \textit{selectivity quota}. Here, this is the relation of events for selecting and marking a target. This quota indicates how well a certain target can be selected. If the quota is too low the position or kernel of the respective \ac{IMI}-exhibit have to be adjusted to compensate, because the pointing position often leaves and re-enters the kernel of the target. In the case of this long term post-study the selectivity quota for each \ac{IMI}-exhibit of the Haßleben-showcase is shown in Figure \ref{fig:post-study_selectivity}.
\begin{figure}[H]%
\includegraphics[width=\columnwidth]{../pics/selectivity.eps}%
\caption{Overview of the selectivity for each \ac{IMI}-exhibit during the post-study.}
\label{fig:post-study_selectivity} %Balkendiagramm der selected targets
\end{figure}

Another sign for an intricate selection of targets is a high rate of transitions between two different targets. There are many transitions between certain pairs of targets. This indicates that the \ac{IMI}-system has difficulties in distinguishing between the intended targets. These pairs of targets either lie very close to each other or it can be the same target twice (selectivity quota). A high number of transitions of the first constellation tells that the respective \ac{IMI}-exhibits need to be separated more clearly from each other. Also stricter separation by the kernels is a way to improve on that matter.

Out of 72 possible transitions between different targets only 36 actually happened. The most common transitions are listed in Table \ref{tab:post-study_transitions}. From this ranking it is clear that distinguishing the torc from the hairpins is the most difficult task of the \ac{IMI}-exhibition inside the Haßleben-showcase. 

\begin{table}[H]
	\centering
	\begin{tabular}{ bl !{\vrule width 1pt} nr }
		\rowstyle{\bfseries}
		Transition																					& Occurrence	\\
		\toprule
		Torc $\leftrightarrow$ Hairpins											& 777	 				\\
		Coins $\leftrightarrow$ Torc												& 517					\\
		Coins $\leftrightarrow$ Hairpins										& 374					\\
		Key $\leftrightarrow$ Ring													& 198					\\
		Silver plate $\leftrightarrow$ Ring									& 92					\\
		Box brooch $\leftrightarrow$ Coins									& 85					\\
		Box brooch $\leftrightarrow$ Hairpins								& 82					\\
		Key to the jewel box $\leftrightarrow$ Silver plate	& 67					\\
		Skeleton of the dog $\leftrightarrow$ Torc					& 64					\\
		Hairpins $\leftrightarrow$ Skeleton of the dog			& 36					\\
	\end{tabular}
	\caption{Transitions between targets during the long-term post-study.}
	\label{tab:post-study_transitions}  
\end{table}

The age of the visitors that used the \ac{IMI}-system could not be resolved. Assuming the demographics offered by the sample of the pre-study was representative for casual visitors, logged sessions can be associated with this sample. Further, the museum is also visited by school classes on excursions and the population has to be adjusted accordingly. Hence, the complete table of all observed visitors during the pre-study, which included to school classes, can be seen as representative for the age distribution of the post-study. This results in a lower average age of 
