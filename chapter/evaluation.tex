\chapter{Evaluation}
\label{evaluation}

The \ac{IMI}-system in its final configuration was developed to be a reliable for every day use and easy to maintain. Although tests in the controlled environment of the lab were promising, it had to proof itself in a realistic scenario. Therefore, the \ac{IMI}-system was installed inside the Haßleben-showcase. Afterwards, the \ac{IMI}-exhibition about the showcase was defined by the museum staff and me. The staff was responsible for descriptive texts, detailed images and the overview sketch for proper feedback. Meanwhile, I assisted during the definition of the exhibition plane and the exhibits' positions.
\\
However, in order to determine, whether or not the \ac{IMI}-system raised awareness for the topic of the showcase a pre-study had to be made. Therefore, the behavior of visitors around the un-augmented showcase had to be observed. In addition, their awareness of the showcase's contents had to be found out as well. Upon this baseline, behavior of visitors with the \ac{IMI}-system present could be evaluated.
\\
A true insight could only be gained by examining how the \ac{IMI}-system would be accepted by visitors over a longer period of time. Furthermore, they should not be influenced by as less unusual circumstances as possible.

The observation and questioning of visitors mostly leads to qualitative data. This data has to be evaluated differently than quantitative data gained from experimental research. Thus, methods of \textit{grounded theory} were employed to do so. In experimental research, a hypothesis leads to a study that produces data, which is then used to either accept or reject the hypothesis. Meanwhile, grounded theory first collects data from a study. This data is then used to shape a conclusive theory~\cite{GroundedTheory}. A comparison of the two different evaluation methods can be seen in Figure \ref{fig:experiment_vs_grounded}.

\textbf{F I G U R E}
%\begin{figure}%
%\includegraphics[width=\columnwidth]{filename}%
%\caption{.}%
%\label{fig:new_target} %flow chart von updateTarget 
%\end{figure}
 
\textbf{Methods of grouded theory are...}

The observations of visitors around the Haßleben-showcase already had the purpose to gain a basic understanding of visitors' interaction. In contrast to grounded theory, the basic theory that \textbf{interaction with exhibits of a showcase raises the awareness about it and its contents} had already been assumed. This means, that casual engagement with a topic increases the knowledge about it.
\\
Related observations have been made with an interactive installation at the Science Museum in London, where visitors, mostly children, were invited by the installation to interact with it. Therefore, a user had to perform different gestures to trigger an animation. Feedback was given in textual and verbal form~\cite{Engagement}. After analyzing their observations with the methods of grounded theory, Haywood an Cairns among other things concluded that
\textit{
\begin{quote}
''engagement with the exhibit does have parallels with what is needed for successful learning, and this was not previously known.''\textnormal{~\cite{Engagement}}
\end{quote}}
Since, the Museum für Ur- und Frühgeschichte Thüringens is regularly visited by classes learning about the Roman age, a similar learning effect was striven for.

During the observations of the studies, the interaction of visitors with the showcase itself and amongst each other was observed and noted. Further, the size of a group of visitors along with their age were noted as well. In some cases the age of visitors had to be estimated. The time visitors spend around the Haßleben-showcase was also measured to be later taken as an indicator for the engagement of the visitors. A correlation of time spend with the showcase as a measure for engagement and the elaborateness of the answers could reveal supportive conclusions, later. When visitors left the \textit{Haßleben-room}, they were asked about the showcase. The intention was to gain information about their grade of awareness considering the Haßleben-showcase. Therefore, a \textit{semi-structured interview} with each group of visitors leaving the Haßleben-room was conducted.
\\
I further stated that the awareness considering a showcase can be graded into the following three stages:
\begin{enumerate}
	\item Awareness of its \textbf{mere existence}.
	\item Awareness of its \textbf{general composition}.
	\item Awareness of its \textbf{specific composition}.
\end{enumerate}  

Hence, my questions were intended to grade each group of visitors with respect to those stages. In the end of the interview visitors were asked about their visiting habits concerning museum. 

The questions were:
\begin{itemize}
	\item ''Can you remember the grave of the princess of Haßleben?''
	\item ''What can you remember? -- What objects were on display?''
	\item ''What is, in your opinion, shown in the image?''\footnote{An image of a \textit{jewel box} positioned by the feet of the princess from the Haßleben-showcase was shown to the visitors.}
	\item ''What would you change (positive or negative)?''
	\item ''What were you especially interested in? What would you like to know more about?''
	\item ''Did you read the grave's explanatory text?''
	\item ''On what occasions do you usually visit museums and how often?''
\end{itemize}
All observations and answers were noted in a protocol-sheet. A summarizing table of all the answers can be seen in the Appendix of this work. 

%-----------------------------------------------------------------------------

\section{Pre-Study}
\label{evaluation_pre}

The Haßleben-showcase is at the beginning of the second room on the second floor of the museum. Visitors frontally approach it when thez walk through the door. There are several related showcases in the room. Among them is a coffin in the middle of the room. In the following room, the topic changes. There is a pottery oven in the corner and a bench on front of it. The layout of the rooms can be seen in Figure \ref{fig:hassleben_pottery_layout}. From the bench, I observed visitors in the previous room with the Haßleben-showcase in it. To disguise myself, I had one of the museum's audio guides\footnote{The Museum für Ur- und Frühgeschichte Thüringens offers audio guides for free. Visitors only have to leave a deposit. The audio guide is an app installed on an iPod Touch. It provides brief information about certain showcases and exhibits. They are identified by a sticker with the number of the corresponding audio track on it. The audio guide has German and English versions of each track.} with me.  

\textbf{F I G U R E}
%\begin{figure}%
%\includegraphics[width=\columnwidth]{filename}%
%\caption{.}%
%\label{fig:hassleben_pottery_layout} %Grundriss der drei Räume im 2. OG 
%\end{figure}
 
The observations of the pre-study took place from December 18th to 20th 2013 and on January 2nd and 3rd 2014. As museum staff explained, according to their experience the museum is usually well frequented during these dates. In total, 19 groups including 53 visitors were encountered during the observations. In addition to that, there were also two school classes visiting the museum on field trips. A fifth grade could only be observed while walking through the museum. The sixth grade on the other hand, used a workbook provided by the museum. Their topic in class was the Roman age. After their visit of the second floor, they were divided into groups of three or four pupils and given the protocol-sheets as a questionnaire. However, their answers are not taken into consideration here, because the class spent 45 minutes on the 2nd floor and the were not interviewed as the remaining subjects. Nevertheless, both classes' feedback is still noted in the summarizing table.
\\
\textbf{- Groups and their size (total, min, max, avg) graph $\to$ Scatter plot with mean and/or median
\\
- Visitors and their age (total, min, max, avg) graph $\to$ Scatter plot with mean and/or median $\to$ bimodal distribution
\\
- Time spent around the showcase (classifications, amounts) table
\\
- Interaction with the showcase (classifications, amounts) table
\\
- Interaction between visitors (classifications, amounts) table
\\
- First Question (classifications, amounts) table
\\
- Second Question (classifications, amounts) table
\\
- Third Question (classifications, amounts) table
\\
- Fourth Question (classifications, amounts) table
\\
- Fifth Question (classifications, amounts) table
\\
- Sixth Question (classifications, amounts) table
\\
- Seventh Question (classifications, amounts) table}

%-----------------------------------------------------------------------------

\section{Study}
\label{evaluation_study}

The study was conducted at ... from ... to ... and ... participants took part. Also casual visitors were interviewed.
\\
\textbf{- Groups and their size (total, min, max, avg) graph $\to$ Scatter plot with mean and/or median
\\
- Visitors and their age (total, min, max, avg) graph $\to$ Scatter plot with mean and/or median $\to$ modal distribution
\\
- Time spent around the showcase (classifications, amounts) table ZUORDNEN!!!
\\
- Interaction with the showcase (classifications, amounts) table
\\
- Interaction between visitors (classifications, amounts) table
\\
- First Question (classifications, amounts) table
\\
- Second Question (classifications, amounts) table
\\
- Third Question (classifications, amounts) table
\\
- Fourth Question (classifications, amounts) table
\\
- Fifth Question (classifications, amounts) table
\\
- Sixth Question (classifications, amounts) table
\\
- Seventh Question (classifications, amounts) table}
%\paragraph{Annotations}
%
%\begin{itemize}
	%\item Pre- and postcondition of exhibition
	%\item Survey of visitors' behavior prior to system's installation and afterwards
	%\begin{itemize}
		%\item Interaction between visitors
		%\item Interaction with display
		%\item \ac{LOS} SUS-test
		%\item Interviews
		%\item Evaluation-Forms
	%\end{itemize}
%\end{itemize}

%-----------------------------------------------------------------------------

\section{Post-Study}
\label{evaluation_post}

\textbf{Keine Altersbetrachtung möglich}