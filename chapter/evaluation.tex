\chapter{Evaluation}
\label{evaluation}

The \ac{IMI}-system in its final configuration was developed to be a reliable for every day use and easy to maintain. Although tests in the controlled environment of the lab were promising, it had to proof itself in a realistic scenario. Therefore, the \ac{IMI}-system was installed inside the Haßleben-showcase. Afterwards, the \ac{IMI}-exhibition about the showcase was defined by the museum staff and me. The staff was responsible for descriptive texts, detailed images and the overview sketch for proper feedback. Meanwhile, I assisted during the definition of the exhibition plane and the exhibits' positions.
\\
However, in order to determine, whether or not the \ac{IMI}-system raised awareness for the topic of the showcase a pre-study had to be made. Therefore, the behavior of visitors around the un-augmented showcase had to be observed. In addition, their awareness of the showcase's contents had to be found out as well. Upon this baseline, behavior of visitors with the \ac{IMI}-system present could be evaluated as well.
\\
A true insight could only be gained by examining how the \ac{IMI}-system would be accepted by visitors over a longer period of time. Furthermore, they should not be influenced by as less unusual circumstances as possible.

The observation and questioning of visitors mostly leads to qualitative data. This data has to be evaluated differently than quantitative data gained from experimental research. Thus, I employed methods of \textit{grounded theory}. In experimental research, a hypothesis leads to a study that produces data, which is then used to either accept or reject the hypothesis. Meanwhile, grounded theory also collects data from a study. This data is then used to shape a theory~\cite{GroundedTheory}. A comparison of the two different evaluation methods can be seen in Figure \ref{fig:experiment_vs_grounded}.

\textbf{F I G U R E}
%\begin{figure}%
%\includegraphics[width=\columnwidth]{filename}%
%\caption{.}%
%\label{fig:new_target} %flow chart von updateTarget 
%\end{figure}
 
The observations of visitors around the Haßleben-showcase already had a purpose, to gain a basic understanding of visitors' interaction. In contrast to grounded theory, we already had the basic theory that \textbf{interaction with exhibits of a showcase raises the awareness about it and its contents}. This means, that casual engagement with a topic increases the knowledge about it.
\\
Related observations have been made with an interactive installation at the Science Museum in London. Visitors were invited by the installation to interact with it. Therefore, a user had to perform different gestures to trigger an animation. Feedback was given in textual and verbal form~\cite{Engagement}. After analyzing their observations with the methods of grounded theory, Haywood an Cairns among other things concluded that
\textit{
\begin{quote}
''engagement with the exhibit does have parallels with what is needed for successful learning, and this was not previously known.''\textnormal{~\cite{Engagement}}
\end{quote}}


%-----------------------------------------------------------------------------

\section{Pre-Study}
\label{evaluation_pre}

The Haßleben-showcase is at the beginning of the second room on the second floor of the museum. Visitors frontally approach it when the come through the door. There are several related showcases in the room. Among them is a coffin in the middle of the room. In the following room, a the topic changes. There is a pottery oven in the corner and a bench on front of it. From this bench, I observed visitors in the previous room with the Haßleben-showcase in it. To disguise myself, I had one of the museum's audio guides\footnote{The Museum für Ur- und Frühgeschichte Thüringens offers audio guides for free. Visitors only have to leave a deposit. The audio guide is an app installed on an iPod Touch. It provides brief information about certain showcases and exhibits. They are identified by a sticker with the number of the corresponding audio track on it. The audio guide has German and English versions of each track.} with me.  

The interaction of visitors with the showcase itself and amongst each other was observed and noted. Further, I noted the size of a group of visitors along with their age, which in some cases had to be estimated. When visitors left the \textit{Haßleben-room}, they were asked about the showcase. The intention was to gain information about their grade of awareness considering the Haßleben-showcase. Therefore, I conducted a \textit{semi-structured interview} with each group of visitors leaving the Haßleben-room. \textbf{engagement}
\\
I hypothesized that the awareness considering a showcase can be graded into the following three stages:

\begin{enumerate}
	\item Awareness of its \textbf{mere existence}.
	\item Awareness of its \textbf{general composition}.
	\item Awareness of its \textbf{specific composition}.
\end{enumerate} 
 
Hence, my questions were intended to grade each group of visitors with respect to those stages. In the end of the interview visitors were asked about their visiting habits concerning museum. The questions were:

\begin{itemize}
	\item ''Can you remember the grave of the princess of Haßleben?''
	\item ''What can you remember? -- What objects were on display?''
	\item ''What is, in your opinion, shown in the image?''\footnote{An image of a \textit{jewel box} positioned by the feet of the princess from the Haßleben-showcase was shown to the visitors.}
	\item ''What would you change (positive or negative)?''
	\item ''What were you especially interested in? What would you like to know more about?''
	\item ''Did you read the grave's explanatory text?''
	\item ''On what occasions do you usually visit museums and how often?''
\end{itemize}

All observations and answers were noted in a protocol-sheet, which can be seen whole in the Appendix of this work. 

Ergebnisse der Protokolle:
\\- Alter (min, max, avg)
\\- Gruppengrößen (min, max, avg)
\\- Interaktion untereinander
\\- Interaktion mit Vitrine
\\- einzelne Antworten vorstellen

%-----------------------------------------------------------------------------

\section{Study}
\label{evaluation_study}

\paragraph{Annotations}

\begin{itemize}
	\item Pre- and postcondition of exhibition
	\item Survey of visitors' behavior prior to system's installation and afterwards
	\begin{itemize}
		\item Interaction between visitors
		\item Interaction with display
		\item \ac{LOS} SUS-test
		\item Interviews
		\item Evaluation-Forms
	\end{itemize}
\end{itemize}

Bimodale Verteilung der 1. Stichprobe gegen modale Verteilung der 2. Stichprobe

%-----------------------------------------------------------------------------

\section{Post-Study}
\label{evaluation_post}
