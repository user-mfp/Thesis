\chapter*{}
\section*{Abstract}

In cryptography it is necessary to show how well some cryptographic constructs are built against potential adversaries.
Usually this is done by presenting a security proof regarding to a group of adversaries to this constructs. This master thesis
introduces a special case of security proofs, namely game-playing proofs. The motivation for using game-playing proofs instead
of normal proofs is given through their natural and intuitive structure and understanding. This thesis is written as a
students' guide and should introduce this topic to bachelor and master students in computer science. It describes the
several techniques of game-playing proofs and provides examples that help the reader to understand the underlying concept.
% It is riddled with easy to understand examples and provides the description of techniques to put game-playing proofs into practice.
\vfill

\section*{Zusammenfassung}

In der Kryptografie ist es notwendig zu zeigen, wie sicher ein kryptografisches Konstrukt gegen vorstellbare Angreifer ist.
Normalerweise wird diese Sicherheit durch die Veröffentlichung eines Sicherheitsbeweises gezeigt, der sich meist auf eine Gruppe
von Angreifern bezieht. Diese Masterarbeit gibt eine Einführung in einen speziellen Fall der
Sicherheitsbeweise, die Game-Playing Beweise. Motiviert wird die Benutzung von Game-Playing Beweisen anstelle von normaler
Beweise durch ihre natürliche und intuitive Struktur. Desweiteren sind Game-Playing Beweise im Regelfall leichter zu verstehen
und nachzuvollziehen. Diese Arbeit ist als eine Einführung in das Thema für Bachelor- und Masterstudenten der Informatik geschrieben.
Die Arbeit enthält eine Reihe einfach zu verstehender Beispiele, um dem Leser das Konzept der Game-Playing Beweise nahezubringen
und zeigt Techniken, um diese zu realisieren.


%A Concrete Securty Treatment of Symmetric Encryption \cite{DBLP:conf/focs/BellareDJR97}.\newline
%Code-Based Game-Playing Proofs and the Security of Triple Encryption \cite{DBLP:conf/eurocrypt/BellareR06}.\newline
%Security of CBC, CFB and OFB Modes \cite{Lecture:IntroCrypto/Dodis08}. \newline
%Two-Pass Authenticated Encryption Faster than Generic Composition \cite{DBLP:conf/fse/Lucks05}.\newline
%The Security of the Cipher Block Chaining Message Authentication Code \cite{DBLP:journals/jcss/BellareKR00}.\newline
%Improved Security Analysis for CBC MACs \cite{BLP:conf/crypto/BellarePR05}.\newline
%Optimale Asymmetric Encryption \cite{DBLP:conf/eurocrypt/BellareR94}.\newline
%The EAX Mode of Operation \cite{DBLP:conf/fse/BellareRW04}.\newline
%Building PRFs from PRPs \cite{DBLP:conf/crypto/HallWKS98}.\newline
%Authenticated-encryption with associated-date \cite{DBLP:conf/ccs/Rogaway02}.\newline
%OCB: A block-cipher mode of operation for efficient authenticated encryption \cite{DBLP:journals/tissec/RogawayBB03}.\newline
%Sequences of games: a tool for taming complexity in security proofs \cite{Shoup04sequencesof}.\newline
%Relations Among Notions of Security for Public-Key Encryption Schemes \cite{DBLP:conf/crypto/BellareDPR98}.\newline
%XOR MACs: New Methods for Message Authentication Using Finite Pseudorandom Functions \cite{DBLP:conf/crypto/BellareGR95}.\newline
%The Notion of Security for Probabilistic Cryptosystems \cite{BLP:journals/siamcomp/MicaliRS88}.\newline
%






