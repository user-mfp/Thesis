\chapter{Abstract}
\label{abstract}

\paragraph{Annotations}

\begin{itemize}
	\item Exciting summary
	\item Create interest 
\end{itemize}



\subsection*{Old version}

Three dimensional (3D) graphics are a common sight in modern media, while two dimensional techniques are widely used for interaction. In virtual reality, several 3D devices are used to navigate and manipulate the virtual contents. Nevertheless, they are often not easy to use or error prone. At the same time, home entertainment systems (i.e. Kinect) can be operated with simple hand gestures. Hence, a novel interaction prototype has been developed as an interactive museum information (IMI)-system. Here, users are tracked with an ASUS Xtion-motion sensor. Gestures can be analyzed using the OpenNI framework. By simply pointing at it, a user then describes ones interest in a specific exhibit and the software will provide further information regarding the exhibit. The IMI-System is a low cost and maintenance system. Thus, the museum's staff defines and edits the objects of interest and the corresponding information themselves.
