\chapter{Abstract}
\label{abstract}

%\subsection*{New Version}

Museums tend to be perceived as old fashioned. At least, that is what some people assume and therefore not even consider having a look for themselves. Nevertheless, there are modern and open minded ones, which are willing to experiment with new possibilities, to get rid of their dusted reputation and to evolve.
\\
So, I was called to do exactly that. -- Implement a novel informatory interaction system for a museum of pre- and protohistoric history, where precious artifacts are locked up behind thick glass. The challenge was not only to develop a working prototype, but also make it intuitive, low maintenance and robust enough for everyday use. The system I developed employs the natural behavior of visitors. It detects potential users and enables them to interact with the system via pointing-gestures. Moreover, it can easily been set up and altered by museum personnel.

%\subsection*{Annotations}
%
%\begin{itemize}
	%\item Exciting summary
	%\item Create interest 
%\end{itemize}
%
%\subsection*{Old version}
%
%Three dimensional (3D) graphics are a common sight in modern media, while two dimensional techniques are widely used for interaction. In virtual reality, several 3D devices are used to navigate and manipulate the virtual contents. Nevertheless, they are often not easy to use or error prone. At the same time, home entertainment systems (i.e. Kinect) can be operated with simple hand gestures. Hence, a novel interaction prototype has been developed as an interactive museum information (IMI)-system. Here, users are tracked with an ASUS Xtion-motion sensor. Gestures can be analyzed using the OpenNI framework. By simply pointing at it, a user then describes ones interest in a specific exhibit and the software will provide further information regarding the exhibit. The IMI-System is a low cost and maintenance system. Thus, the museum's staff defines and edits the objects of interest and the corresponding information themselves.
