\chapter{Introduction}
\label{introduction}

%\textbf{Worum geht's hier überhaupt??? Und eine Masterarbeit fängt man nicht damit an 'Mein Auftrag war' sondern man erklärt das Ziel dahinter, warum ist das interessant, was ist das Forschungsinteresse...}

In this work, I describe the development of an \ac{IMI}. The system presents a novel way to augment public displays with a system that is intuitive to use, easy to maintain, and inexpensive. Natural interaction without additionally required devices on the users end lowers inhibition and frustration. Simultaneously, awareness for the displayed contents is raised.

Before this system could be developed, a collaboration with a local museum had to be established. Because the installation would be based on in- and output modalities that actually make sense in a museum, visitors and staff had to be observed and interviewed. After looking at several suitable museum candidates in Weimar. I chose the one with the most promise in fitting properties as well as institutional openness for my purposes. Together, we conceived some ideas for possible installations. Not all of them were applicable an some were too far off my expertise. Nevertheless, there were two concepts for augmenting the \textit{gravesite of Haßleben-showcase}, that we were very interested in and excited about.

The first concept, \textit{''Interaction with Tangibles''}, was directly addressing visitors' haptic perception. Therefore, it was planned to use \ac{MS} Gadgeteer-hardware\footnote{\ac{MS} Gadgeteer is a modular system of various hardware-components distrubuted by GHI Electronics. It resembles Arduino- and other microcontrollers.} as embedded components of tangible devices. A number of reproductions could be placed outside the showcase. Each interactive tangible could then be manipulated or placed on a pedestal to gain information about its corresponding exhibit. Here, certain exhibits could have been photogrammetically scanned in three dimensions. After that, the digital model could be scaled to a handy size and otherwisely modified. Ultimately, the tangible could be printed or casted. Such an object could then be enhanced by using \ac{RFID}-technology\footnote{RFID-transponders or -tags do not require any batteries, are cheap and robust. In addition, their range is very limited, which allows several tags on one tangible.}. In order to make it interactive, it would be fitted with such a \ac{RFID}-tag. There is a \ac{RFID}-module for Gadgeteer, which would have allowed identification of each tangible. The corresponding information could then be provided by any medium compatible with Gadgeteer.
\\
A different approach was based on an assumption of natural behavior of visitors. After a meeting at the museum, a second concept of \textit{''Interaction by Pointing''} emerged. Later the underlying assumption was confirmed by the observation of visitors' behavior around showcases. Visitors do not only talk about exhibits, but they also point at certain exhibits during interaction with each other. Therefore, a device should be build or utilized for users to point with, enabling them to select a certain exhibit inside the showcase. Additional information about the point of interest would then be displayed in an appropriate manner. 
\\
While all involved understood these concepts were fairly comprehensible, their technical realizations were unclear at first. Throughout further investigations, the work turned from testing various modes of input to a more technical approach. Both ways of input are special and revealed different challenges along the way.

%\textbf{Add a paragraph describing the final system with a sketch - this will make things much clearer for the reader!}

%\begin{center}
	%\textit{''Developing an innovative museum installation for and with a local museum.''}
%\end{center}
%Previous specifications
%\begin{itemize}
	%\item interactive museum installation
	%\item Gadgeteer (embedded hardware)
	%\item observations and interviews
	%\item input and output modalities that make sense in a museum
	%\item prototype
	%\item interactive design process
%\end{itemize}

%-----------------------------------------------------------------------------

Throughout the following chapters I documented my proceedings during the development of the aforementioned system. Chapter \ref{motivation} gives background information about the fields of study which are included in my work. Thus, there is a brief outline about the progression of technologies employed by museums, users behavior around public interfaces and with tangibles. In addition, a brief overview of virtual reality-techniques is given.
\\
Afterwards, I describe my goals for the development of this system. Before I come to explain the schematics and evaluation of my implementations, I give a short review of my partnering process. Thus, Chapter \ref{partnering} deals with finding a suitable museum for a collaboration. 
\\
Chapter \ref{conception} explains the whole development-process of the system's schematics and functionality. It begins with possible system designs and explains their possibilities and constraints. In the end of Chapter \ref{conception}, the final concept is shown along with necessary obligations such as an \ac{FSD} and the contract between me, the university and the museum.
\\
Chapter \ref{implementation} addresses the implementation of the system's functionality. Therefore, all libraries and softwares are explained in more detail.
\\
Experimental lab-installations and the final museum-installation are described in Chapter \ref{installation}. Therefore, measurements, hardware specifications, and other influential criteria are presented in detail.
\\
The final installation is evaluated in Chapter \ref{evaluation}, where visitors were observed and interviewed before and after alterations by the system. Chapter \ref{discussion} then deals with the discussion of the evaluation's findings.
\\
In the end, I thought about future work, which could improve, extend, and follow my system. In Chapter \ref{future_work}, I would also like to mention reactions and suggestions I encountered along my work.

%\paragraph{Introduction - Annotations}
%
%\begin{itemize}
	%\item Short overview, about what has been build
	%\item Summary
	%\\
	%\item System of libraries for pointing interaction
	%\item Information system (Information On Demand)
	%\item 'Uncharted territory' $\to$ technical focus
	%\item Template solution / 'just a proof of concept'
  %\\
	%\item Motivation
	%\item Working within the confines of museums respectively public installations
%\end{itemize}