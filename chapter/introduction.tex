\chapter{Motivation}
\label{motivation}

%\paragraph{Introduction - Annotations}
%
%\begin{itemize}
	%\item Short overview, about what has been build
	%\item Summary
	%\\
	%\item System of libraries for pointing interaction
	%\item Information system (Information On Demand)
	%\item 'Uncharted territory' $\to$ technical focus
	%\item Template solution / 'just a proof of concept'
  %\\
	%\item Motivation
	%\item Working within the confines of museums respectively public installations
	%\\
	%\item \ac{IMI} %acronym definition
	%\item \ac{UI}
%\end{itemize}

%-----------------------------------------------------------------------------

\paragraph{Related Work - Annotations}

\begin{itemize}
	\item Backgrounds
	\begin{itemize}
		\item Historical
		\item Technical
	\end{itemize}
	\item Application areas
	\item Not to much detail
	\item Only in respect to the thesis' topic
\end{itemize}

%-----------------------------------------------------------------------------

\section{Museums}
\label{motivation_museums}

Museums, much like libraries, are foremost seen as a place of knowledge and its preservation. Hence, visitors behave in a very reserved manner. Whilst applying for libraries, museums are willing to involve people instead of merely providing information. Many Museums therefor employ guides, who give tours and tell visitors about the exhibits. In addition to their factual knowledge, they also provide interesting anecdotes and other exciting information needed to bond with a certain topic. Apart of instructive and teaching staff, museums have tried many other ways to involve their visitors. One of those is employing technology. With time technology evolved, and so did technological augmentations in museums.
\\
It started with panoramas, dioramas and later simple mechanics, which moved models. After that basic electronics were included, which illuminated particular exhibits. Microchips and computers became more and more popular and affordable. So, the technological equipment of museums grew with what was available. Another chapter was opened, when the internet and wireless communication were introduced. Burgard et al. build an autonomous tour-guide robot and  called it \textit{RHINO}. It was able to navigate through the museum freely and without bumping into visitors. It could be used as a tour-guide for present visitors as well as for visitors on the internet, because it had a simple build-in web interface [Bur98]. RHINO was deployed at the \textit{Deutsches Museum Bonn} in 1998.
\\
In 2002, a group from the \textit{University of Limmerick} made a survey in \textit{Hunt Museum}. The museum is owned and run by the Hunt-family. Its tradition is to involve the visitors since its early days. Therefor, they had so-called \textit{cabinets of curiosity} [Cio02], special compartments within the exhibition, where additional exhibits were hidden. For example, a curious visitor had to open drawers in order to find a collection of plates. Via this exploration, the visitors became involved. Inspired by their observations, Ciolfi et al. implemented a completely new and interactive part of the exhibition in 2005. Two new rooms were introduced. First, there was the \textit{study room} with three interactive devices for getting further information about certain exhibits. They were disguised as a chest, a painting and a desk. The second room, the \textit{room of opinion}, was plain white with plinths, on which visitors could record their interpretations of the intended function of certain exhibits. In  order to manage all the data, a third and hidden room was used to host all the data-servers [Cio05].
\\
Something about [Hor06].

%\paragraph{Annotations}
%
%\begin{itemize}
	%\item Historical evolution
	%\begin{itemize}
		%\item Museums are believed to be old fashioned
		%\item Mostly willing to experiment (Examples)
		%\begin{itemize}
			%\item Dioramas
			%\item ...
			%\item Animatronics
			%\item Robotics
		%\end{itemize}
	%\end{itemize}
%\end{itemize}
%
%------------------------------------------------------------------------------------------
%
%\subsection*{Old version}
%
%Museums, much like libraries, are foremost seen as a place of knowledge and its preservation. Hence, visitors behave in a very reserved manner. Whereas this may apply for a library, museums are willing to involve people instead of merely providing information. Many Museums therfore employ guides, who give tours and tell visitors about the exhibits. In addition to their factual knowledge, they can also provide anecdotes and other information needed to bond with a certain topic. Apart of instructive and teaching staff, museums have tried many other ways to involve their visitors more. One of those is employing technology. With time technology evolved, and so did technological augmentations in museums.
%\\
%It may have started with simple mechanics, which moved some models, and later included basic electronics, which illuminated particular exhibits. Microchips and computers became more and more popular and affordable. So, the next step was immanent. There were info-terminals (...) Yet another chapter was opened, when the internet and wireless communication were introduced. Burgard et al. build an autonomous tour-guide robot called RHINO. It was able to navigate through the museum freely without bumping into visitors. RHINO could be used as a tour-guide for present visitors as well as for visitors on the internet, for it had a simple build-in and a web interface [Bur98]. RHINO was deployed at the “Deutsches Museum Bonn” in 1998.
%\\
%In 2002, a group from the University of Limmerick made a survey in the Hunt Museum. The
%museum is owned and run by the Hunt family, whose tradition it was from the beginning to involve the visitors. Therefore, they had so-called cabinets of curiosity [Cio02], special compartments within the exhibition, where additional exhibits were hidden. For example, one had to open drawers in order to find a collection of plates. Via this exploration, the visitors became involved. Inspired by their observations, Ciolfi et al. implemented a completely new and interactive part of the exhibition in 2005. Two new rooms were  introduced. First, there was a study room with three interactive devices for getting further information about certain exhibits. They were disguised as a chest, a painting and a desk. The second room, the room of opinion, was plain white with plinths, on which visitors could record their interpretations of intended function of certain exhibits. In  order to manage all the data, a third and hidden room was used to host all the data-servers [Cio05].
%\\
%Something about [Hor06].

%------------------------------------------------------------------------------------------

\section{Public single-user interfaces}
\label{motivation_single}

\paragraph{Annotations}

\begin{itemize}
	\item Human behavior concerning public interfaces
	\begin{itemize}
		\item self-service at train-stations
		\item public interfaces, such as Tobias Fischer's \textit{SMS-Schleuder für Fassaden}
		\item Intuitive usage vs. inhibition
	\end{itemize}
\end{itemize}

\section{Tangible Interfaces}
\label{related_work_tangible}

\paragraph{Annotations}

\begin{itemize}
	\item Technologies for input / interaction
	\item Hands-free
	\item Gestural interaction (Kinect)
\end{itemize}

%------------------------------------------------------------------------------------------

\section{Virtual Reality}
\label{motivation_vr}

\paragraph{Annotations}

\begin{itemize}
	\item Input
	\begin{itemize}
		\item Metaphors and devices
		\begin{itemize}
			\item Navigation and selection in 3d space
			\item Possibilities
			\item Difficulties
			\item Constraints
		\end{itemize}
	\end{itemize}
	\item Output
	\begin{itemize}
		\item Ordinary screen
		\item Stereoscopic displays
		\item \ac{HMD} such as Oculus Rift
	\end{itemize}
\end{itemize}

%------------------------------------------------------------------------------------------

\section{Goal}
\label{motivation_goal}

\paragraph{Annotations}

\begin{itemize}
	\item 'What did I want to do?'
\end{itemize}