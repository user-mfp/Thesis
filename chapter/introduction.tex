\chapter{Introduction}
\label{introduction}
%\textbf{''In short: Form matters.}'' -- Bellare and Rogaway\newline%\cite{Bellare04thegameplaying}}\newline
\begin{center}
\textbf{''All modern cryptographic systems are breakable in principle; it is just a question of how long it takes.''} -- Bellare and Rogaway from \cite{Bellare05}\newline
\end{center}

\section*{Security}
Security is a criterion for how well some object is steeled against any form of aggressors. The term security is present in nearly all areas of the world
we are living in. If you want to protect your home against intruders; if you want to avoid people stealing things from a shopping center; if you want
your child to be safe in school; if you want to save your country from wars or other political
disagreements or if you want to see your money safe and untouchable for others, it is all security.

This is just a small choice of realms where security is anchored and necessary. With the arrival in the computer age, IT security was born,
which includes the need to keep third parties from reading your secret information, to save your applications from crashing or being destroyed or to
install a network which should keep people from hacking your servers. And finally, the area of cryptography is anxious to increase security in
all areas where it is necessary.

\section*{Security and Cryptography}
Cryptography -- the science of hiding information in practice and theory -- is a very large field. The techniques developed in cryptography
are used to assure that some confidential data stays confidential and no adversary can reach them. Security of data, authenticity of an author
and the integrity of data are also goals of cryptography. To be sure that some cryptographic system is secure, a proof of its security against
an imaginable adversary or more than one has to be presented. Thus, assuming such a proof exists, a system can be described as being provably secure.
A security proof is often done by assuming that a specific adversary to the system exists and the proof shows that the success probability of
this adversary is for example smaller than some chosen threshold.

The statement from Bellare and Rogaway given above describes some fundamental knowledge about cryptographic systems. If an adversary has access to
infinite resources like time or storage it can break the most of the existing cryptographic systems. This is obvious, this is true, but this does not conform to
the real world. An adversary is always restricted by its resources and so security proofs are mostly done by upper bound on the probability that some given
adversary is able to break some given construct. A system is called secure if no practical adversary to this system exists.
A counterexample, which cannot be broken with infinite resources is the One-Time-Pad (OTP) \cite{DBLP:books/sp/voecking2011/Tantau11}, if it is used correctly,
i.e. a new key -- which has the same length as the message -- is used for every new encryption.
The OTP was introduced by Gilbert Vernam in 1918 and applied the first time by Joseph O. Mauborgne.

\section*{Game-Playing Proofs}
This master thesis deals with a special case of security proofs called game-playing proofs. ''In our opinion, many proofs in cryptography have become
essentially unverifiable. Our field may be approaching a crisis of rigor [$\hdots$] game-playing may play a role in the answer.''
This cite is from \cite{DBLP:conf/eurocrypt/BellareR06} and the authors claim that game-playing proofs are usually less error-prone and better structured
than normal proofs. Thus, they give a great opportunity to increase the understanding of security proofs.

When we began to learn the concepts of game-playing proofs, we realized that the given descriptions and proofs have a very steep learning curve.
So it was quite hard to understand the complex methodology due to the lack of a tutorial. To us, there was a need of an introduction
that helps us and other developing a solid understanding of the concept of game-playing proofs. The examples given in \cite{DBLP:conf/eurocrypt/BellareR06}
are from our point of view too complex to introduce game-playing and thus we decided to give some basic examples to you which are easier to understand
while showing the basic techniques applied in the game-playing scenario. After reading this thesis one should be able to understand
some more complex examples like the proof for the CBC MAC or the triple encryption given in \cite{DBLP:conf/eurocrypt/BellareR06} and
should be able to generate first examples on one's own. To convince the reader to use game-playing proofs instead of normal proofs we give an example
of these two approaches in the proof of the PRP/PRF switching lemma in Chapter \ref{PRP/PRF}. Furthermore, the whole thesis should suggest the
well chosen structure and the simplicity of game-playing proofs in comparison to normal proofs to the reader.

Everyone knows what a game is. And games can have alot of different characteristics and numbers of players. In the scenario of game-playing proofs,
two players are usually considered, an adversary $A$ and a challenger $C$; and a game $G$ is constructed in regard to some rules and restrictions given
to both the challenger $C$ and the adversary $A$. The challenger itself provides the game $G$ to $A$. An example for such a challenge can be given by
describing the \textit{shell game}. As we can see in Figure \ref{fig:shell_game} the challenger $C$ provides the game $G$ by hiding a small ball under one out of
three shells and shuffling in front of the player, i.e., the adversary $A$. After $C$ stops, $A$ has to decide, under which shell the ball lies.

%\begin{center}
%\begin{figure}[htb!]
%\centering
    %\psfrag{Adversary A}[cc][cc]{Adversary $A$}
    %\psfrag{Challenger C}[cc][cc]{Challenger $C$}
    %\psfrag{Choice}[cc][cc]{Choice}
    %\psfrag{Game G}[cc][cc]{Game $G$}
    %\psfrag{hide ball and shuffle}[cc][cc]{hide ball / shuffle}
    %\psfrag{observe}[cc][cc]{observe}
    %\psfrag{Success/Fail}[cc][cc]{Success/Fail}
    %\includegraphics[scale=0.4]{../pics/shell_game}
%\caption{The shell game considering an adversary $A$ and a challenger $C$.}
%\label{fig:shell_game}
%\end{figure}
%\end{center}

In this scenario, considering an adversary which has only one chance to find the ball, $A$ has a success probability of at least $1/3$ under the assumption
that the challenger $C$ is an honest player. The probability can be higher than $1/3$, if the power of observation of the adversary is taken into consideration.
This example can be extended to a so called \textit{chain of games}. A \textit{chain of games} is generated by transforming the orginal game (in this case the
game described as above) which leads to a new game $H$. A transformation of Game $H$ will lead to another game $I$ and so on. At the end of a \textit{chain of games},
a terminal game is placed. The terminal game is usually reducable to a mathematic hard problem and thus, the success probability of an adversary in the context
of the terminal game is negligible. A chain can also be defined the other way around, s.t. the chain ends with the original game.
For our example a new game $H$ can be achieved by adding another shell. Considering again an adversary $A$ with only one chance to find the ball,
its success probability for game $H$ is at least $1/4$.

\section*{Outline}
This guide provides the basic knowledge about game-playing in the first and second chapter by introducing the game-playing technique itself, showing some
widely spread examples and introducing pseudorandom functions and permutations, respectively.
Chapter \ref{chap:playing_techniques} and \ref{chap:hopping_techniques} describe some techniques to build up a game and to generate a \textit{chain of games}.
Beneith the definition of these techniques we are showing easy to understand examples, where these techniques can be applied in games.
Chapter \ref{chap:FAQ} concludes this tutorial with a question-and-answer part with the most frequently asked questions that arose when
giving this thesis to some of our fellow students.


