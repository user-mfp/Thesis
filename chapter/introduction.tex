\chapter{Introduction}
\label{introduction}

In the beginning, there only was a raw concept of collaboration with a local museum to develop an innovative museum installation. The installation would be interactive and based on in- and output modalities that actually make sense in a museum. Therefore, visitors and staff should be observed and interviewed. Further, it was planned to use Microsoft Gadgeteer-hardware\footnote{Microsoft Gadgeteer is a modular system of various hardware-components distrubuted by GHI Electronics. It resembles Arduino- and other microcontrollers.} as embedded components of tangible devices.
%\begin{center}
	%\textit{''Developing an innovative museum installation for and with a local museum.''}
%\end{center}
%Previous specifications
%\begin{itemize}
	%\item interactive museum installation
	%\item Gadgeteer (embedded hardware)
	%\item observations and interviews
	%\item input and output modalities that make sense in a museum
	%\item prototype
	%\item interactive design process
%\end{itemize}
I looked out for suitable museums in Weimar and found some interested ones. Later, I narrowed them down to a single one, which had the most fitting properties and attitude. Together, we conceived some ideas for possible installations. Not all of them were applicable or to far off my expertise. Nevertheless, there were two concepts, we were very interested in and excited about.
\\
The first concept, \textit{''Interaction with Tangibles''}, was directly addressing the visitors' haptic perception. A number of reproductions could be placed outside the showcase. Each somehow interactive tangible could then be manipulated or placed on a pedestal to gain information about its corresponding exhibit. Here, certain exhibits could have been photogrammetically scanned in three dimensions. After that, the digital model could be scaled to a handy size and otherwisely modified. Ultimately, the tangible could be printed or casted. Such an object could then be enhanced by using \ac{RFID}-technology\footnote{RFID-transponders or -tags do not require any batteries, are cheap and robust. In addition, their range is very limited, which allows several tags on one tangible.}. In order to make it interactive, it would be fitted with such a \ac{RFID}-tag. There is a \ac{RFID}-module for Gadgeteer, which would have allowed identification of each tangible. The corresponding information could then be provided by any medium compatible with Gadgeteer.
\\
A completely opposing approach was based on an assumption of natural behavior of visitors. After a meeting at the museum, a second concept of \textit{''Interaction by Pointing''} emerged. Later the underlying assumption was confirmed by the observation of visitors' behavior around showcases. Whereupon, visitors do not only talk about exhibits, but they also point at certain exhibits during interaction with each other. Therefore, a device should be build or utilized to point into the showcase and select a certain exhibit. Additional information of it would then be displayed in an appropriate manner. 
\\
These concepts were fairly comprehensible, but their exact technical realizations were not this clear, yet. Throughout further investigations, the work turned from testing various modes of input to a more technical approach. Both ways of input are fairly special and revealed different challenges along the way.

%-----------------------------------------------------------------------------

Throughout the following chapters I documented my proceedings during the development of the aforementioned system. Chapter \ref{motivation} gives basic information about the fields of study, which are included in my work. Thus, there is a brief outline about the progression of technologies employed by museums, users behavior around public interfaces and with tangibles. In addition, a brief overview of virtual reality-techniques is given.
\\
Afterwards, I describe my goals for the development of this system. Before I come to explain the schematics and evaluation of my implementations, I give a short review of my partnering process. Thus, chapter \ref{partnering} deals with finding the fitting museum for a collaboration. 
\\
Conception explains the whole development-process of the system's schematics and functionality. It begins with possible system designs and explains their possibilities and constraints. In the end of chapter \ref{conception}, the final concept is shown along with necessary obligations such as an \ac{FSD} and the contract between me, the university and the museum.
\\
Chapter \ref{implementation} addresses the implementation of the system's functionality. Therefore, all libraries and softwares are explained in more detail.
\\
Experimental lab-installations and the final museum-installation are described in chapter \ref{installation}. Therefore, measurements, hardware specifications, and other influential criteria are presented in detail.
\\
The final installation is evaluated in chapter \ref{evaluation}, where visitors were observed and interviewed before and after alterations by the system. Chapter \ref{discussion} then deals with the discussion of the evaluation's findings.
\\
In the end, I thought about future work, which could improve, extend, and follow my system. In chapter \ref{future_work}, I would also like to mention reactions and suggestions I encountered along my work.

%\paragraph{Introduction - Annotations}
%
%\begin{itemize}
	%\item Short overview, about what has been build
	%\item Summary
	%\\
	%\item System of libraries for pointing interaction
	%\item Information system (Information On Demand)
	%\item 'Uncharted territory' $\to$ technical focus
	%\item Template solution / 'just a proof of concept'
  %\\
	%\item Motivation
	%\item Working within the confines of museums respectively public installations
%\end{itemize}