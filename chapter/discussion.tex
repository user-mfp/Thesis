\chapter{Discussion}
\label{discussion}

\paragraph{The Samples}

Gruppengröße- und anzahl, sowie durchschnittliche Mitgliederzahl
\\
(Keine) Vergleichbarkeit: Gleiches Durchschnittsalter, aber bimodale Verteilung der 1. Stichprobe gegen modale Verteilung der 2. Stichprobe
\\
1. Stichprobe war unbiased; 2. Stichprobe wusste lediglich, dass es um die Evaluierung des Systems geht. Nicht jedoch, dass es auch im die Abfrage der Awareness geht. -> Weniger biased aber motivierter. <- Auch, weil die Probanden direkt ins 2. OG vorgeschickt worden sind. Weniger Ermüdung. Alles in allem aber beide Stichproben kein bis wenig Vorwissen über díe Theatik des Vitrineninhalts. 

%-----------------------------------------------------------------------------

\paragraph{Observed Interactions}

Interaktion bzw. Umgang mit Haßleben-showcase
\\
Interaktion bzw. Umgang untereinander

%-----------------------------------------------------------------------------

\paragraph{Interviews}

''Can you remember the grave of the princess of Haßleben?''
\\
''What can you remember? -- What objects were on display?''
\\
''What is, in your opinion, shown in the image?''
\\
''What would you change (positive or negative)?''
\\
''What were you especially interested in? What would you like to know more about?''
\\
''Did you read the grave's explanatory text?''
\\
''On what occasions do you usually visit museums and how often?''

%-----------------------------------------------------------------------------

\paragraph{Usability}

What does the outcome of the \ac{SUS}-questionnaire mean?

%-----------------------------------------------------------------------------

\paragraph{Conclusion}

The \ac{IMI}-system reliably works in a real world environment and on a daily basis. Hence, the developed system complies to our initial ambitions.

The Interaction with free-hand pointing gestures is as intuitive as estimated. Visitors that were observed did not show great shyness or restraint to use the system. Because the \ac{IMI}-system is an augmentation and not a fundamental part of the showcase, the exhibition is not disturbed. 


%\paragraph{Annotations}
%
%\begin{itemize}
	%\item Conclusions
	%\begin{itemize}
		%\item Comparison to Conception
		%\item Comparison to 'Pflichtenheft' see \textit{Ref: Appendix}
	%\end{itemize}
	%\item Anecdotes
	%\begin{itemize}
		%\item Very short short-time memory $\to$ Instruction-sticker
		%\item Misconception of screen an a simple video and no interaction
		%\item Inhibitional factors (shyness, frustration, being watched)
	%\end{itemize}
%\end{itemize}