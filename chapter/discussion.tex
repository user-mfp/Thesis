\chapter{Discussion}
\label{discussion}

The pre- and main study were conducted during a period of five days. They both compare the awareness of visitors concerning the Haßleben-showcase and its exhibits. Therefore, visitors were observed during their time around the showcase and interviewed afterwards. The questions aimed at the participants \textit{stages of awareness} introduced in the previous Chapter.
\\
Moreover, a post-study was evaluated to gain an idea of how the \ac{IMI}-system was used by visitors when they did not feel monitored. Furthermore, the stored data can give an insight into necessary improvements of the current \ac{IMI}-exhibition of the Haßleben-showcase.

\paragraph{Samples and Comparability} During the pre-study 53 visitors participated in the interviews. They were distributed over 19 groups. That is an average group size of about 2.8 visitors per group. Their average age was 24.33 years. The properties of average age and group size of pre-study and main study are approximately the same. In the main study, 58 participants from 32 groups were participating. The average group size of 1.8 was considerable lower. However, the age was nearly identical. The average age of participants from the main study was 25 and hence only 6 months above the value of the pre-study.
\\
Nevertheless, the studies can not be treated as a between-subjects test. Reason for this restriction is the distribution of the participants' ages. Both samples do have the same average age, yet the distribution of the ages varies. While the sample of the main study is normally distributed, the sample of the pre-study is not. Here, the distribution is bimodal. This means that both might have a similar average age, but the reason for this fact is different. Hence, the samples are not statistically comparable when it comes to that particular criteria. The distributions of the pre-study and main study can be seen in Figure \ref{fig:discussion_age-distribution}.

\textbf{F I G U R E}
%\begin{figure}%
%\includegraphics[width=\columnwidth]{filename}%
%\caption{.}%
%\label{fig:discussion_age-distribution} %Graph der transitions zwischen den 9 targets
%\end{figure}

The majority of all casual visitors and invited participants of both studies were on an identical level of knowledge about the Haßleben-showcase. Thus, their engagement with the \ac{IMI}-system can be seen as impartial. There were sufficiently less casual visitors among the sample of the main study and, therefore, more technical experienced participants. Those pre-recruited participants might have been less restrained in using novel technologies. Their technical expertise, however, was of little use, because no devices had to be operated and the \ac{GUI} of the presentation software had not been shown to anyone prior to the main study. Hence, all participants, casual and invited, had to rely on their physical capabilities alone. Furthermore, only a few had prior knowledge of the contents of the \ac{IMI}-exhibition inside the Haßleben-showcase. Consequently, the answers of both samples can be seen as equally impartial, while the interaction of several pre-recruited participants is certainly more experienced.

%-----------------------------------------------------------------------------

\paragraph{Observed Interactions}

Interaktion bzw. Umgang mit Haßleben-showcase
\\
Interaktion bzw. Umgang untereinander

%-----------------------------------------------------------------------------

\paragraph{Interviews}

''Can you remember the grave of the princess of Haßleben?''
\\
''What can you remember? -- What objects were on display?''
\\
''What is, in your opinion, shown in the image?''
\\
''What would you change (positive or negative)?''
\\
''What were you especially interested in? What would you like to know more about?''
\\
''Did you read the grave's explanatory text?''
\\
''On what occasions do you usually visit museums and how often?''

%-----------------------------------------------------------------------------

\paragraph{Usability}

What does the outcome of the \ac{SUS}-questionnaire mean?

%-----------------------------------------------------------------------------

\paragraph{Conclusion}

The \ac{IMI}-system reliably works in a real world environment and on a daily basis. Hence, the developed system complies to our initial ambitions.

Interaction by free-hand pointing gestures is as intuitive as estimated. Visitors that were observed did not show great shyness or restraint to use the system. Because the \ac{IMI}-system is an augmentation and not a fundamental part of the showcase, the exhibition is not disturbed. With low cost and little effort, the \ac{IMI}-system is able to augment a showcase of an exhibition or other presentable setups.

%\paragraph{Annotations}
%
%\begin{itemize}
	%\item Conclusions
	%\begin{itemize}
		%\item Comparison to Conception
		%\item Comparison to 'Pflichtenheft' see \textit{Ref: Appendix}
	%\end{itemize}
	%\item Anecdotes
	%\begin{itemize}
		%\item Very short short-time memory $\to$ Instruction-sticker
		%\item Misconception of screen an a simple video and no interaction
		%\item Inhibitional factors (shyness, frustration, being watched)
	%\end{itemize}
%\end{itemize}