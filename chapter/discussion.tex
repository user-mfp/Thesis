\chapter{Discussion}
\label{discussion}

The pre- and main study were conducted during a period of five days. They both compare the awareness of visitors concerning the Haßleben-showcase and its exhibits. Therefore, visitors were observed during their time around the showcase and interviewed afterwards. The questions aimed at the participants \textit{stages of awareness} introduced in the previous Chapter.
\\
Moreover, a post-study was evaluated to gain an idea of how the \ac{IMI}-system was used by visitors when they did not feel monitored. Furthermore, the stored data can give an insight into necessary improvements of the current \ac{IMI}-exhibition of the Haßleben-showcase.

\paragraph{Samples and Comparability} During the pre-study 53 visitors participated in the interviews. They were distributed over 19 groups. That is an average group size of about 2.8 visitors per group. Their average age was 24.33 years. The properties of average age and group size of pre-study and main study are approximately the same. In the main study, 58 participants from 32 groups were participating. The average group size of 1.8 was considerable lower. However, the age was nearly identical. The average age of participants from the main study was 25 and hence only 6 months above the value of the pre-study.
\\
Nevertheless, the studies can not be treated as a between-subjects test. Reason for this restriction is the distribution of the participants' ages. Both samples do have the same average age, yet the distribution of the ages varies. While the sample of the main study is normally distributed, the sample of the pre-study is not. Here, the distribution is bimodal. This means that both might have a similar average age, but the reason for this fact is different. Hence, the samples are not statistically comparable when it comes to that particular criteria. The distributions of the pre-study and main study can be seen in Figure \ref{fig:discussion_age-distribution}.

\textbf{F I G U R E}
%\begin{figure}%
%\includegraphics[width=\columnwidth]{filename}%
%\caption{.}%
%\label{fig:discussion_age-distribution} %Graph der transitions zwischen den 9 targets
%\end{figure}

The majority of all casual visitors and invited participants of both studies were on an identical level of knowledge about the Haßleben-showcase. Thus, their engagement with the \ac{IMI}-system can be seen as impartial. There were sufficiently less casual visitors among the sample of the main study and, therefore, more technical experienced participants. Those pre-recruited participants might have been less restrained in using novel technologies. Their technical expertise, however, was of little use, because no devices had to be operated and the \ac{GUI} of the presentation software had not been shown to anyone prior to the main study. Hence, all participants, casual and invited, had to rely on their physical capabilities alone. Furthermore, only a few had prior knowledge of the contents of the \ac{IMI}-exhibition inside the Haßleben-showcase. Consequently, the answers of both samples can be seen as equally impartial, while the interaction of several pre-recruited participants is certainly more experienced.

%-----------------------------------------------------------------------------

\paragraph{Observed Interactions}

The first value that was observed about groups around the Haßleben-showcase was the \ac{LOS}, the duration visitors and participants were addressing the showcase. The average time visitors from the pre-study spent with the showcase was around one minute. The longest stay that was observed did not last longer than two minutes, while the shortest was between 0 and 30 seconds long. The shorted session during the main study was 48 seconds and the longest session took 13:11 minutes. On average the \ac{LOS} was 5:25 minutes. However, the pre-recruited participants were present to evaluate the presentation software and, therefore, stayed longer and were more engaged with the \ac{IMI}-system. Hence, the average and maximum \ac{LOS} of this sample is so much longer. Nevertheless, the post-study was conducted under daily circumstances and it revealed that the average \ac{LOS} was 1:34 minutes. This means an increase of the average \ac{LOS} of about 50$\%$. The difference between the longest (11:43 minutes) and shortest (11 seconds) stay was similar to that of the main study, whereas the observations from the pre-study only showed a small range of only two minutes. Thus, the shortest stay could not be improved by much, but the longest stay was nearly increased sixfold.
\\
In conclusion to these observations, visitors spent significantly more time with the Haßleben-showcase than before. Hence, the visitors are more engaged with the Haßleben-showcase. This should raise their awareness of its existence, as stated in Chapter \ref{evaluation}.

During the pre-study visitors perceived the Haßleben-showcase like any other showcase of the museum. They approached it and looked at the exhibits inside the showcase. Visitors did that from the broad and narrow side, whereas the narrow side was used four times less than the broad side. Seven groups out of 19 read the explanatory text. One visitor seemed to think about something and tried to look it up in the text. Some visitors went past the showcase or only glanced at its contents. The main study introduced the \ac{IMI}-system to the showcase and all visitors and participants were looking into the grave. The display drew their interest and they started interacting with the \ac{IMI}-system. Nearly every participant gave it a try and started pointing. Thereby, some issues arose. The main observation was that half of the 32 groups were over-fixated on the visual feedback given by the navigational view of the presentation software. Thus, they did not use the feedback to fine tune their pointing, but completely relied on it. Because the display was raised and not in their direct view on the exhibition plane, the feedback positions of the users began to tremble. The movement of their head to look up had changed their aiming position and consequently the feedback position as well. Additionally, some participants perceived the interface to be more natural than it was, and pointed with their left arm or tried other gestures such as a swiping move to change the images during the presentation of an \ac{IMI}-exhibit. Another fact that could be observed was that the readability was compromised by two factors. The letters or the display were too small and previously calculated time for reading was too short.
\\
Interaction of visitors during the post-study was not observable. Nevertheless, the amount of active sessions shows that the \ac{IMI}-system is used on a day to day bases by the normal audience of the museum. The quote of about 1:1 between active and empty sessions, however, does not have to necessarily mean that only 50$\%$ of all visitors take notice of the exhibits inside the Haßleben-showcase. If the durations of empty sessions would be evaluated as well, they will show for how long non-interacting visitors stay around the showcase. There \ac{LOS} might also be longer than before.
\\
Conclusively, it can be said that the presence of the \ac{IMI}-system has increased the engagement of visitors with the Haßleben-showcase and the \ac{IMI}-exhibits inside it. There are indicators for issues that need to be addressed to improve the interaction of visitors with the presentation software. 

After the augmentation of the Haßleben-showcase with the \ac{IMI}-system, visitors were more engaged with the exhibition. This involvement also influenced the interaction between visitors. Thus, their behavior among each other changed as well. During the pre-study, visitors moved in closed groups, clustered in particular places or walked through the museum floor individually. 31 out of the 51 visitors that were observed moved as a closed group. That is about 60.8$\%$ of all visitors. 25.5$\%$ moved individually and clustered in particular places and the remaining 5.7$\%$ (seven visitors) were moving detached from their group. In the main study only two cases of individual movement were observed. Moreover, verbal interaction within the groups increased. 15 cases of verbal interaction by 17 groups of visitors of the pre-study were observed. Meanwhile, 18 cases of talking, explaining, discussing, whispering and reading out loud were observed among the 17 groups. Lone visitors and participants were not taken into consideration, although at least one lone user was overheard thinking aloud during the main study.
\\
In summary, visitors and participants of the main study showed more interaction overall. Modalities of the interaction with the \ac{IMI}-system were a topic, but also the contents of the presentations were discussed. Altogether, the possibility of interacting with the \ac{IMI}-system increases the engagement of visitors with the \ac{IMI}-exhibits inside the Haßleben-showcase and promotes the interaction within a group. Visitors are more active. 

%-----------------------------------------------------------------------------

\paragraph{Interviews} Visitors of both the pre- and the main study were asked if they would remember the grave of the princess of Haßleben. Nine groups from the pre-study did remember the Haßleben-showcase correctly, whereas the rest did not, recalled a wrong showcase or did not take part in the interview. This means that 60$\%$ correctly remembered the showcase they had seen a few moments earlier. During the main study, 31 out of 32 interviewed groups recalled the Haßleben-showcase. This confirms the aforementioned first stage of awareness, which refers to the awareness of a showcase's existence. Hence, the engagement through interaction with the \ac{IMI}-exhibits inside the showcase increased the visitors' awareness of it. 
\\
Unfortunately, this conclusion can not confirmed. Only four of the 32 groups of the main study were not previously informed, what the study was about. The remaining 28 groups were either invited or museum staff. It can not be said with certainty how many of the positive answers were given under the influenced of the invitation itself. In order to get a valid and comparable result to this question, casual visitors will have to be asked after leaving the area of the Haßleben-showcase.

''What can you remember? -- What objects were on display?''
\\
''What is, in your opinion, shown in the image?''
\\
''What would you change (positive or negative)?''
\\
''What were you especially interested in? What would you like to know more about?''
\\
''Did you read the grave's explanatory text?''
\\
''On what occasions do you usually visit museums and how often?''

%-----------------------------------------------------------------------------

\paragraph{Usability}

What does the outcome of the \ac{SUS}-questionnaire mean?

%-----------------------------------------------------------------------------

\paragraph{Conclusion}

The \ac{IMI}-system reliably works in a real world environment and on a daily basis. Hence, the developed system complies to our initial ambitions.

Interaction by free-hand pointing gestures is as intuitive as estimated. Visitors that were observed did not show great shyness or restraint to use the system. Because the \ac{IMI}-system is an augmentation and not a fundamental part of the showcase, the exhibition is not disturbed. With low cost and little effort, the \ac{IMI}-system is able to augment a showcase of an exhibition or other presentable setups.



%\paragraph{Annotations}
%
%\begin{itemize}
	%\item Conclusions
	%\begin{itemize}
		%\item Comparison to Conception
		%\item Comparison to 'Pflichtenheft' see \textit{Ref: Appendix}
	%\end{itemize}
	%\item Anecdotes
	%\begin{itemize}
		%\item Very short short-time memory $\to$ Instruction-sticker
		%\item Misconception of screen an a simple video and no interaction
		%\item Inhibitional factors (shyness, frustration, being watched)
	%\end{itemize}
%\end{itemize}