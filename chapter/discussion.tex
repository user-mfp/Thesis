\chapter{Discussion}
\label{discussion}

The pre- and main study were conducted during a period of five days. They both compare the awareness of visitors concerning the Haßleben-showcase and its exhibits. Therefore, visitors were observed during their time around the showcase and interviewed afterwards. The questions aimed at the participants' \textit{stages of awareness} introduced in the previous chapter.
\\
Moreover, a post-study was evaluated to gain an idea of how the \ac{IMI}-system was used by visitors when they did not feel monitored. Furthermore, the stored data can give an insight into necessary improvements of the current \ac{IMI}-exhibition of the Haßleben-showcase.

\paragraph{Samples and Comparability} During the pre-study 53 visitors participated in the interviews. They were distributed over 19 groups. That is an average group size of about 2.8 visitors per group. Their average age was 31.51 years. The properties of average age and group size of pre-study and main study are approximately the same. In the main study, 58 participants from 32 groups were participating. The average group size of 1.8 was considerable lower. The age was nearly identical. The average age of participants from the main study was 31.43 year. Participants of the pre-study were 31.51 years on average. However, the median of the pre-study was 33 years of age, whilst the main study's participants had a median age of 28 years. A juxtaposition of the two age distributions can be seen in Figure \ref{fig:discussion_age-distribution}.

\begin{figure}[H]%
\includegraphics[width=\columnwidth]{../pics/both-age.eps}%
\caption{Age distribution of participants from the pre- and main study.}
\label{fig:discussion_age-distribution} 
\end{figure}

The studies can not be treated as a between-subjects test. Reason for this restriction is the distribution of the participants' ages. Both samples do have the same average age, yet the distribution of their ages varies. While the sample of the main study is mostly normally distributed, the sample of the pre-study is not. This distribution is trimodal, because the visitors were children, adults, and seniors. Meanwhile, participants of the main study were mostly students and research assistants. This means that both studies' participants might have had a similar average age, but the reason for this fact is different. Hence, the samples are not statistically comparable when it comes to that particular criteria. 

The majority of all casual visitors and invited participants of both studies were on an identical level of knowledge about the Haßleben-showcase. Thus, their engagement with the \ac{IMI}-system can be seen as impartial. There were sufficiently less casual visitors among the sample of the main study and, therefore, more technical experienced participants. Those pre-recruited participants might have been less restrained in using novel technologies. Their technical expertise, however, was of little use, because no devices had to be operated and the \ac{GUI} of the presentation software had not been shown to anyone prior to the main study. Hence, all participants, casual and invited, had to rely on their physical capabilities alone. Furthermore, only a few had prior knowledge of the contents of the \ac{IMI}-exhibition inside the Haßleben-showcase. Consequently, the answers of both samples can be seen as equally impartial, while the interaction of several pre-recruited participants is more experienced.

%-----------------------------------------------------------------------------

\paragraph{Observed Interactions}

The first value observed about groups around the Haßleben-showcase was the \ac{LOS}, the duration visitors and participants were addressing the showcase. Due to the way visitors were observed, a precise timing for each group was not possible. Hence, the \ac{LOS} was categorized in intervals of thirty seconds or by a minute. The average time visitors spent with the showcase was around one minute. The longest stay that was observed did not last longer than two minutes, while the shortest was between 0 and 30 seconds long. The shorted session during the main study was 48 seconds and the longest session took 13:11 minutes. On average the \ac{LOS} was 5:25 minutes. However, the pre-recruited participants were present to evaluate the presentation software and, therefore, stayed longer and were more engaged with the \ac{IMI}-system. Hence, the average and maximum \ac{LOS} of this sample is so much longer. Nevertheless, the post-study was conducted under daily circumstances and it revealed that the average \ac{LOS} was 1:34 minutes. This means an increase of the average \ac{LOS} of about 50$\%$. The difference between the longest (11:43 minutes) and shortest (11 seconds) stay was similar to that of the main study, whereas the observations from the pre-study only showed a small range of only two minutes. Thus, the shortest stay could not be improved by much, but the longest stay was increased nearly sixfold.
\\
In conclusion to these observations, visitors spent significantly more time with the Haßleben-showcase than before. Hence, the visitors are more engaged with the Haßleben-showcase. This should raise their awareness of its existence, as stated in Chapter \ref{evaluation}.

During the pre-study visitors perceived the Haßleben-showcase like any other showcase of the museum. They approached it and looked at the exhibits inside the showcase. Visitors did that from the broad and narrow side, whereas the narrow side was used four times less than the broad side. Seven groups out of 19 read the explanatory text. One visitor seemed to think about something and tried to look it up in the text. Some visitors went past the showcase or only glanced at its contents. The main study introduced the \ac{IMI}-system to the showcase and all visitors and participants were looking into the grave. The display drew their interest and they started interacting with the \ac{IMI}-system. Nearly every participant gave it a try and started pointing. Thereby, some issues arose. The main observation was that half of the 32 groups were over-fixated on the visual feedback given by the navigational view of the presentation software. Thus, they did not use the feedback to fine tune their pointing, but completely relied on it. Because the display was raised and not in their direct view on the exhibition plane, the feedback positions of the users began to tremble. The movement of their head to look up had changed their aiming position and consequently the feedback position as well. Additionally, some participants perceived the interface to be more natural than it was, and pointed with their left arm or tried other gestures such as a swiping move to change the images during the presentation of an \ac{IMI}-exhibit. Another fact that could be observed was that the readability was compromised by two factors. The letters or the display were too small and previously calculated time for reading was too short.
\\
Interaction of visitors during the post-study was not observable. Nevertheless, the amount of active sessions shows that the \ac{IMI}-system is used on a day to day basis by the regular audience of the museum. The quote of about 1:1 between active and empty sessions, however, does not have to necessarily mean that only 50$\%$ of all visitors take notice of the exhibits inside the Haßleben-showcase. When the durations of empty sessions would be evaluated as well, they should show for how long non-interacting visitors stay around the showcase. Their \ac{LOS} might also be longer than before.
\\
Conclusively, it can be said that the presence of the \ac{IMI}-system has increased the engagement of visitors with the Haßleben-showcase and the \ac{IMI}-exhibits inside it. There are indicators for issues that need to be addressed to improve the interaction of visitors with the presentation software. 

After the augmentation of the Haßleben-showcase with the \ac{IMI}-system, visitors were more engaged with the exhibition. This involvement also influenced the interaction between visitors. Thus, their behavior among each other changed as well. During the pre-study, visitors moved in closed groups, clustered in particular places or walked through the museum floor individually. 31 out of the 51 visitors that were observed moved as a closed group. That is 60.8$\%$ of all visitors. 25.5$\%$ moved individually and clustered in particular places and the remaining 5.7$\%$ (seven visitors) were moving detached from their group. In the main study only two cases of individual movement were observed. Moreover, verbal interaction within the groups increased. 15 cases of verbal interaction by 17 groups of visitors of the pre-study were observed. Meanwhile, 18 cases of talking, explaining, discussing, whispering and reading out loud were observed among the 17 groups. Lone visitors and participants were not taken into consideration, although at least one lone user was overheard thinking aloud during the main study.
\\
In summary, visitors and participants of the main study showed more interaction overall. Modalities of the interaction with the \ac{IMI}-system were a topic, but also the contents of the presentations were discussed. Altogether, the possibility of interacting with the \ac{IMI}-system increases the engagement of visitors with the \ac{IMI}-exhibits inside the Haßleben-showcase and promotes the interaction within a group.

%-----------------------------------------------------------------------------

\paragraph{Interviews} Visitors of both the pre- and the main study were asked if they would remember the grave of the princess of Haßleben. Nine groups from the pre-study did remember the Haßleben-showcase correctly, whereas the rest did not, recalled a wrong showcase or did not take part in the interview. This means that 60$\%$ correctly remembered the showcase they had seen a few moments earlier. During the main study, 31 out of 32 interviewed groups recalled the Haßleben-showcase. This confirms the aforementioned first stage of awareness, which refers to the awareness of a showcase's existence. Hence, the engagement through interaction with the \ac{IMI}-exhibits inside the showcase increased the visitors' awareness of it. 
\\
Unfortunately, this conclusion can not confirmed. Only four of the 32 groups of the main study were not previously informed, what the study was about. The remaining 28 groups were either invited or museum staff. It cannot be said with certainty how many of the positive answers were given under the influence of the invitation itself. In order to get a valid and comparable result to this question, casual visitors will have to be asked after leaving the area of the Haßleben-showcase.

The following question aimed at the next stage of awareness. All Participants were asked what they could remember about the Haßleben-showcase and what objects were on display. In both studies, two main categories included most of the objects inside the showcase. They are jewelry and everyday objects. During the pre-study all kinds of jewelry were remembered 26 times and 19 everyday objects were named. additionally, the skeleton itself was recalled 4 times. That is 39 objects in total by 14 groups taking part in the interview. For the same quota, participants of the main study would have had to remember a total of 89 objects from all categories and the skeleton. However, the participants of the main study recalled 178 objects. That is exactly twice as much as their predecessors. 75 jewelry-related and 61 everyday objects were named. In addition, the skeleton of the princess was named 16 times. This is four times more often than during the pre-study. Participants of the main study might have been more alert due to their motivation for being at the museum. Nevertheless, this could also be the effect of increased awareness caused be engagement with the \ac{IMI}-system.
\\
Moreover, participants remembered more \ac{IMI}-exhibits than other exhibits from the Haßleben-showcase. In total, 80 \ac{IMI}-exhibits were named by the 35 groups, whereas only 67 non-interactive exhibits were recalled. This is still more than the total amount of the pre-study. Furthermore, participants of the main study were able to name objects that were never mentioned by the previous groups. For instance, no one of the interviewed participants from the pre-study named the box brooch, silver plate, key to the jewel box or the skeleton of the dog. These \ac{IMI}-exhibits alone were named 44 times.
\\
Summarizing, participants of the main study were able to recall distinctly more exhibits than those of the pre-study. Pre-recruited participants were informed about the evaluation of the system, but not about the contents of the interview. They were prepared to test a novel kind of interaction. This statement was confirmed by a number of invited participants after the main study. Hence, it can be concluded that the engagement with the \ac{IMI}-system increased the participants' awareness of composition of the Haßleben-showcase.

In succession of the remembering task, all groups of participants were shown an image of the jewel case located by the feet of the princess of Haßleben and asked what they think it was. The necessary information was given by the explanatory text, the audio guide, and by the presentation software of the \ac{IMI}-system upon selection of the jewel box. Seven groups of the pre-study and 12 groups of the main study were able to name the object in the image correctly. That is about the same rate for both studies. However, only two groups of the pre-study identified the remains of the contents of the jewel box. During the main study, 11 groups managed to name at least the content of the jewel box.
\\
After all, the jewel box itself was equally as often recognized by participants of both studies. The content of the jewel case, however, was correctly identified by the groups of the main study about three times as often as by those of the pre-study. 

When participants where asked what they would change about the Haßleben-showcase, the answers varied. Participants of the pre-study asked for a map of the site of the find. There is a map of the site above the explanatory text. The three groups that gave this answer did not see it, though. The next topic was suitability for children. Parents criticized the height of the showcase. The visual angle at which small children look at the showcase does not allow a good overview. The most frequent answer was that everything was fine and nothing should be changed. This reaction either indicates a bias towards conformity with the current state of the Haßleben-showcase or participants were not motivated any more. This lack of motivation might have been induced by the time they had already spent at the museum beforehand. 
\\
Participants of the main study, however, were more critical. They addressed their issues with the \ac{IMI}-system. According to them, certain aspects of the feedback, readability and interaction should be improved. Furthermore, they also mentioned general aspects about the Haßleben-showcase. The visibility of exhibits was most commonly addressed. Occlusions, reflections, and other lighting-related issues were mentioned. Further, it was observed but never mentioned that small children were not recognized by the system. Hence, the suitability for children is another issue that needs improvement.
\\
In summary, participants of the pre-study were more concerned about topical facts, whereas participants of the main study gave more feedback about their experience with the \ac{IMI}-system.
       
After the general feedback, the interview got more precise and participants were asked what they were especially interested in and would like to know more about. Again, the groups of the pre-study gave rather general answers. The epoch and its procedures were mentioned seven times, more information about the whole gravesite of Haßleben were requested 18 times, and other topics were mentioned 17 times. The most frequently given answer was ''nothing''. Participants replied eleven times that they would not like to know anything more about the Haßleben-showcase.
\\
During the main study, participants would request further information about the historical significance of the findings eight times. Information about particular exhibits were mentioned thirteen times. Six of them were the skeleton of the dog an the box brooch. As mentioned above, those \ac{IMI}-exhibits were not even named by the groups of the pre-study, when the were asked to recall objects from the Haßleben-showcase. In addition, three groups asked for more \ac{IMI}-exhibits inside the showcase.
\\
In conclusion, the \ac{IMI}-system increased the awareness of the contents of the Haßleben-showcase. This is done to such an extent that several groups from the main study reached the third stage of awareness and requested more specific knowledge about certain exhibits. These \textit{exhibits of increased interest} were not only interactive, but also a comb that was recalled only once.

When participants were asked whether or not they had read the explanatory text, five members of groups of the pre-study and three of the main study did so. The rest did not read the related information about the gravesite of Haßleben. Hence, only a few participants were willing to read additional information. Nevertheless, as answers of the previous questions revealed, the participants of the main study were better informed about the Haßleben-showcase than those of the pre-study were.  

In the end, participants had to tell their usual reasons for visiting a museum. Here, the answers of groups of the two studies were alike. Both named special occasions and interests as main categories for their visits. Participants of the main study further referred to reduced fees as an appealing reason for visiting a museum.
\\
Finally, members of all groups were asked how often they do visit a museum. The participants of the pre-study were vague about their answers. ''Seldom'' was the major answer followed by no concrete answer. Only three groups gave a definite stretch of time. They stated to visit a museum once a year, biannually and three to four times a year. The participants of the main study were more precise and their answers raged from ''once a decade'' to ''monthly''. The majority, however, said they would visit museums two or three times a year.

%-----------------------------------------------------------------------------

\paragraph{Standard Usability Scale-Questionnaire} The \ac{SUS}-questionnaire was done by 36 participants of the main study. The presentation software of the \ac{IMI}-system installed inside the Haßleben-showcase achieved an overall score of 77.53$\%$. This is a good score and pleads for a good usability of the presentation software of the \ac{IMI}-system.
\\
The difference between the highest and lowest score is 21.94$\%$. However, the lowest score is 68.08$\%$. This is only a little below the threshold of 70$\%$, which implies a good value. The statement that caused the lowest score was the third: ''I thought the system was easy to use.'' This means that some users thought the presentation software was not easy to use. Four of them rated this statement with 30$\%$ and one user gave it 20$\%$. Without their verdict, the score would have been 74.52$\%$. At this point, it has to be mentioned, that the presentation software of the \ac{IMI}-system froze with two of those users. Since then, the error has been identified and fixed\footnote{The system unintendedly paused the \texttt{trackingThread} and did not resume it after the \texttt{lockTime} was over. Similar to the description of the definition and validation of a position in Chapter \ref{implementation_administration}, the thread responsible for tracking was aborted by a thread, that also aborted, before it could re-start the tracking thread.}. The ease of use could be further improved by addressing the feedback, participants gave regarding the visual feedback and readability of the \ac{IMI}-system.
\\
The best score of the \ac{SUS}-questionnaire was the tenth statement. It says: ''I needed to learn a lot of things before I could get going with this system.'' This statement was confirmed with a score of 90$\%$. Hence, the users felt the presentation software was very intuitive to use and did not require a lot of prior knowledge.

The \ac{SUS}-questionnaire is a \textit{quick and dirty}-method for gather information about the usability of a system. Moreover, it only represents the subjective perception of the usability. Nevertheless, it yielded promising results for possible further investigation of interaction via free-hand pointing gestures. 

%-----------------------------------------------------------------------------

\paragraph{Conclusion}

The \ac{IMI}-system reliably works in a real world environment and on a daily basis. Hence, the developed system complies to our initial ambitions.

The three stages of awareness described in Chapter \ref{evaluation} were recognizable with the participants of the main study. However, the groups of both studies were similarly treated and had about the same level of prior knowledge about the Haßleben-showcase. Concerning their background and motivation of visiting the museum during the days of the different studies were not comparable, though. Hence, further investigation of regular visitors is necessary to gain completely conclusive results.

Finally, interaction by free-hand pointing gestures is as natural and intuitive as previously estimated. Observed visitors not show great shyness or restraint to use the system. Because the \ac{IMI}-system is an augmentation and not a fundamental part of the showcase, the exhibition is not disturbed. With low cost and little effort, the \ac{IMI}-system is able to augment a showcase of an exhibition or other presentable setups.

%\paragraph{Annotations}
%
%\begin{itemize}
	%\item Conclusions
	%\begin{itemize}
		%\item Comparison to Conception
		%\item Comparison to 'Pflichtenheft' see \textit{Ref: Appendix}
	%\end{itemize}
	%\item Anecdotes
	%\begin{itemize}
		%\item Very short short-time memory $\to$ Instruction-sticker
		%\item Misconception of screen an a simple video and no interaction
		%\item Inhibitional factors (shyness, frustration, being watched)
	%\end{itemize}
%\end{itemize}