\chapter{Background and Motivation}
\label{motivation}

Over time public places became more and more enriched with all kinds of technology. Nowadays, on nearly every corner something is beeping or blinking and buttons, leavers and knobs make us - their potential users - interact with our environment. This trend does not spare anyone or anything. Even traditionally calm and sophisticated places open up to the possibilities of contemporary technologies.
\\


%-----------------------------------------------------------------------------

%\paragraph{Related Work - Annotations}
%
%\begin{itemize}
	%\item Backgrounds
	%\begin{itemize}
		%\item Historical
		%\item Technical
	%\end{itemize}
	%\item Application areas
	%\item Not to much detail
	%\item Only in respect to the thesis' topic
%\end{itemize}

%-----------------------------------------------------------------------------

\section{Museums}
\label{motivation_museums}

Museums, much like libraries, are foremost seen as a place of knowledge and its preservation. Hence, visitors behave in a very reserved manner. Whilst applying for libraries, museums are willing to involve people instead of merely providing information. Many Museums therefor employ guides, who give tours and tell visitors about the exhibits. In addition to their factual knowledge, they also provide interesting anecdotes and other exciting information needed to bond with a certain topic. Apart of instructive and teaching staff, museums have tried many other ways to involve their visitors. One of those is employing technology. With time technology evolved, and so did technological augmentations in museums.
\\
The name ''museum'' comes from the ancient greek's ''Museion''. It describes an, in honor of a muse, sanctified place~\cite{DeutscherMuseumsbundGeschichte}. Basically, museums are collections of arts and science or at least parts of them on display. In modern history, those collections were of an artistic nature and mostly private. Later, scientific and otherwise cultural museums were established for the general public. 
\\
One of the first high-tech installation of the modern age was the \textit{Diorama}. In 1821, Louis Jacque Mandé Daguerre\footnote{Daguerre is a scene painter and stage designer by trade. He also is the inventor of the first photographic process called daguerreotypy} and the painter Charles Marie Bouton partnered up to develop this spectacle. It is an elaborate combination of painting and lighting~\cite{DioramaDaguerre}. Through ingenious lighting, the paintings became vivid. This way, a diorama could simulate the moods of a whole day within minutes. Thus, it might be seen as an early predecessor of the cinema. Even today, although in much smaller size, dioramas are still thematised~\cite{DioramaSchneider}.
\\
The first interactive displays appeared at the \textit{Urania} in Berlin around 1889, when they introduces visitor-activated models and a scientific theater. In 1907 the \textit{Deutsches Museum} in Munich also began experimenting with film and mechanical models, which were operated by visitors~\cite{MuseumsHistoryMcLean}. Later, other museums all over the world followed. Since then the \begin{quote}[...] wider museological community's understanding of nature and purpose of interactiveness \end{quote} has taken shape. \begin{quote}This understanding almost invariably involves:
	\begin{enumerate}
		\item The presence of some technological medium.
		\item A physical exhibit which is added to the main display.
		\item A device which the visitor can operate, involving physical activity.~\cite{MuseumsHistoryWitcomb}
	\end{enumerate}
\end{quote}

As electronics and microchips evolved, computers became popular and affordable. The technological equipment of museums grew with what was available and new kinds of devices and installations appeared. Today, nearly every museum has a certain guide system such as an audio guide. It either leads visitors through the museum on a predefined course or a visitor can choose the track according to a given code for each included exhibit. In 2004, Chou et al.  compared different museum guide systems in various categories, which were considered necessary to provide a user-friendly and informative experience. Expositors, tape machines, CD-players and a PDA were judged. The PDA was most versatile and easy to use system~\cite{MuseumGuideSystems} (\textit{Figure 1.}). The described system had the portability of an audio guide, but due to position recognition the PDA would always present the current exhibit. The system could replace common audio guides and immobile information terminals all together. In addition, it still was able to give predefined tours depending on the user's interests.
\\
Yet another chapter was opened, when the internet and wireless communication were introduced. Museums began to also maintain websites. Burgard et al. went a step further, included robotics and build an autonomous tour-guide robot called \textit{RHINO}. It was able to navigate through the museum freely and without bumping into visitors. On demand, RHINO worked as an information terminal for present visitors and it could be used as a tour-guide as well, because it had a simple build-in web interface. Thus, the museum's contents where simultaneously used by the website and the robotic tour-guide. RHINO was deployed at the \textit{Deutsches Museum Bonn} in 1998.

In 2002, a group from the \textit{University of Limmerick} made a survey in \textit{Hunt Museum}. The museum is owned and run by the Hunt-family. Its tradition is to involve the visitors since its early days. Therefor, they had so-called \textit{cabinets of curiosity} [Cio02], special compartments within the exhibition, where additional exhibits were hidden. For example, a curious visitor had to open drawers in order to find a collection of plates. Via this exploration, the visitors became involved. Inspired by their observations, Ciolfi et al. implemented a completely new and interactive part of the exhibition in 2005 [Cio05]. Two new rooms were introduced. First, there was the \textit{study room} with three interactive devices for getting further information about certain exhibits. They were disguised as a chest, a painting and a desk. The second room, the \textit{room of opinion}, was plain white with plinths, on which visitors could record their interpretations of the intended function of certain exhibits. In  order to manage all the data, a third and hidden room was used to host all the data-servers.
\\
The \textit{medien.welten}-exhibition at \textit{Technisches Museum Wien} was not only showcasing technical devices from all eras and genres of modern media, it also invited visitors to make use of some. As Hornecker et al. described, throughout the exhibition users were given opportunities to produce their own medial content. The interfaces ranged from an abacus over a telegraph to a whole rebuild of a news-studio from an Austrian TV-channel. Visitors could not only use the devices, but store some of their produced contents in a \textit{digital backpack} [Hor06]. This way, visitors did not only have an exciting experience, but also something to remember it by later.

%\paragraph{Annotations}
%
%\begin{itemize}
	%\item Historical evolution
	%\begin{itemize}
		%\item Museums are believed to be old fashioned
		%\item Mostly willing to experiment (Examples)
		%\begin{itemize}
			%\item Dioramas
			%\item ...
			%\item Animatronics
			%\item Robotics
		%\end{itemize}
	%\end{itemize}
%\end{itemize}
%
%------------------------------------------------------------------------------------------
%
%\subsection*{Old version}
%
%Museums, much like libraries, are foremost seen as a place of knowledge and its preservation. Hence, visitors behave in a very reserved manner. Whereas this may apply for a library, museums are willing to involve people instead of merely providing information. Many Museums therfore employ guides, who give tours and tell visitors about the exhibits. In addition to their factual knowledge, they can also provide anecdotes and other information needed to bond with a certain topic. Apart of instructive and teaching staff, museums have tried many other ways to involve their visitors more. One of those is employing technology. With time technology evolved, and so did technological augmentations in museums.
%\\
%It may have started with simple mechanics, which moved some models, and later included basic electronics, which illuminated particular exhibits. Microchips and computers became more and more popular and affordable. So, the next step was immanent. There were info-terminals (...) Yet another chapter was opened, when the internet and wireless communication were introduced. Burgard et al. build an autonomous tour-guide robot called RHINO. It was able to navigate through the museum freely without bumping into visitors. RHINO could be used as a tour-guide for present visitors as well as for visitors on the internet, for it had a simple build-in and a web interface [Bur98]. RHINO was deployed at the �Deutsches Museum Bonn� in 1998.
%\\
%In 2002, a group from the University of Limmerick made a survey in the Hunt Museum. The
%museum is owned and run by the Hunt family, whose tradition it was from the beginning to involve the visitors. Therefore, they had so-called cabinets of curiosity [Cio02], special compartments within the exhibition, where additional exhibits were hidden. For example, one had to open drawers in order to find a collection of plates. Via this exploration, the visitors became involved. Inspired by their observations, Ciolfi et al. implemented a completely new and interactive part of the exhibition in 2005. Two new rooms were  introduced. First, there was a study room with three interactive devices for getting further information about certain exhibits. They were disguised as a chest, a painting and a desk. The second room, the room of opinion, was plain white with plinths, on which visitors could record their interpretations of intended function of certain exhibits. In  order to manage all the data, a third and hidden room was used to host all the data-servers [Cio05].
%\\
%Something about [Hor06].

%------------------------------------------------------------------------------------------

\section{Interfaces and Interaction}
\label{motivation_interfaces}

The only connecting link between user and system is the interface. It dictates the way of interaction and, therefore, whether the whole systems works or not. A ground breaking system is worth nothing without proper interaction between its operator and it. This presents the need for suitable kinds of input and feedback. Thus, Ben Shneiderman once introduced his \textit{eight golden rules of interface design}:
\begin{quote}
	\begin{enumerate}
		\item Strive for consistency.
	%Consistent sequences of actions should be required in similar situations; identical terminology should be used in prompts, menus, and help screens; and consistent commands should be employed throughout.
		\item Enable frequent users to use shortcuts.
	%As the frequency of use increases, so do the user's desires to reduce the number of interactions and to increase the pace of interaction. Abbreviations, function keys, hidden commands, and macro facilities are very helpful to an expert user.
		\item Offer informative feedback.
	%For every operator action, there should be some system feedback. For frequent and minor actions, the response can be modest, while for infrequent and major actions, the response should be more substantial.
		\item Design dialogs to yield closure.
	%Sequences of actions should be organized into groups with a beginning, middle, and end. The informative feedback at the completion of a group of actions gives the operators the satisfaction of accomplishment, a sense of relief, the signal to drop contingency plans and options from their minds, and an indication that the way is clear to prepare for the next group of actions.
		\item Offer error prevention and simple error handling.
	%As much as possible, design the system so the user cannot make a serious error. If an error is made, the system should be able to detect the error and offer simple, comprehensible mechanisms for handling the error.
		\item Permit easy reversal of actions.
	%This feature relieves anxiety, since the user knows that errors can be undone; it thus encourages exploration of unfamiliar options. The units of reversibility may be a single action, a data entry, or a complete group of actions.
		\item Support internal locus of control.
	%Experienced operators strongly desire the sense that they are in charge of the system and that the system responds to their actions. Design the system to make users the initiators of actions rather than the responders.
		\item Reduce short-term memory load.~\cite{DesigningTheUI}
	%The limitation of human information processing in short-term memory requires that displays be kept simple, multiple page displays be consolidated, window-motion frequency be reduced, and sufficient training time be allotted for codes, mnemonics, and sequences of actions. 
	\end{enumerate}
\end{quote}

Following this guideline should yield a well operable interface. However, there still might be particular difficulties for specialized or novel systems. Especially public user-interfaces bring new factors, which are not explicitly included in Shneiderman's rules. How should an interface behave,
\begin{itemize}
	\item in order to invite users?
	\item on a user's first encounter?
	\item if there are multiple users?
\end{itemize}

The ideal interface should be as intuitive and naturally to use as possible. This requirement was already addressed in 1980 by Richard A. Bolt. He described the \textit{Media Room} at \ac{MIT} as an office with a chair, a wall-sized screen and other analogue or electronic installations. The room's equipment allowed the user to navigate through the \textit{\ac{SDMS}} called \textit{Dataland}. The user would sit down and had several input-devices at its disposal. One of them was a small device that could measure its position and orientation in space. The device was used to calculate where on the big screen the user was pointing. In combination with simple voice-commands the user was able to create and manipulate geometrical primitives~\cite{PutThatThereUI}.
\\
Since then, this seemingly futuristic furnishing could not be established as a common way of interaction. Keyboard and Mouse are still the most widely used input-devices. Meanwhile, touchscreens and voice-recognition are closing the gap though. The principle of \ac{WYSIWYG}\footnote{\ac{WYSIWYG} first came up in the 1970s, when the first office-programs appeared. Due to different resolution-densities of displays and printers, it was necessary to show correct relations of letters and page. Later on, the term was synonymously used for \ac{GUI}-elements.} became of ever greater importance. Many interfaces are designed with Shneiderman's rules and usability in mind. Developments in \ac{HCI} seem promising. More intuitive devices and interfaces are developed and tested thoroughly. 

There are basically two types of interfaces. A \ac{SUI} is designed to be operated by only one user, whereas a \ac{MUI} can be operated by a group of users at once. However, there is no strict distinction between the two. People might look over a \ac{SUI}'s user's shoulder and give instructions, or a lone person could operate a \ac{MUI} on its own. Another factor that influences how people use an interface is the occasion. Public interfaces, such as a kiosk system at a cinema or for photo-developing provide a \ac{GUI} on a touchscreen. Those systems are intended to be used by a single user, but might also be confronted with groups. Azad et al. investigated how groups behave around such kiosks. Most groups approached the interface asynchronously. Meaning, one member is interacting with the system, while the others watch. As time passes, the rest of the group might get more active due to \textit{intra-group communication}. They further observed, that \begin{quote}there is a semantic, profound difference between pointing and touching. Users who point are communicating ideas within a social group and may not want the technology to treat it as input.~\cite{AzadMultiUI}\end{quote} Moreover, inter-group communication is also of great importance. Shyness or frustration may inhibit an individual or group from interacting with a system. Other, strange people can ease the use, when the act as an example or explain their actions to those shy or frustrated~\cite{PublicUI}.~\cite{InteractiveExhibits}

Interfaces can not only be categorized by their intended demographics. Input- and output-devices dictate the kind of interaction. As mentioned earlier, keyboard and mouse are being replaced with novel technologies. 
[...]

%\begin{enumerate}
	%\item Anzahl - Interfaces
	%\begin{enumerate}
		%\item Single-user interface
		%\item Multi-user interface
	%\end{enumerate}
	%\item Eingabemethoden - Interaction
	%\begin{enumerate}
		%\item Established Devices (Keyboard, Mouse, and Touch interaction)
		%\item Tangible interaction~\cite{TangibleUI}
		%\item VR-interaction~\cite{VRObjectSelectionCnG}
	%\end{enumerate}
%\end{enumerate}

%------------------------------------------------------------------------------------------

%\section{Public Interfaces}
%\label{motivation_single}
%
%\paragraph{Public Interfaces - Annotations}
%
%\begin{itemize}
	%\item Technologies for input / interaction
	%\item Hands-free
	%\item Gestural interaction (Kinect)
%\end{itemize}

%------------------------------------------------------------------------------------------

%\paragraph{Single User-Interfaces - Annotations}
%
%\begin{itemize}
	%\item Human behavior concerning public interfaces
	%\begin{itemize}
		%\item self-service at train-stations
		%
		%\item public interfaces, such as Tobias Fischer's \textit{SMS-Schleuder fuer Fassaden}
		%\item Intuitive usage vs. inhibition
	%\end{itemize}
%\end{itemize}

%------------------------------------------------------------------------------------------

%\section{Tangible Interfaces}
%\label{related_work_tangible}
%
%\begin{itemize}
	%\item Paper: 'Token+constraint' Ullmer/Ishii/Jacob
	%\item This way, even orientation could influence the type of information. 
%\end{itemize}

%------------------------------------------------------------------------------------------

%\section{Virtual Reality}
%\label{motivation_vr}
%
%\paragraph{Annotations}
%
%\begin{itemize}
	%\item Input
	%\begin{itemize}
		%\item Metaphors and devices
		%\begin{itemize}
			%\item Navigation and selection in 3d space
			%\item Possibilities
			%\item Difficulties
			%\item Constraints
		%\end{itemize}
	%\end{itemize}
	%\item Output
	%\begin{itemize}
		%\item Ordinary screen
		%\item Stereoscopic displays
		%\item \ac{HMD} such as Oculus Rift
	%\end{itemize}
%\end{itemize}

%------------------------------------------------------------------------------------------

\section{Goal}
\label{motivation_goal}

\paragraph{Annotations}

\begin{itemize}
	\item 'Why did I do, what I did?'
	\\
	\textit{''Unfortunately, current user interfaces often lack adequate support for 3D interactions: 2D desktop systems are limited in cases where natural interaction with 3D content is required, and 3D user interfaces consisting of stereoscopic projections and tracked input devices are rarely adopted by ordinary users. The success, both in research and commercial applications, of recent touch-based interfaces raise an interesting possibility. Can the immediacy, control, and expressiveness of recent touch-based natural interfaces be applied to 3D problems?''}~\cite{ForewordCnG}
	\item 'What did I want to do?'
	\\
	\textit{Freehand Selection Gesture Pointing}~\cite{FreehandSelectionCnG}
	\\
	\textit{SMSlingshot}~\cite{SMSlingshot}
	\\
	\textit{Shared Encounters}~\cite{UrbanHCI}
\end{itemize}