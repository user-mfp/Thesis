\chapter{Implementation -- Interactive Museum Installation}
\label{implementation}

The \ac{IMI}-system consists of two main parts. First, the hardware, which involves the physical tracking and computing of that data in the background. Second, the software, which includes the \ac{IMI}-libraries and pieces of software utilizing them.

%\paragraph{Annotations}
%
%\begin{itemize}
	%\item Explanation of functionalities
	%\item Diagrams
	%\begin{itemize}
		%\item Classes
		%\item Sequences
	%\end{itemize}
	%\item Sketches
%\end{itemize}

%-----------------------------------------------------------------------------

The hardware consists of and PC, the sensor, a screen and peripheral input devices for maintenance. The designated PC is an \textit{ASUS VIVO VC60-B013M}. It employs a \textit{Intel Core i5 3210M} with a clock speed of 2x2,5GHz and 4GB DDR3 SDRAM~\cite{VivoPCVC60}. Since the system does not have complex graphics, there was no need for a sophisticated graphics card. Hence, the PC is compact and does not need much electricity. 
\\
As a display, we chose an \textit{ASUS MB168B+} with a size of 15.6'' and es resolution of 1920x1080 pixels. The display further houses its own graphics chip, which enables it to be connected to a PC via a USB3.0-cable~\cite{MB168B}. This configuration eased the installation, because of the smaller plug the holes drilled into the back wall of the showcase could be smaller than for a usual display-cable. The display is supported by a modified bookend, which is attached to the back wall with two screws. An additional hole has been drilled for the cable. This hole is hidden by the display. Although it was planned to camouflage the display with a foil, it did not seem necessary, because lighting at the new position of the display is already dim.

The soft- and firmware used for developing and running the system is based on \textit{\ac{MS} Windows 7 Professional x86}. As mentioned earlier, the software was included into \ac{FUBI}. This framework is available as an \textit{\ac{MS} Visual Studio 2010}-project. The all software is written in \text{$C\#$} and hence makes use of the \textit{\ac{MS} .NET Framework 4.5.1}~\cite{MSNET} and further \textit{\ac{MS} XNA Framework 4.0}~\cite{MSXNA} for certain functions throughout certain regions of the software. Moreover, \ac{FUBI} utilizes the functionalities of \textit{OpenNI 2.2.0.33}~\cite{OpenNI} and \textit{NiTE 2.2.0.11}~\cite{NiTE}.

Maintenance is reduced by automatic booting and shutdown. The system's BIOS boots the PC at 8:30 in the morning. A tool shuts the system down at 4:45 in the afternoon. Hence, no member of the staff has to access the PC to start or shut the system down.

%------------------------------------------------------------------------------

\section{Libraries}
\label{implementation_libraries}

\ac{IMI} includes a collection of libraries, which can be used to implement interactive applications. At the time the functionality is restricted to free-hand pointing gestures. Other mechanisms can be easily added. An \ac{IMI}-exhibition is modularly build by those libraries. The Exhibition- and Exhibit-class have structural functionality and manage an exhibition's data. Meanwhile, the Handler-classes use this data to generate and operate the interaction between user and system.

\paragraph{Preliminaries} Terms used hereafter are depicted by Figure \ref{implementation_terms} and defined as follows:
\begin{itemize}
	\item A point or corner is a point in \ac{3D} space of type \texttt{Point3D}.
	\item A plane is defined by three points or corners. 
	\item A position is a point on a plane of type \texttt{Point3D}.
	\item A pointing position is the position a user is pointing at.
	\item A pointing location it the location a pointing user is standing at.
	\item A kernel is an exhibit's membership function. It has a size and radius.
	\item A target is an exhibit's representation on the exhibition plane. It is determined by its position and kernel.
\end{itemize}

\textbf{F I G U R E}
%\begin{figure}%
%\includegraphics[width=\columnwidth]{filename}%
%\caption{.}%
%\label{fig:implementation_terms} %
%\end{figure}

%-----------------------------------------------------------------------------

%\paragraph{Annotations}
%
%\begin{itemize}
	%\item 'What are the libraries?'
	%\begin{itemize}
		%\item Overview
		%\item Structure of Exhibition and Exhibits
	%\end{itemize}
	%\item 'What does each one do?'
	%\begin{itemize}
		%\item Modularity
		%\item Config-files (XML)
		%\item Particular methods (Lotfußpunkte, Ebenenschnittpunkt, DataLogger etc.)
	%\end{itemize}
%\end{itemize}

\paragraph{Exhibition.cs} Members of the Exhibition-class are needed to define an \ac{IMI}-exhibition. All necessary information is contained and managed in this class.

The \texttt{exhibitionPlane} is the most essential component of an \ac{IMI}-exhibition. All exhibits' positions are defined on this fundamental property. It is constructed by three corners. The structure of such a Plane is declared by the GeometryHandler-class. The Exhibition-class further holds a list of \texttt{Exhibit}-objects. In this \texttt{List} all selectable \ac{IMI}-exhibits of an \ac{IMI}-exhibition are stored. In addition, the earlier mentioned user position that defines the interaction space is also saved in this class as a point. An \ac{IMI}-exhibition's name and file path are members of the Exhibition-class, too. So are the images, which are used to camouflage the display against the background and the overview of the exhibition plane with sketches of exhibits in their respective positions. Furthermore, data crucial for interaction are also members of this class. So are the default values for the threshold needed for defining exhibits and the dwell time for target selection. Timespans used by the presentation software are \texttt{lockTime}, which determines a short period of paused interaction after selecting a target and \texttt{slideTime} which determines for how long a single image is shown during an \ac{IMI}-exhibit's presentation.

The characteristic functionality of the Exhibition-class is its get- and set-methods. They allow for reading and writing all those essential variables.

%------------------------------------------------------------------------------

\paragraph{Exhibit.cs} The virtual representation of an \ac{IMI}-exhibit is structured by the Exhibit-class. This includes data relevant for interaction with the system and an \ac{IMI}-exhibit's presentation upon selection. Like an Exhibition-object, an Exhibit-object stores its own file path as well. This is necessary for organizational reasons. 

The most important member of an \ac{IMI}-exhibit is its position. In combination with the two kernel-defining members \texttt{kernelSize} and \texttt{kernelRadius} the SessionHandler-class is able to compute which target a user is pointing at.  

Members relevant for the presentation of an \ac{IMI}-exhibit are its name, descriptive text and collection of images. The images are stored in a \texttt{Dictionary} as a pair of an image's file path and the image itself. 

The Exhibit-class, just like the Exhibition-class, is of structural nature and therefore only has the same administrative methods that get and set members.

%------------------------------------------------------------------------------

\paragraph{FileHandler.cs} The objective of the FileHandler-class is \textit{saving} and \textit{loading} data of an \ac{IMI}-exhibition, its exhibits, and all their properties. Therefore, a temporary Exhibition- and Exhibit-object are needed. Exhibitions along with all related information are saved in \textit{XML-format}. The hierarchical structure of XML-files allows adding, changing and removing elements without much effort. 

To save an \ac{IMI}-exhibition and all its properties several XML-files have to be written. The main file contains the aforementioned defining properties of the exhibition. This means that each \ac{IMI}-exhibit is stored separately. It is saved as an XML-file as well. Each \ac{IMI}-file contains all properties as attributes or \textit{forwarding file paths}. Equally, corresponding images are saved as file paths leading to the actual image. The exhibition plane is also separately saved as an XML-file. 
\\
Loading works in reverse to this procedure. As en \ac{IMI}-exhibition is being loaded, its properties are either saved into a \textit{temporary Exhibition-instance} or forwarding file paths are followed. A \textit{Temporary Exhibit-instance} is used to load all of an \ac{IMI}-exhibits.
\\
The modular composition allows the \ac{IMI}-system to be flexible. \ac{IMI}-exhibits exist as independent entities and can be easily removed from or added to an existing \ac{IMI}-exhibition. The same applies for the exhibition plane. Figure \ref{fig:filehandler_data_structure} depicts the whole data-structure behind an \ac{IMI}-exhibition.

\textbf{F I G U R E}
%\begin{figure}%
%\includegraphics[width=\columnwidth]{filename}%
%\caption{.}%
%\label{fig:filehandler_data_structure} %Entity-Relationship
%\end{figure}

An additional function of the FileHandler-class is reading TXT-files. This feature is used to load pre-written descriptions of \ac{IMI}-exhibits.

%------------------------------------------------------------------------------

\paragraph{GeometryHandler.cs} The GeometryHandler-class is solving essential computations of analytic geometry-problems. Therefore, the two structs \texttt{Vector} and \texttt{Plane} are declared, because standard types of C$\#$ do not deliver the complexity needed for those computations.
\\
A \texttt{Vector}, in contrast to a standard \texttt{Vector3D} or \texttt{Vector3}, is not only a \textit{direction vector}. It also has a \textit{starting} and a \textit{terminal point}. Hence, a \texttt{Vector} $V$ is defined by a starting point $P_{S}$ and either an endpoint $P_{E}$ or a direction vector leading towards it $\overrightarrow{d_{E}}$, where $P_{S}$ is either a user's tracked head- or elbow-joint and $P_{E}$ is its tracked hand-joint. 
\begin{align*}
	V &= P_{S} + \lambda \cdot \overrightarrow{d_{E}} \\
	P_{V} &= P_{S} + \lambda_{P} \cdot \overrightarrow{d_{E}}
\end{align*}
As the factor $\lambda$ is varied, any point $P_{V}$ along a \texttt{Vector}'s direction of propagation can be expressed. 
\\
A \texttt{Plane} is defined by one starting corner $C_{1}$ and either two additional corners $C_{2}$ and $C_{3}$ or two additional \texttt{Vector}s $\overrightarrow{d_{C_{2}}}$ and $\overrightarrow{d_{C_{3}}}$ towards those corners. In short, a \texttt{Plane} $E$ is defined by two \texttt{Vector}s $V_{C_{2}}$ and $V_{C_{3}}$ of identical origin.  
\begin{align*}
	E = V_{C_{2}} + V_{C_{3}} \quad \to \quad E &= C_{1} + \lambda \cdot \overrightarrow{d_{C_{2}}} + \mu \cdot \overrightarrow{d_{C_{3}}} \\
	P_{E} &= C_{1} + \lambda_{P} \cdot \overrightarrow{d_{C_{2}}} + \mu_{P} \cdot \overrightarrow{d_{C_{3}}}
\end{align*}
Any position $P_{E}$ on a \texttt{Plain} can be expressed by varying $\lambda_{C_{2}}$ and $\mu_{C_{3}}$.

The GeometryHandler-class has two types on members. First, \textit{input-members} are pointing- and aiming points or positions. Second, \textit{output-members} are lists of classified and weighted points or positions.

However, the main task of this class is to calculate \textit{pedal points} between \texttt{Vector}s and \textit{intersections} of a \texttt{Vector} with a \texttt{Plane}. 
\\
A pedal points is calculated to define a point. Therefore, a subject has to point at this point in \ac{3D} space from two different locations. Vectors  from both locations should meet in the point to define. Intersections in \ac{3D} space are very unlikely, though. Hence, a substitute point has to be found. The closest two skew vectors come to each other is in the pedal point. Thus, this point is computed instead of an intersection. Given two \texttt{Vector}s $V_{A}$ and $V_{B}$ with starting points $S_{A}$ and $S_{B}$ and their respective directions $\overrightarrow{d_{A}}$ and $\overrightarrow{d_{B}}$.
$$V_{A} = S_{A} + \lambda_{A} \cdot \overrightarrow{d_{A}} \quad and \quad V_{B} = S_{B} + \lambda_{B} \cdot \overrightarrow{d_{B}}$$
A normal is a vector, which is perpendicular to its original vector. To get the shortest distance between two \texttt{Vector}s, their normals $\overrightarrow{n_{A}}$ and $\overrightarrow{n_{B}}$ have to be perpendicular to both \texttt{Vector}s. 
$$\overrightarrow{n_{A}} = \overrightarrow{d_{A}} \times (\overrightarrow{d_{A}} \times \overrightarrow{d_{B}}) \quad and \quad \overrightarrow{n_{B}} = \overrightarrow{d_{B}} \times (\overrightarrow{d_{A}} \times \overrightarrow{d_{B}})$$
To compute the pedal point, and stay in the metaphor, the two foot points $F_{A}$ and $F_{B}$ are needed. These points lie on either \texttt{Vector} and in between them lies the pedal point. Hence, the factors $\lambda_{F_{A}}$ and $\lambda_{F_{B}}$ have to be altered.
\begin{align*}
	F_{A} &= S_{A} + \lambda_{F_{A}} \cdot \overrightarrow{d_{A}} \quad &and \quad F_{B} &= S_{B} + \lambda_{F_{B}} \cdot \overrightarrow{d_{B}} \\
	F_{A} &= S_{A} + \frac{(S_{B} - S_{A}) \bullet \overrightarrow{n_{B}}}{\overrightarrow{d_{A}} \bullet \overrightarrow{n_{B}}} \cdot \overrightarrow{d_{A}} \quad &and \quad F_{B} &= S_{B} + \frac{(S_{A} - S_{B}) \bullet \overrightarrow{n_{A}}}{\overrightarrow{d_{B}} \bullet \overrightarrow{n_{A}}} \cdot \overrightarrow{d_{B}} 
\end{align*}
The only step left is to calculate the pedal point $P_{AB}$ between the two foot points~\cite{PedalPoint} $F_{A}$ and $F_{B}$. Since the feet already represent the points on each \texttt{Vector} that are closest to the other \texttt{Vector} and $P_{AB}$ is in the middle, $P_{AB}$ is the average of them.
$$P_{AB} = (F_{A} \cdot F_{B}) / 2$$
An exemplary configuration of the problem is depicted by Figure \ref{fig:pedal_point}.

\textbf{F I G U R E}
%\begin{figure}%
%\includegraphics[width=\columnwidth]{filename}%
%\caption{.}%
%\label{fig:pedal_point} 
%\end{figure}

The \ac{IMI}-system's interaction is based in the principle of intersections of a \texttt{Vector} and a \texttt{Plain}. Positions of \ac{IMI}-exhibits are defined this way. Furthermore, a user's pointing and aiming position is calculated exactly the same. Thus, this computation is crucial for interaction. Given a \texttt{Vector} $V$ and a \texttt{Plane} $E$ are not parallel, they intersect in a position $P_{I}$, where $P_{I} \in V \wedge E$.   
\begin{align*}
	V = P_{S} + \lambda \cdot \overrightarrow{d} \quad &\to \quad P_{I} = P_{S} + \lambda_{I} \cdot \overrightarrow{d} \\
	E = C_{1} + \mu \cdot \overrightarrow{d_{C_{2}}} + \nu \cdot \overrightarrow{d_{C_{3}}} \quad &with \quad \overrightarrow{n_{E}} = \overrightarrow{d_{C_{2}}} \times \overrightarrow{d_{C_{3}}}
\end{align*}
A \texttt{Vector} $\overrightarrow{u}$ between $P_{I}$ and any other position on $E$ has to be perpendicular to $E$'s normal $\overrightarrow{n_{E}}$. Further, a \texttt{Vector} $\overrightarrow{w}$ from $V$'s origin to any position on $E$ is needed. It describes $V$'s origin with respect to $E$. In both cases, $C_{1}$ can be taken as a known position on the \texttt{Plane}.
$$\overrightarrow{u} = P_{I} - C_{1} \quad and \quad \overrightarrow{w} = P_{S} - C_{1}$$
As Figure \ref{fig:vector_plane_intersect} shows, going from $C_{1}$ to $P_{I}$ by $\overrightarrow{u}$ is the same as by $\overrightarrow{w}$ and then along $\overrightarrow{d}$. Since $P_{I} \in E$, $\overrightarrow{n}$ also stands perpendicular on $\overrightarrow{u}$. Hence, it is also perpendicular to its equal path from $C_{1}$ to $P_{I}$.
\begin{align*}
	\overrightarrow{u} = \overrightarrow{w} + \lambda_{I} \cdot \overrightarrow{d} \quad and \quad 0 &= \overrightarrow{n_{E}} \bullet \overrightarrow{u} \\
	\quad \Rightarrow \quad 0 &= \overrightarrow{n_{E}} \bullet (\overrightarrow{w} + \lambda_{I} \cdot \overrightarrow{d})
\end{align*}
This equation can be solved for $\lambda_{I}$, which is then put back into the pointing \texttt{Vector}'s equation.
\begin{align*}
	\lambda_{I} = \frac{\overrightarrow{n_{E}} \bullet \overrightarrow{w}}{\overrightarrow{n_{E}} \bullet \overrightarrow{d}} \quad in \quad P_{I} &= P_{S} + \lambda_{I} \cdot \overrightarrow{d} \\	
	\quad \to \quad P_{I} &= P_{S} + \frac{\overrightarrow{n_{E}} \bullet \overrightarrow{w}}{\overrightarrow{n_{E}} \bullet \overrightarrow{d}} \cdot \overrightarrow{d}
\end{align*}

\textbf{F I G U R E}
%\begin{figure}%
%\includegraphics[width=\columnwidth]{filename}%
%\caption{.}%
%\label{fig:vector_plane_intersect} 
%\end{figure}

The GeometryHandler-class utilizes the \ac{MS} XNA framework. It features specialized analysis functionality such as computation dot and cross products and the normal of a plane. These functions replaced manual calculations and increased the efficiency of the class.

The third feature of the GeometryHandler-class is classifying and weighing points and positions. As mentioned in Chapter \ref{installation_testing}, the focus of this work is not the investigation of \textit{biased free-hand pointing}, but the development of an interactive system using free-hand pointing.
\\
Pointing and aiming points and positions are classified by comparing each axis' values separately. In the current state, the biggest absolute value is considered the dominant value. The combined point or position is then computed by weighing the two inputs 70:30 in favour of the dominant value.

%------------------------------------------------------------------------------

\paragraph{CalibrationHandler.cs} Points, planes and positions of an \ac{IMI}-exhibition are defined and validated by the CalibrationHandler-class. Therefore, two members are used. The first stores the samples per position from which an administrative user is pointing. It is needed to organize the inputs of aiming and pointing \texttt{Vector}s. The other member saves the threshold for which points, planes and positions are defined and validated.

To define a \texttt{Plane}, three corners are needed. Those corners are points in \ac{3D} space. A Point is defined by pointing at it from different pointing locations. This way \texttt{Vector}s from various angles define what can be described as a \textit{cloud of intersections}. They are pedal points between every \texttt{Vector} with every other \texttt{Vector} and computed by an instance of the GeometryHandler-class . Since there are three pointing locations, there are also three \texttt{Vector}s for each corner. Hence, each corner's cloud consists of three pedal points. These pedal points are averaged to form a center point.
\\
This process is repeated to validate the once defined corners. Therefore, an administrative user defines the same corners again and the previously defined corners are compared to the newly defined validation corners by checking, whether they lie within the saved threshold. If every pair of corners validates each other, the \texttt{Plane} can be saved as the corresponding member of the \ac{IMI}-exhibition.

Defining the position of an \ac{IMI}-exhibit is similar to the definition of a corner. An admin points at the exhibit on the earlier defined exhibition plane. This is done from three different pointing locations. Only this time the GeometryHandler-instance does not calculate a pedal point, but the \texttt{Vector}s' intersections with the \texttt{Plane}. After that, a center of the three points is computed.
\\
A position on an \ac{IMI}-exhibit has to be validated as well. Hence, the defining processes is repeated and the two centers are compared according to the threshold of the respective \ac{IMI}-exhibition. If they validate each other, the position is saved for the particular \ac{IMI}-exhibit. 

Should corners or a position not successfully validate, the whole definition process has to be repeated all over again. Another validation round is not sufficient, because the first defined corners or position may be flawed.
\\
The definition and validation for both corners and positions are done for pointing and aiming. Thus, there are always two values for each corner or position. These two get classified, weighted and combined into one corner or position after validation. Hence, the reliability of a corner or position is higher, because only validation of both pointing modes yield validation of the combined corner or position value.

%------------------------------------------------------------------------------

\paragraph{SessionHandler.cs} All data necessary for interaction is computed by the SessionHandler-class. It determines a user from surrounding visitors and calculates whether a target is pointed at or not. Furthermore, this handler also calculates a user's \textit{feedback position} on the exhibition plane. Thus, a user is able to adjust his or her way of pointing.

For all these calculations the SessionHandler-class needs various members. To determine who is a user amongst visitors, the class stores the predefined user position of an \ac{IMI}-exhibition as a member upon initialization. Along this particular point it further saves the radius of the interaction space around the user position.
\\
Additional members that are necessary for interacting with the \ac{IMI}-exhibition are the exhibition plane, the screen and canvas\footnote{A canvas is an XAML-element on which the overview, exhibits' positions and later the feedback position are drawn. More about the process in Chapter \ref{implementation_presentation}.} size. The size of the feedback buffer is a constant member for smoothing a user's feedback position on the display. 

A short but important function is the determination of a user. Without the identification of a user, all further interaction is not possible. All trackable visitors recognized by the system have an \ac{ID}. A \texttt{Dictionary} with this \ac{ID} as \texttt{Key} and the respective person's hip-joint position as \texttt{Value} is given to a function. This function checks if any of the visitors' hips is within the interaction space defined by the \texttt{userPosition} and \texttt{radius}. It then returns the \ac{ID} of the user closest to the \ac{IMI}-exhibition's user position. Hence, the software knows the \ac{ID} of the user to further track and hence interact with.
\\
Intersections of \texttt{Vector} and \texttt{Plane} are calculated by the GeometryHandler, but the intersection is not enough. It has to be checked, if some target was hit or not. This task is done by the SessionHandler. It is very hard to hit just the position of an \ac{IMI}-exhibit. This is where the aforementioned kernel of an \texttt{Exhibit} comes into play. As depicted by Figure \ref{fig:exhibit_kernel}a the kernel defines an \textit{\ac{AOA}}. The closer an intersection is to the center of the \ac{AOA}, the higher the affinity for its corresponding \texttt{Exhibit}. Thus, the actual size of an \ac{IMI}-exhibit does not influence its selectability. Only its \ac{AOA}'s properties determine how well an \ac{IMI}-exhibit is selectable. Especially for densely arranged exhibits, this is a useful feature. A small but important object can be given a big kernel size and radius and hence attract more selections than a less important object in its direct proximity (see Figure \ref{fig:exhibit_kernel}b).

\textbf{F I G U R E}
%\begin{figure}%
%\includegraphics[width=\columnwidth]{filename}%
%\caption{.}%
%\label{fig:exhibit_kernel} %a) basic principle and b) comparison of big an small AOA 
%\end{figure}

To select a target, it has to be determined which \ac{AOA} has the strongest affinity for the position in question. Therefore, the affinity of each \texttt{Exhibit} at this position has to be calculated and it has to be distinguished which one's is the highest. To do that for th ewhole exhibition plane and all exhibits on it in real turned out to require a lot of computing power. Hence, the efficiency was improved by using a \textit{pre-processed lookup-table of all \ac{AOA}s}. Therefore, the \texttt{Plane} is rasterized by 1000 steps in each direction. In this case each step between the matrix dots is about 3mm. Then, the all affinities for each of the matrix dots are calculated. Subsequently, the index\footnote{All \texttt{Exhibit}s of an \texttt{Exhibition} are stored in a \texttt{List}.} of the \texttt{Exhibit} with the highest affinity is stored in a \textit{lookup table}. This lookup table is a \texttt{Dictionary} with a position as \texttt{Key} and the corresponding \texttt{Exhibit}'s index as \texttt{Value}. After all positions are processed, all positions referring to no \ac{IMI}-exhibit are removed from the lookup table. Thus, in this case, the number of \textit{pre-determined affinities} is reduced from 1.000.000 to about 50.000. Thereafter, affinities do not have to be computed each frame, but positions are compared\footnote{Pointing positions might not appear in the lookup table. Therefore, the closest position is interpolated.} with those in the lookup table. If there is a match, the corresponding index is returned.
\\
Every position a user is pointing at is buffered and a smoothened \textit{feedback position} is returned. This feedback position is already converted in \texttt{Canvas}-coordinates to match the dimensions of the overview. As Figure \ref{fig:plane_screen_canvas_coords} shows, \texttt{Plane}-coordinates have to be converted into \texttt{Canvas}-coordinates with the ratios of \texttt{Plane} to \texttt{Screen} to \texttt{Canvas} in mind. Depending on the ratio of the \texttt{Plane}, the ratio between \texttt{Screen} and \texttt{Canvas} changes.

\textbf{F I G U R E}
%\begin{figure}%
%\includegraphics[width=\columnwidth]{filename}%
%\caption{.}%
%\label{fig:plane_screen_canvas_coords} %skizze zur umrechnung 
%\end{figure}

The \texttt{planeCanvasRatio} is used to calculate the correct position for the representation of the feedback position in an \ac{IMI}-exhibition's overview. Moreover, it is also used to correctly scale and display the \ac{AOA} of each \texttt{Exhibit} it in the defined position on the overview.

%------------------------------------------------------------------------------

\paragraph{StatisticsHandler.cs} Statistical calculations are done by the StatisticsHandler-class. It computes the \textit{mean}, \textit{variance}, \textit{(empiric) standard deviation} and standard error for a sample of \texttt{Point3D} or \texttt{double}-values. This feature was used for informal analysis of raw data during the iterative development-process. Nevertheless, it might be used for more representative long-term evaluation. 

%------------------------------------------------------------------------------

\paragraph{DataLogger.cs} The DataLogger-class can be used for saving all kinds of data of the \ac{IMI}-system. It was developed to write log-files in TXT-format. Therefore, it is possible to initiate an instance with only a file path. After that, new paragraphs can be set up with either a headline or no headline. These paragraphs can then be addressed with an index. Thus, new lines are addable at any point during the runtime of an \ac{IMI}-application. A logs can be saved as TXT-files anytime, too.
\\
The DataLogger-class was used during the iterative development-process to log pointing and aiming \texttt{Vector}s as well as defined points and positions in one file for each subject along with related statistics of these values.
\\
The presentation-software, uses the DataLogger-class to save all events during a session. Thus, the museum is able to keep track of its visitors' behavior and interests. These session-logs can be further analyzed by the \ac{IMI}-statistics tool described in Chapter \ref{implementation_tool}.

%------------------------------------------------------------------------------

\section{Administration-software}
\label{implementation_administration}

\paragraph{Annotations}

\begin{itemize}
	\item 'What is the administration-software?'
	\begin{itemize}
		\item Define and edit exhibitions
		\begin{itemize}
			\item ExhibitionPlane
			\item Define, load and remove Exhibits
			\item Define and change UserPosition
			\item Edit dwelltimes
			\item Load Background(s)
		\end{itemize}
		\item Define and edit exhibits
		\begin{itemize}
			\item Define and change Position
			\item Load and remove Images
			\item Write and load Description (up to 310 charcters)
		\end{itemize}
	\end{itemize}
	\item 'What does it do?'
	\begin{itemize}
		\item \textit{Sequences}
		\item Paper-mockup
		\item Create (re-)loadable Config-files
	\end{itemize}
\end{itemize}

%------------------------------------------------------------------------------

\section{Presentation-software}
\label{implementation_presentation}

\paragraph{Annotations}

\begin{itemize}
	\item 'What is the presentation-software?'
	\begin{itemize}
		\item Display information of previously defined interactive exhibits
		\item Overview-map of ExhibitionPlane
		\item Feedback of exhibits' positions and pointing position
		\item Description (Readability, Sehwinkel) and Images as slide show
	\end{itemize}
	\item 'What does it do?'
	\begin{itemize}
		\item Check for Exhibition
		\item Pre-calculate Lookup for exhibit-selection (saves processing power)
		\item Recognize visitors
		\item Identify user by predefined UserPosition 
	\end{itemize}
\end{itemize}

%------------------------------------------------------------------------------

\section{Presentation-remote}
\label{implementation_remote}

\paragraph{Annotations}

\begin{itemize}
	\item 'What is the presentation-remote?'
	\begin{itemize}
		\item Microsoft Gadgeteer-Device
		\item Bluetooth / WiFi-connection to PC
		\item For lecturers in order to explain exhibits themselves
	\end{itemize}
	\item 'What does it do?'
	\begin{itemize}
		\item Automatically connect to Presentation-software
		\item Toggle Presentation-software's blindness
	\end{itemize}
\end{itemize}

%------------------------------------------------------------------------------

\section{Statistics-tool}
\label{implementation_tool}

\paragraph{Annotations}

\begin{itemize}
	\item 'What is the statistics-tool and what does it do?'
	\begin{itemize}
		\item Small tool to evaluate logged user-data
		\item Statistics, such as average length of stay/session, exhibits chosen and how many transitions 
	\end{itemize}
\end{itemize}
