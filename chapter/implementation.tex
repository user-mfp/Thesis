\chapter{Implementation}
\label{implementation}

The \ac{IMI}-system consists of two main parts. First, the hardware part involves the physical tracking and computing of its data in the background. Second, the software part, which includes the \ac{IMI}-libraries and -softwares utilizing them.

%\paragraph{Annotations}
%
%\begin{itemize}
	%\item Explanation of functionalities
	%\item Diagrams
	%\begin{itemize}
		%\item Classes
		%\item Sequences
	%\end{itemize}
	%\item Sketches
%\end{itemize}

%-----------------------------------------------------------------------------

\section{Testing}
\label{conception_testing}

Before anything could be installed or evaluated, a reliable system had to be developed. Therefore, I researched suitable environments for an extensible system. Because most \ac{SDK}s for PrimeSense's hardware are implemented in C++ or C$\#$ and Gadgeteer uses Microsoft's .NET framework and C$\#$, the final system should be implemented in C$\#$. Thus, an \ac{SDK} written in C$\#$ was to be found. After having tried several open source frameworks, the \ac{FUBI} developed at Universität Augsburg proved to fit our needs best. \ac{FUBI} came with a C$\#$-wrapper, which incorporated all functionality of OpenNI and NiTE that was necessary to achieve our goals. Moreover, its leading developer, \textit{Dipl.-Inf. Felix Kistler}, kindly explained how to incorporate the new approach to \ac{FUBI}.  

\paragraph{Annotations}

\begin{itemize}
	\item Test of pointing accuracy
	\begin{enumerate}
		\item One centered Point I
		\begin{itemize}
			\item Only Pointing
			\item \textit{Images and sketches}
			\item \textit{Data and Statistics}
			\item results and conclusion
			\item See appendix
		\end{itemize}
		\item One centered Point II
		\begin{itemize}
			\item Pointing, Aiming and Combined
			\item \textit{Images and sketches}
			\item \textit{Data and Statistics}
			\item results and conlusion
			\item See appendix
		\end{itemize}
		\item Four Points on each corner of the plane
		\begin{itemize}
			\item Classification of combined values
			\item \textit{Images and sketches}
			\item \textit{Data and Statistics}
			\item results and conlusion
			\item See appendix
		\end{itemize}
	\end{enumerate}
	\item Development of algorithms for eye-hand mismatch (elbow/hand + head/hand)
	\begin{itemize}
		\item Description of Eye-Hand Mismatch [ref]
		\item \textit{Sketches of classification}
	\end{itemize}
	\item Test of algorithm's accuracy
	\begin{itemize}
		\item Target = '90 percent of all values within a 10cm radius of mean value'
		\item Differentiation between real and virtual point
		\item Necessity of 1:1-mapping of real and virtual point
	\end{itemize}
\end{itemize}

%------------------------------------------------------------------------------

\section{Interactive Museum Installation - Libraries}
\label{implementation_libraries}

\paragraph{Annotations}

\begin{itemize}
	\item 'What are the libraries?'
	\begin{itemize}
		\item Overview
		\item Structure of Exhibition and Exhibits
	\end{itemize}
	\item 'What does each one do?'
	\begin{itemize}
		\item Modularity
		\item Config-files (XML)
		\item Particular methods (Lotfußpunkte, Ebenenschnittpunkt, DataLogger etc.)
	\end{itemize}
\end{itemize}



\section{Interactive Museum Installation - Administration-software}
\label{implementation_administration}

\paragraph{Annotations}

\begin{itemize}
	\item 'What is the administration-software?'
	\begin{itemize}
		\item Define and edit exhibitions
		\begin{itemize}
			\item ExhibitionPlane
			\item Define, load and remove Exhibits
			\item Define and change UserPosition
			\item Edit dwelltimes
			\item Load Background(s)
		\end{itemize}
		\item Define and edit exhibits
		\begin{itemize}
			\item Define and change Position
			\item Load and remove Images
			\item Write and load Description (up to 310 charcters)
		\end{itemize}
	\end{itemize}
	\item 'What does it do?'
	\begin{itemize}
		\item \textit{Sequences}
		\item Paper-mockup
		\item Create (re-)loadable Config-files
	\end{itemize}
\end{itemize}



\section{Interactive Museum Installation - Presentation-software}
\label{implementation_presentation}

\paragraph{Annotations}

\begin{itemize}
	\item 'What is the presentation-software?'
	\begin{itemize}
		\item Display information of previously defined interactive exhibits
		\item Overview-map of ExhibitionPlane
		\item Feedback of exhibits' positions and pointing position
		\item Description (Readability, Sehwinkel) and Images as slide show
	\end{itemize}
	\item 'What does it do?'
	\begin{itemize}
		\item Check for Exhibition
		\item Pre-calculate Lookup for exhibit-selection (saves processing power)
		\item Recognize visitors
		\item Identify user by predefined UserPosition 
	\end{itemize}
\end{itemize}



\section{Interactive Museum Installation - Presentation-remote}
\label{implementation_remote}

\paragraph{Annotations}

\begin{itemize}
	\item 'What is the presentation-remote?'
	\begin{itemize}
		\item Microsoft Gadgeteer-Device
		\item Bluetooth / WiFi-connection to PC
		\item For lecturers in order to explain exhibits themselves
	\end{itemize}
	\item 'What does it do?'
	\begin{itemize}
		\item Automatically connect to Presentation-software
		\item Toggle Presentation-software's blindness
	\end{itemize}
\end{itemize}



\section{Interactive Museum Installation - Statistics-tool}
\label{implementation_tool}

\paragraph{Annotations}

\begin{itemize}
	\item 'What is the statistics-tool and what does it do?'
	\begin{itemize}
		\item Small tool to evaluate logged user-data
		\item Statistics, such as average length of stay/session, exhibits chosen and how many transitions 
	\end{itemize}
\end{itemize}
