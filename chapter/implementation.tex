\chapter{Implementation -- Interactive Museum Installation}
\label{implementation}

The \ac{IMI}-system consists of two main parts. First, the hardware, which involves the physical tracking and computing of that data in the background. Second, the software, which includes the \ac{IMI}-libraries and pieces of software utilizing them.

%\paragraph{Annotations}
%
%\begin{itemize}
	%\item Explanation of functionalities
	%\item Diagrams
	%\begin{itemize}
		%\item Classes
		%\item Sequences
	%\end{itemize}
	%\item Sketches
%\end{itemize}

%-----------------------------------------------------------------------------

The hardware consists of and PC, the sensor, a screen and peripheral input devices for maintenance. The designated PC is an \textit{ASUS VIVO VC60-B013M}. It employs a \textit{Intel Core i5 3210M} with a clock speed of 2x2,5GHz and 4GB DDR3 SDRAM. Since the system does not have complex graphics, there was no need for a sophisticated graphics card. Hence, the PC is compact and does not need much electricity. 
\\
The Display is connected to the PC via USB3.0 and houses its own graphics chip. This configuration eased the installation, because...

%------------------------------------------------------------------------------

\section{Libraries}
\label{implementation_libraries}

\paragraph{Annotations}

\begin{itemize}
	\item 'What are the libraries?'
	\begin{itemize}
		\item Overview
		\item Structure of Exhibition and Exhibits
	\end{itemize}
	\item 'What does each one do?'
	\begin{itemize}
		\item Modularity
		\item Config-files (XML)
		\item Particular methods (Lotfußpunkte, Ebenenschnittpunkt, DataLogger etc.)
	\end{itemize}
\end{itemize}



\section{Administration-software}
\label{implementation_administration}

\paragraph{Annotations}

\begin{itemize}
	\item 'What is the administration-software?'
	\begin{itemize}
		\item Define and edit exhibitions
		\begin{itemize}
			\item ExhibitionPlane
			\item Define, load and remove Exhibits
			\item Define and change UserPosition
			\item Edit dwelltimes
			\item Load Background(s)
		\end{itemize}
		\item Define and edit exhibits
		\begin{itemize}
			\item Define and change Position
			\item Load and remove Images
			\item Write and load Description (up to 310 charcters)
		\end{itemize}
	\end{itemize}
	\item 'What does it do?'
	\begin{itemize}
		\item \textit{Sequences}
		\item Paper-mockup
		\item Create (re-)loadable Config-files
	\end{itemize}
\end{itemize}



\section{Presentation-software}
\label{implementation_presentation}

\paragraph{Annotations}

\begin{itemize}
	\item 'What is the presentation-software?'
	\begin{itemize}
		\item Display information of previously defined interactive exhibits
		\item Overview-map of ExhibitionPlane
		\item Feedback of exhibits' positions and pointing position
		\item Description (Readability, Sehwinkel) and Images as slide show
	\end{itemize}
	\item 'What does it do?'
	\begin{itemize}
		\item Check for Exhibition
		\item Pre-calculate Lookup for exhibit-selection (saves processing power)
		\item Recognize visitors
		\item Identify user by predefined UserPosition 
	\end{itemize}
\end{itemize}



\section{Presentation-remote}
\label{implementation_remote}

\paragraph{Annotations}

\begin{itemize}
	\item 'What is the presentation-remote?'
	\begin{itemize}
		\item Microsoft Gadgeteer-Device
		\item Bluetooth / WiFi-connection to PC
		\item For lecturers in order to explain exhibits themselves
	\end{itemize}
	\item 'What does it do?'
	\begin{itemize}
		\item Automatically connect to Presentation-software
		\item Toggle Presentation-software's blindness
	\end{itemize}
\end{itemize}



\section{Statistics-tool}
\label{implementation_tool}

\paragraph{Annotations}

\begin{itemize}
	\item 'What is the statistics-tool and what does it do?'
	\begin{itemize}
		\item Small tool to evaluate logged user-data
		\item Statistics, such as average length of stay/session, exhibits chosen and how many transitions 
	\end{itemize}
\end{itemize}
