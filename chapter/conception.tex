\chapter{Conception}
\label{conception}

After the \textit{Museum für Ur- und Frühgeschichte Thürigens} was chosen as a partner, all previous ideas had to be analyzed more thoroughly with feasibility in mind. Thus, impractical and too complex or too simple ideas were eliminated in two rounds of review. At first, vague ideas were either improved or discarded. Hence, a screen displaying only information about a fossilized fireplace was eliminated. The idea of a system for digitizing stone carvings was considered too complex to realize and therefor discarded as well. Afterward, some of the museum's staff and I looked at the contents, that could be provided for the remaining candidates. This left us with two remaining possibilities, that were promising enough from an educational and a technical standpoint. The first one was the reproduction of the \textit{Fürstengrab von Haßleben}, which contains replicas and original findings from a 1700 year old grave of a teutonic princess. A close second was a workshop, which should have shown how archeologists and preparateurs work behind the scenes of a museum. Here, the latter consisted of too many single parts and a lot of questions remained unanswered.
\\
According to the aforementioned review, the Fürstengrab von Haßleben was most promising and therefore chosen in the end. It contains many special relics from ordinary, teutonic pottery to rare, roman coins and jewelry. This apparent eclecticism is, what makes the grave so special though. It is a sublime showcase for thriving trade and cultural exchange between Teutons and Romans as far east as Thuringia. Further, it proves how Teutons began adapting roman traditions, such as burials. In order to emphasize this insight, an interactive system was to be developed.

%---------------------------------------------------------------------------- 

\section{System design}
\label{conception_system}

The final system was developed and tested by me, while the museum-staff will be responsible for future maintenance. Some visitors might not have proper technical experiences to operate a contemporary interface. Consequently, it was crucial to design the system with that in mind. It had to be operable by absolute lay persons, who have no prior experience concerning information technologies. Hence, the interface had to be intuitive. Three major points had to be considered.
\\
First, established and common input devices, such as keyboard and mouse, had to be replaced by something different. In order to be intuitive, the interaction was designed to capture and use the natural behavior of visitors. Outputs, on the other hand, had to be as discreet and as conservative as possible to not disturb or interfere with the exhibition. Thus, invasive technologies such as speakers and animatronics were excluded from the beginning. This consideration only left visual and haptic channels for output. The third point was, that daily operations at the museum were not to be compromised. So, it was not possible to develop the prototype inside the Haßleben-showcase itself and a full-size mockup had to be build somewhere else. Therefore, I measured the showcase and acquired a room in which a mockup could be placed for the prototype's implementation and testing\footnote{For a further description of the lab-setup see chapter \ref{installation_lab}}.
\\


%\paragraph{Annotations}
%
%\begin{itemize}
	%\item User perspective
	%\begin{itemize}
		%\item Visitor
		%\item Curator / staff
	%\end{itemize}
	%\item System view
	%\\
	%\item Development of ideas according to the plan
	%\begin{itemize}
		%\item Method of elimination
		%\item Feasibility
		%\begin{itemize}
			%\item Effort
			%\item Cost
		%\end{itemize}
	%\end{itemize}
%\end{itemize}

%-----------------------------------------------------------------------------

%\paragraph{Annotations}
%
%\begin{itemize}
	%\item Possibilities of hard- and software
	%\item Capabilities of a single programmer (me)
%\end{itemize}

%-----------------------------------------------------------------------------

\section{Constraints}
\label{conception_constraints}

\paragraph{Annotations}

\begin{itemize}
	\item Technical
	\item From the museums perspective
	\begin{itemize}
		\item Size
		\item Cost
		\item Inclusion
	\end{itemize}
	\item Limitations of hard- and software
	\item Capabilities of a single programmer (me)
\end{itemize}

%-----------------------------------------------------------------------------

\section{Final concept}
\label{conception_final}

\paragraph{Annotations}

\begin{itemize}
	\item 'Pflichtenheft'-criteria
	\begin{itemize}
		\item Must
		\begin{itemize}
			\item 
		\end{itemize}
		\item Should
		\begin{itemize}
			\item 
		\end{itemize}
		\item Could
		\begin{itemize}
			\item 
		\end{itemize}
		\item See appendix
	\end{itemize}
	\item Contract 
		\begin{itemize}
			\item MUFT, BUW and me
			\item Avoid misconceptions
			\item Commitments / Obligations
			\item Responsibilities
			\item Boundaries
			\item Legal stuff
			\item See appendix
		\end{itemize}
\end{itemize}

%-----------------------------------------------------------------------------

\section{Testing}
\label{conception_testing}

\paragraph{Annotations}

\begin{itemize}
	\item Test of pointing accuracy
	\begin{enumerate}
		\item One centered Point I
		\begin{itemize}
			\item Only Pointing
			\item \textit{Images and sketches}
			\item \textit{Data and Statistics}
			\item results and conlusion
			\item See appendix
		\end{itemize}
		\item One centered Point II
		\begin{itemize}
			\item Pointing, Aiming and Combined
			\item \textit{Images and sketches}
			\item \textit{Data and Statistics}
			\item results and conlusion
			\item See appendix
		\end{itemize}
		\item Four Points on each corner of the plane
		\begin{itemize}
			\item Classification of combined values
			\item \textit{Images and sketches}
			\item \textit{Data and Statistics}
			\item results and conlusion
			\item See appendix
		\end{itemize}
	\end{enumerate}
	\item Development of algorithms for eye-hand mismatch (elbow/hand + head/hand)
	\begin{itemize}
		\item Description of Eye-Hand Mismatch [ref]
		\item \textit{Sketches of classification}
	\end{itemize}
	\item Test of algorithm's accuracy
	\begin{itemize}
		\item Target = '90 percent of all values within a 10cm radius of mean value'
		\item Differentiation between real and virtual point
		\item Necessity of 1:1-mapping of real and virtual point
	\end{itemize}
\end{itemize}
