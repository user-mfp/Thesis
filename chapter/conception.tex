\chapter{Conception}
\label{conception}

After the \textit{Museum für Ur- und Frühgeschichte Thürigens} was chosen as a partner, all previous ideas had to be analyzed more thoroughly with feasibility in mind. Thus, impractical and too complex or simple ideas were eliminated in two rounds of review. At first, merely vague ideas were either improved or discarded. Hence, a screen displaying blunt information about a fossilized fireplace was eliminated. A system for digitizing stone carvings was considered too complicated to realize and therefor discarded as well. Afterward, some of the museum's staff and I looked at the contents, which could be provided by the remaining candidates. This left us with only two remaining possibilities, that were promising enough from an educational and a technical standpoint. The first one was the reproduction of the \textit{Fürstengrab von Haßleben}, which contained replicas and original findings from a 1700 year old grave of a germanian princess. A close second was a workshop, which should have show how archeologists and preparateurs work behind the scenes of a museum. Here, the latter consisted of too many single parts and a lot of questions remained unanswered. 

%------------------------------------------------------------------------------------------

\section{System design}
\label{conception_system}

\paragraph{Annotations}

\begin{itemize}
	\item User perspective
	\begin{itemize}
		\item Visitor
		\item Curator / staff
	\end{itemize}
	\item System view
	\\
	\item Development of ideas according to the plan
	\begin{itemize}
		\item Method of elimination
		\item Feasibility
		\begin{itemize}
			\item Effort
			\item Cost
		\end{itemize}
	\end{itemize}
\end{itemize}

%------------------------------------------------------------------------------------------

\section{Design options}
\label{conception_design}

\paragraph{Annotations}

\begin{itemize}
	\item Possibilities of hard- and software
	\item Capabilities of a single programmer (me)
\end{itemize}

%------------------------------------------------------------------------------------------

\section{Constraints}
\label{conception_constraints}

\paragraph{Annotations}

\begin{itemize}
	\item Technical
	\item From the museums perspective
	\begin{itemize}
		\item Size
		\item Cost
		\item Inclusion
	\end{itemize}
	\item Limitations of hard- and software
	\item Capabilities of a single programmer (me)
\end{itemize}

%------------------------------------------------------------------------------------------

\section{Final concept}
\label{conception_final}

\paragraph{Annotations}

\begin{itemize}
	\item 'Pflichtenheft'-criteria
	\begin{itemize}
		\item Must
		\begin{itemize}
			\item 
		\end{itemize}
		\item Should
		\begin{itemize}
			\item 
		\end{itemize}
		\item Could
		\begin{itemize}
			\item 
		\end{itemize}
		\item See appendix
	\end{itemize}
	\item Contract 
		\begin{itemize}
			\item MUFT, BUW and me
			\item Avoid misconceptions
			\item Commitments / Obligations
			\item Responsibilities
			\item Boundaries
			\item Legal stuff
			\item See appendix
		\end{itemize}
\end{itemize}

%------------------------------------------------------------------------------------------

\section{Testing}
\label{conception_testing}

\paragraph{Annotations}

\begin{itemize}
	\item Test of pointing accuracy
	\begin{enumerate}
		\item One centered Point I
		\begin{itemize}
			\item Only Pointing
			\item \textit{Images and sketches}
			\item \textit{Data and Statistics}
			\item results and conlusion
			\item See appendix
		\end{itemize}
		\item One centered Point II
		\begin{itemize}
			\item Pointing, Aiming and Combined
			\item \textit{Images and sketches}
			\item \textit{Data and Statistics}
			\item results and conlusion
			\item See appendix
		\end{itemize}
		\item Four Points on each corner of the plane
		\begin{itemize}
			\item Classification of combined values
			\item \textit{Images and sketches}
			\item \textit{Data and Statistics}
			\item results and conlusion
			\item See appendix
		\end{itemize}
	\end{enumerate}
	\item Development of algorithms for eye-hand mismatch (elbow/hand + head/hand)
	\begin{itemize}
		\item Description of Eye-Hand Mismatch [ref]
		\item \textit{Sketches of classification}
	\end{itemize}
	\item Test of algorithm's accuracy
	\begin{itemize}
		\item Target = '90 percent of all values within a 10cm radius of mean value'
		\item Differentiation between real and virtual point
		\item Necessity of 1:1-mapping of real and virtual point
	\end{itemize}
\end{itemize}
