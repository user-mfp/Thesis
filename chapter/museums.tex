\chapter{Museums}
\label{museums}

\paragraph{Annotations}

\begin{itemize}
	\item Project process: Partnering
	\\
	\item Preselection of possible partners
	\item Criteria
	\begin{itemize}
		\item Proximity
		\item Flexibility
		\item Open-mindedness
		\item Attractiveness of theme
	\end{itemize}
	\item 'Supply and demand'
\end{itemize}



\section{Requirement analysis}
\label{museums_requirement}

\paragraph{Annotations}

\begin{itemize}
	\item 'What do we have to offer?'
	\item 'What do we need?'
	\item 'What should the museum be offering?'
	\item 'What does the museum want?' \textit{better: need}
\end{itemize}



\section{Further investigation}
\label{museums_investigation}

\paragraph{Annotations}

\begin{itemize}
	\item Visit preselected museums
	\item Getting an Overview $\to$ (Im)Possibilities
	\item Establish a first contact
\end{itemize}



\section{Determination}
\label{museums_determination}

\paragraph{Annotations}

\begin{itemize}
	\item Offical introduction at the museum
	\begin{itemize}
		\item Personal
		\item Present requirements see \ref{museums_requirement}
	\end{itemize}
	\item Brainstorming
	\begin{itemize}
		\item Museum-staff: 'Emphases'
		\item Me: 'Possible solutions'
	\end{itemize}
\end{itemize}



\subsection*{Old version}

In order to finding a museum to cooperate with several steps had to be made. They included getting an overview of all museums in Weimar, finding several candidates for that cooperation, scouting those candidates and getting in contact with the most promising of them, and, finally, discussing possible concepts within their exhibitions.
\\
The first step was to find out about all the museums in Weimar and close by. So I looked them up on the website of Museumsverband Thüringen [...], where there is a list of all members with links to further information. This list included museums in Weimar, Jena and Erfurt. Some were run privately, others by a foundation or a club, and a few by a public owner. Since there was a total of fifty museums and half of them in Weimar alone, there had to be a preselection.
\\
Hence, as the following step, only museums in Weimar were chosen. In addition, the museums run by Klassikstiftung were taken out of consideration, for the foundation seemingly being too big and too inflexible concerning innovation in their historic premises. Some very small museums were struck off the list as well. This left four candidates remaining. They were Pavillon Presse, Pallais Schardt, Bienenmuseum and Museum für Ur- und Frühgeschichte Thüringens.
\\
The next step was to get some first hand experience of each of the aforementioned museums. So, I went to visit all of them. During the visit I took notes and pictures of the exhibitions. Afterward, I talked to some staff members, explained what I was about to do, and arranged an appointment for an official introduction later on.
\\
The first visit was to the Bienenmuseum. It is run by a club of beekeepers and displays exhibits of beekeeping throughout the ages and several cultures. The exhibition is mainly conventional with vitrines and open exhibits. Moreover, they offer workshops, in which attendees learn more about bees in general, 'making' honey and even dipping our pouring candles.
\\
Pallais Schardt was the second visit. It is the historical home of the Schardt family, a very influential family at the court of Sachsen-Weimar. This place is owned by the Brinkmann family and run aside a cafe with traditional pastries from that particular era. Mr. Brinkmann is giving tours around the premisses and explains the building's significance in close contact to historical events. In addition, the saloon and other rooms can be rented for festivities.
\\
Right next to Pallais Schardt is Pavillon Presse. It used to be a printery and now accomodates printing presses and equipment from all ages. The museum is privately run by a foundation and volunteers. This museum was struck of the list immediately after the visit, for being to capricious to work with.
\\
The final visit was to the Museum für Ur- und Frühgeschichte Thüringens. There, artifacts from fossils, which a millions of years old, to medieval times are exihibited. The museum was overhauled in 1999 and thus, has a modern touch already. It is owned and run by the Thüringisches Landesamt für Denkmalpflege und Archäologie.
\\
After those field trips, I fashioned a presentation, in which I would introduce myself and previous projects I participated in. Later on in the meeting, I would show pictures of the museums in Limmerick and Vienna and explained the work, which had been done there. Finally, I prepared a short presentation of the Microsoft Gadgeteer-system and some of its capabilities. Following my presentation, the attending museum-staff, my professor end I discussed possible deployment scenarios. During the brainstorming the museum-officials named exhibits, which could or rather should receive more attention, whilst me and my professor suggested fitting solutions or explained further technological possibilities.
\\
At Pallais Schardt, the owners were very interested in technology, but they could not imagine how and where to make use of it. The best thing we could come up with was a guided tour. Thus, I was invited to one of their soirees with classical music and a tour of the house, in order to making up my own mind. Although it was very interesting, nothing ground-breaking arose.
\\
At Museum für Ur- und Frühgeschichte, the director was very fascinated by the demo and immediately came up with several exhibits, which seemed fitting to him. Yet, his  ptimism had to be reined a little. Some of the tasks he had in mind were unfortunately not realizable with the tool I have in hand.
\\
At Bienenmuseum, there were two main topics. First, social interaction of bees. For instance, bees dance to communicate the direction of plenty resources. Second, bees' perception of their environment. Bees see in another spectrum than we do and they can smell a lot better than us. In the end of our meeting, we were discussing about a virtual bee hive. This installation would be able
to simulate the behavior of a bee colony according to some certain inputs, which could be made by visitors.
\\
The final decisions were made after working out several key criteria for the best possible cooperation. Those were common criteria every museum could or could not meet and special criteria, which could also tip the scales. Three common criteria were identified. First of all, the amount and age of visitors was very important. Since the prototype had to be evaluated, a sufficient number of potential test subjects with a certain grade of affinity for technology would be needed. Second and not much less important, was the size and quality of the staff. If there was no expert of the museum's subject, who was able to work together with me, the project would be a fail. The third criteria was plainly budget. At some point, additional electronics and/or other equipment would be necessary. The special criteria more or less had an influence on the aforementioned main criteria. For example, monument protection, seasons, and motivation were some of them. First, Bienenmuseum had to go, because the club's chairman was not very fond of our discussions. Furthermore, the staff was not particularly professional and seemed to run the museum more as a hobby. The fact, that the museum has a large variety of visitors was a big plus, which was neutralized with the other fact, that bees are seasonal, and so are the according numbers of visitors. This makes an evaluation rather difficult, for not providing a constant number of test subjects.
\\
Finally, Pallais Schardt was struck from the list. Although, its owner was a restorer by trade, very approachable, and there were lots of events at the cafe, it had some corresponding cons as well. The building ant its historic role was very interesting, yet is a landmark. Thus, it must not be altered in any form, which might prove hard later on. The many people visiting the cafe are mostly 50 years an older. Hence, their abilities to understand and use technology as intended could be too much a risk during evaluation. Sadly, it is just a cafe and not a museum.
\\
The last item on the list is the Museum für Ur- und Frühgeschichte Thüringens. The major con was the planned exhibition, which leaves not much space for alterations. But, it is controlled by regional authorities. Hence, there is a budget for innovation projects. Moreover, the staff at the museum is interested in innovation and highly qualified in their field of expertise.
