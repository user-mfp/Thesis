\chapter{Future Work}
\label{future_work}

During the development and implementation of free-hand pointing gestures as input for a public interface, a lot of new and interesting perspectives on this style of interaction appeared. The \ac{IMI}-system as such reliably works and has become an established part of the Museum für Ur- und Frühgeschichte Thüringens. My initial theory was proved to some extend. The presence of an interactive installation, however, increases the engagement of visitors with the Haßleben-showcase and thus their awareness of the topic. Whether the interaction is successful or not did not seem to matter that much. But the attention it creates does, because visitors look at the exhibits more carefully when they point at them. 
\\
Nevertheless, there is still room for improvements and further research as talking to staff of the museum and the university, fellow students and participants of the studies revealed. 

\textbf{Lessons from Development and Implementation}
\\- research of angular error -> more efficient combined positions through improvement of eye hand-mismatch
\\- research of kernel functions -> improvement of kernels (dominant and submissive kernel-functions)  
\\- 3D selectability by using the XNA framework (bounding sphere)
\\- gereal improvement of the implementation and prettier \ac{GUI}

\textbf{Lessons from Studies}
\\- multiple users (leads to mobile devices)
\\- allow mobility of a user
\\- recognize small children
\\- improve visual feedback 
\\---- bigger display in viewing direction
\\---- feedback directly on exhibition plane (spot requested) 
\\---- (as Conception stated earlier)
\\- improve readability
\\---- more time to read
\\---- bigger letters
\\---- re-think parallelism of text and images
\\- incorporate other gestures

\textbf{Lessons from other Discussions}
\\- mobile devices -> App für iPod Touch (Audioguide)
\\- tangibles with Gadgeteer

%\paragraph*{Annotations}
%
%\begin{itemize}
	%\item My work in relation to situation described in chapters \ref{introduction} and \ref{related_work}
	%\item Outlook of possible further developments or optimizations of the system
	%\begin{itemize}
		%\item Multi-user
		%\item Mobile devices
		%\item Audio
		%\item 3-dimensional positioning of objects and users
		%\item different possibilities of feedback 
	%\end{itemize}
%\end{itemize}
