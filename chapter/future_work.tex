\chapter{Future Work}
\label{future_work}

During the development and implementation of free-hand pointing gestures as input for a public interface, a lot of new and interesting perspectives on this style of interaction appeared. The \ac{IMI}-system as such reliably works and has become an established part of the Museum für Ur- und Frühgeschichte Thüringens. My initial theory was proved to some extend. The presence of an interactive installation, however, increases the engagement of visitors with the Haßleben-showcase and thus their awareness of the topic. Whether the interaction is successful or not did not seem to matter that much. But the attention it creates does, because visitors look at the exhibits more carefully when they point at them. 

Nevertheless, there is still room for improvements and further research. As talking to staff of the museum and university, fellow students and participants of the studies revealed.
\\
\textbf{Lessons from Development and Implementation}
\\
\textbf{Lessons from Studies}
\\
\textbf{Lessons from Discussions}

%\paragraph*{Annotations}
%
%\begin{itemize}
	%\item My work in relation to situation described in chapters \ref{introduction} and \ref{related_work}
	%\item Outlook of possible further developments or optimizations of the system
	%\begin{itemize}
		%\item Multi-user
		%\item Mobile devices
		%\item Audio
		%\item 3-dimensional positioning of objects and users
		%\item different possibilities of feedback 
	%\end{itemize}
%\end{itemize}
