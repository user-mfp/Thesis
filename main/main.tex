% vim: foldmethod=marker

%% Dokumentenklasse (Koma Script) ---------------------------------------
\documentclass[%%{{{
	%draft, % Entwurfsstadium, Achtung: links funktionieren nicht
	final, % fertiges Dokument
	11pt, % Grundschriftgroesse (Standard)
	normalheadings, % keine grossen Ueberschriften wie es Standard waere
%	ngerman, % wird an andere Pakete weitergereicht
	a4paper,
%	BCOR5mm, % Bindekorrektur fuer Rand auf der Innenseite
%	DIV11, % Seitengroesse (siehe Koma Skript Dokumentation !)
	1.1headlines, % Zeilenanzahl der Kopfzeilen
	pagesize, % Schreibt die Papiergroesse in die Datei.
%	twoside, % Seitenraender fuer zweiseitiges Layout
	oneside, % kein zweis. Layout, besser fuer das lesen am Bildschirm
%	openright, % Kapitel beginnen immer auf der rechten Seite
	titlepage, % Titel als einzelne Seite ('titlepage' Umgebung)
%	parindent, % Absaetze eingerueckt (Standard)
	halfparskip, % Absaetze getrennt durch halbe Leerzeile, keine Einrueckung
	headsepline, % Linie unter Kolumnentitel
	nochapterprefix, % keine Ausgabe von 'Kapitel:'
	bibtotoc, % Bibliographie ins TOC
	tocindent, % eingerueckte Gliederung
	listsindent, % eingereuckte LOT, LOF
	pointlessnumbers, % ueberschriftnummerierung ohne Punkt, siehe DUDEN
%	fleqn, % Formeln werden linksbuendig angezeigt
]{scrbook} % moegl. Klassen: scrartcl, scrreprt, scrbook%}}}
% -----------------------------------------------------------------------

%@proceedings{DBLP:conf/fse/2004,
%   editor    = {Bimal K. Roy and Willi Meier},
%   title     = {Fast Software Encryption, 11th International Workshop, FSE
%                2004, Delhi, India, February 5-7, 2004, Revised Papers},
%   booktitle = {FSE},
%   publisher = {Springer},
%   series    = {Lecture Notes in Computer Science},
%   volume    = {3017},
%   year      = {2004},
%   isbn      = {3-540-22171-9},
%   bibsource = {DBLP, http://dblp.uni-trier.de}
%}

% für lstlisting verzeichnis


% ------------------------------------------------------------------------
% LaTeX - Preambel  ******************************************************
% ------------------------------------------------------------------------
% von: Matthias Pospiech
% ========================================================================
%%&TODO: mathpackages
% ~~~~~~~~~~~~~~~~~~~~~~~~~~~~~~~~~~~~~~~~~~~~~~~~~~~~~~~~~~~~~~~~~~~~~~~~
% Einige Pakete muessen unbedingt vor allen anderen geladen werden
% ~~~~~~~~~~~~~~~~~~~~~~~~~~~~~~~~~~~~~~~~~~~~~~~~~~~~~~~~~~~~~~~~~~~~~~~~
%
\usepackage{xspace} % Define commands that don't eat spaces.
\usepackage{ifpdf} %\ifpdf \else \fi
\usepackage{calc} % Calculation with LaTeX
\usepackage[english]{babel} % Languagesetting / change to ngerman if needed
%\usepackage[table]{xcolor} % Farben
\usepackage[usenames,dvipsnames]{color} % Farben
\usepackage[]{graphicx} % Bilder
\usepackage{epstopdf} %% If an eps image is detected, epstopdf is automatically called to convert it to pdf format.
\usepackage[]{amsmath} % Amsmath - Mathematik Basispaket
\usepackage{ragged2e} % Besserer Flatternsatz (Linksbuendig, statt Blocksatz)
\usepackage{array}

% ~~~~~~~~~~~~~~~~~~~~~~~~~~~~~~~~~~~~~~~~~~~~~~~~~~~~~~~~~~~~~~~~~~~~~~~~
% Fonts Fonts Fonts
% ~~~~~~~~~~~~~~~~~~~~~~~~~~~~~~~~~~~~~~~~~~~~~~~~~~~~~~~~~~~~~~~~~~~~~~~~

\usepackage[T1]{fontenc} % T1 Schrift Encoding (notwendig für die meisten Type 1 Schriften)
\usepackage{textcomp}	 % Zusätzliche Symbole (Text Companion font extension)

%% - Latin Modern
\usepackage{lmodern}
%% -------------------
%
%% - Times, Helvetica, Courier (Word Standard...)
%\usepackage{mathptmx}
%\usepackage[scaled=.90]{helvet}
%\usepackage{courier}
%% -------------------
%%
%% - Palantino , Helvetica, Courier
%\usepackage{mathpazo}
%\usepackage[scaled=.95]{helvet}
%\usepackage{courier}
%% -------------------
%
%% - Bera Schriften
%\usepackage{bera}
%% -------------------
%
%% - Charter, Bera Sans
%\usepackage{charter}
\linespread{1.2}
%\renewcommand{\sfdefault}{fvs}



% ~~~~~~~~~~~~~~~~~~~~~~~~~~~~~~~~~~~~~~~~~~~~~~~~~~~~~~~~~~~~~~~~~~~~~~~~
% Math Packages
% ~~~~~~~~~~~~~~~~~~~~~~~~~~~~~~~~~~~~~~~~~~~~~~~~~~~~~~~~~~~~~~~~~~~~~~~~

\usepackage[fixamsmath,disallowspaces]{mathtools} % Erweitert amsmath und behebt einige Bugs
%%%\usepackage{fixmath}
%%%\usepackage[all,warning]{onlyamsmath} % Warnt bei Benutzung von Befehlen die mit amsmath inkompatibel sind.
\usepackage{icomma} % Erlaubt die Benutzung von Kommas im Mathematikmodus

% ~~~~~~~~~~~~~~~~~~~~~~~~~~~~~~~~~~~~~~~~~~~~~~~~~~~~~~~~~~~~~~~~~~~~~~~~
% Symbole
% ~~~~~~~~~~~~~~~~~~~~~~~~~~~~~~~~~~~~~~~~~~~~~~~~~~~~~~~~~~~~~~~~~~~~~~~~
\usepackage{amssymb}
%\usepackage{wasysym}
%\usepackage{marvosym}
%\usepackage{pifont}

% ~~~~~~~~~~~~~~~~~~~~~~~~~~~~~~~~~~~~~~~~~~~~~~~~~~~~~~~~~~~~~~~~~~~~~~~~
% Tables (Tabular)
% ~~~~~~~~~~~~~~~~~~~~~~~~~~~~~~~~~~~~~~~~~~~~~~~~~~~~~~~~~~~~~~~~~~~~~~~~

\usepackage{booktabs}
\usepackage{tabularx} % tabularx nach hyperref laden

% ~~~~~~~~~~~~~~~~~~~~~~~~~~~~~~~~~~~~~~~~~~~~~~~~~~~~~~~~~~~~~~~~~~~~~~~~
% text related packages
% ~~~~~~~~~~~~~~~~~~~~~~~~~~~~~~~~~~~~~~~~~~~~~~~~~~~~~~~~~~~~~~~~~~~~~~~~

\usepackage{url} % Setzen von URLs. In Verbindung mit hyperref sind diese auch aktive Links.
\usepackage[stable,perpage, ragged,  multiple]{footmisc} % Fussnoten
\usepackage[english]{varioref} % Intelligente Querverweise /change to ngerman if needed
\usepackage{enumitem} % Listen

% ~~~~~~~~~~~~~~~~~~~~~~~~~~~~~~~~~~~~~~~~~~~~~~~~~~~~~~~~~~~~~~~~~~~~~~~~
% Pakete zum Zitieren
% ~~~~~~~~~~~~~~~~~~~~~~~~~~~~~~~~~~~~~~~~~~~~~~~~~~~~~~~~~~~~~~~~~~~~~~~~

\usepackage[babel, english=british, german=quotes, french=guillemets]{csquotes} % clever quotations
\SetBlockThreshold{2} % Anzahl von Zeilen
\newenvironment{myquote}%
	{\begin{quote}\small}%
	{\end{quote}}%
\SetBlockEnvironment{myquote}

% Zitate =================================================================
\usepackage[%
	square,	% for square brackets;
	comma,	% to use commas as separaters;
	numbers,	% for numerical citations;
	sort,		% orders multiple citations into the sequence in which they appear in the list of references;
	sort&compress,    % as sort but in addition multiple numerical citations
]{natbib}

%%% Bibliography styles created with custombib
%%% Doc: ftp://tug.ctan.org/pub/tex-archive/macros/latex/contrib/custom-bib/makebst.pdf
%%%\bibliographystyle{bib/bst/AlphaDINFirstName}
%\bibstyle{plainnat}

% ~~~~~~~~~~~~~~~~~~~~~~~~~~~~~~~~~~~~~~~~~~~~~~~~~~~~~~~~~~~~~~~~~~~~~~~~
% PDF related packages
% ~~~~~~~~~~~~~~~~~~~~~~~~~~~~~~~~~~~~~~~~~~~~~~~~~~~~~~~~~~~~~~~~~~~~~~~~
\usepackage{pdfpages} % Include pages from external PDF documents in LaTeX documents

% ~~~~~~~~~~~~~~~~~~~~~~~~~~~~~~~~~~~~~~~~~~~~~~~~~~~~~~~~~~~~~~~~~~~~~~~~
% figures and placement
% ~~~~~~~~~~~~~~~~~~~~~~~~~~~~~~~~~~~~~~~~~~~~~~~~~~~~~~~~~~~~~~~~~~~~~~~~

%% Bilder und Graphiken ==================================================

\usepackage{float}             % Stellt die Option [H] fuer Floats zur Verfgung
\usepackage{flafter}          % Floats immer erst nach der Referenz setzen
\usepackage{subfig} % Layout wird weiter unten festgelegt !
\usepackage{wrapfig}	        % Bilder von Text Umfliessen lassen

% Make float placement easier
\renewcommand{\floatpagefraction}{.75} % vorher: .5
\renewcommand{\textfraction}{.1}       % vorher: .2
\renewcommand{\topfraction}{.8}        % vorher: .7
\renewcommand{\bottomfraction}{.5}     % vorher: .3
\setcounter{topnumber}{3}              % vorher: 2
\setcounter{bottomnumber}{2}           % vorher: 1
\setcounter{totalnumber}{5}            % vorher: 3

% ~~~~~~~~~~~~~~~~~~~~~~~~~~~~~~~~~~~~~~~~~~~~~~~~~~~~~~~~~~~~~~~~~~~~~~~~
% science packages
% ~~~~~~~~~~~~~~~~~~~~~~~~~~~~~~~~~~~~~~~~~~~~~~~~~~~~~~~~~~~~~~~~~~~~~~~~

\usepackage{units}

% ~~~~~~~~~~~~~~~~~~~~~~~~~~~~~~~~~~~~~~~~~~~~~~~~~~~~~~~~~~~~~~~~~~~~~~~~
% layout packages
% ~~~~~~~~~~~~~~~~~~~~~~~~~~~~~~~~~~~~~~~~~~~~~~~~~~~~~~~~~~~~~~~~~~~~~~~~

%% Zeilenabstand =========================================================
%
%%% Doc: ftp://tug.ctan.org/pub/tex-archive/macros/latex/contrib/setspace/setspace.sty
\usepackage{setspace}
%\doublespace	        % 2-facher Abstand
%\onehalfspacing        % 1,5-facher Abstand
% hereafter load 'typearea' again

%% Seitenlayout ==========================================================
%
% Layout mit 'typearea'
\typearea[current]{last}
\raggedbottom     % Variable Seitenhoehen zulassen


%% Kopf und Fusszeilen====================================================
%%% Doc: ftp://tug.ctan.org/pub/tex-archive/macros/latex/contrib/koma-script/scrguide.pdf
\usepackage[%
   automark,         % automatische Aktualisierung der Kolumnentitel
   nouppercase,      % Grossbuchstaben verhindern
]{scrpage2}

\pagestyle{scrheadings} % Seite mit Headern
%\pagestyle{scrplain} % Seiten ohne Header
%\pagestyle{empty} % Seiten ohne Header

% loescht voreingestellte Stile
\clearscrheadfoot
%\clearscrheadings
\clearscrplain
%
%\ohead{\pagemark} % Oben aussen: Seitenzahlen
\cfoot{\pagemark} % unten mitte...
%\ihead{\headmark} % Oben innen: Kapitel und Section
\chead{\headmark} % Oben mitte: Kapitel und Section

% Angezeigte Abschnitte im Header
\automark[section]{chapter} %[rechts]{links}
%
\setheadsepline{.4pt}[\color{black}] % Linie zwischen Kopf und Textk�rper

%% Fussnoten =============================================================
% Keine hochgestellten Ziffern in der Fussnote (KOMA-Script-spezifisch):
\deffootnote{1.5em}{1em}{\makebox[1.5em][l]{\thefootnotemark}}
\addtolength{\skip\footins}{\baselineskip} % Abstand Text <-> Fussnote
\setlength{\dimen\footins}{10\baselineskip} % Beschraenkt den Platz von Fussnoten auf 10 Zeilen
\interfootnotelinepenalty=10000 % Verhindert das Fortsetzen von
                                % Fussnoten auf der gegenüberligenden Seite

%% Schriften (Sections )==================================================

% -- Koma Schriften --
\newcommand\SectionFontStyle{\sffamily}
\setkomafont{chapter}{\huge\SectionFontStyle}    % Chapter
\setkomafont{sectioning}{\SectionFontStyle} %  % Titelzeilen % \bfseries
%\setkomafont{pagenumber}{\bfseries\SectionFontStyle}             % Seitenzahl
\setkomafont{pagenumber}{\normalfont\SectionFontStyle}             % Seitenzahl nicht fett...
\setkomafont{pagehead}{\itshape\small\sffamily}        % Kopfzeile, beeinflusst aber auch fusszeile...
\setkomafont{descriptionlabel}{\itshape}        % Kopfzeile
%
\renewcommand*{\raggedsection}{\raggedright} % Titelzeile linksbuendig, haengend
%

%% Captions (Schrift, Aussehen) ==========================================

%%% Doc: ftp://tug.ctan.org/pub/tex-archive/macros/latex/contrib/caption/caption.pdf
\usepackage{caption}
% Aussehen der Captions
\captionsetup{
   margin = 10pt,
   font = {small,rm},
   labelfont = {small,bf},
   format = plain, % default oder 'hang'
   indention = 0em,  % Einruecken der Beschriftung
   labelsep = colon, %period, space, quad, newline
   justification = RaggedRight, % justified, centering
   singlelinecheck = true, % false (true=bei einer Zeile immer zentrieren)
   position = bottom %top
}
%%% Bugfix Workaround
\DeclareCaptionOption{parskip}[]{}
\DeclareCaptionOption{parindent}[]{}

%\DeclareGrphicsExtensions{.pdf,.png,.jpg}

% Aussehen der Captions fuer subfigures (subfig-Paket)
\captionsetup[subfloat]{%
   margin = 10pt,
   font = {small,rm},
   labelfont = {small,bf},
   format = plain, % default oder 'hang'
   indention = 0em,  % Einruecken der Beschriftung
   labelsep = space, %period, space, quad, newline
   justification = RaggedRight, % justified, centering
   singlelinecheck = true, % false (true=bei einer Zeile immer zentrieren)
   position = bottom, %top
   labelformat = parens % simple, empty % Wie die Bezeichnung gesetzt wird
 }

%% Inhaltsverzeichnis (Schrift, Aussehen) sowie weitere Verzeichnisse ====

\setcounter{secnumdepth}{2}    % Abbildungsnummerierung mit groesserer Tiefe
\setcounter{tocdepth}{2}		 % Inhaltsverzeichnis mit groesserer Tiefe
%

% Auszufuehrende Befehle  ------------------------------------------------

\listfiles
%------------------------------------------------------------------------


\newcolumntype{b}{>{\global\let\currentrowstyle\relax}}
\newcolumntype{n}{>{\currentrowstyle}}
\newcommand{\rowstyle}[1]{\gdef\currentrowstyle{#1}%
  #1\ignorespaces
}


% Silbentrennung
\hyphenation{}

%für quellcode
\usepackage{listings}
\lstset{
    language=c#,
    basicstyle=\ttfamily\small,
    frame=single,       % einfacher rahmen um quellcode 
    %frameround=fttt,   % runder rahmen für rahmentyp frame
    %frame=trBL         % komplexer rahmen um quellcode
    %frame=lines,        % rahmen nur unten und oben
    %numbers=left,
    %numberstyle=tiny
    %backgroundcolor=\color{lightgray}
    %keywordstyle=\color{orange}\bfseries,
    %keywordstyle=\color{RoyalBlue}\bfseries,
    keywordstyle=\color{Black}\bfseries,
    commentstyle=\color{darkgray},
    stringstyle=\color{red}
}

%
% WORKAROUND, damit lstlistoflistings funktioniert:
% Quelle: http://www.komascript.de/node/477
%
\makeatletter% --> De-TeX-FAQ
\renewcommand*{\lstlistoflistings}{%
\begingroup
\if@twocolumn
\@restonecoltrue\onecolumn
\else
\@restonecolfalse
\fi
\lol@heading
\setlength{\parskip}{\z@}%
\setlength{\parindent}{\z@}%
\setlength{\parfillskip}{\z@ \@plus 1fil}%
\@starttoc{lol}%
\if@restonecol\twocolumn\fi
\endgroup
}
\makeatother% --> \makeatletter
%für tabellen
\usepackage{multirow}
%\newcolumntype{C}[1]{>{\centering}m{#1}}
%\usepackage{tabularx}


% für TODO-Notes
\usepackage{color}
\usepackage{tikz}

% für mathematische symbole
\usepackage{amsmath}
\usepackage{amssymb}
\usepackage{bm}

% für Einbinden von .pdf's
\usepackage{pdfpages}

% wegen deutschen Umlauten
\usepackage[utf8]{inputenc}

% für Seitenlayout
\pagestyle{useheadings}

\usepackage{amsthm}
% für Zeilennummern
\usepackage{lineno}
% Einschalten der ZN
%%%\linenumbers
% Setzen der ZN
%%%\modulolinenumbers[1]

% BEGIN TODO-Notes support
%{{{
\makeatletter \newcommand \listoftodos{\section*{Todo list} \@starttoc{tdo}}
\newcommand\l@todo[2]
    {\par\noindent \textit{#2}, \parbox{10cm}{#1}\par} \makeatother

\definecolor{orange}{rgb}{1,0.5,0}
\tikzstyle{notestyle} = [draw=black, fill=orange, text width = 2.5cm]
\tikzstyle{notestyleleft} = [notestyle]
\tikzstyle{connectstyle} = [draw = orange, thick]
% Command for inserting a todo item
\newcommand{\todo}[1]{%
% Add to todo list
\addcontentsline{tdo}{todo}{\protect{#1}}%
%
\begin{tikzpicture}[remember picture, baseline=-0.75ex]%
    \node [coordinate] (inText) {};
\end{tikzpicture}%
%
% Make the margin par
\marginpar[%
{% Draw note in left margin
    \tikz[remember picture] \draw node[notestyleleft] (inNote) {#1};%
    \begin{tikzpicture}[remember picture, overlay]%
        \draw[connectstyle]
            ([yshift=-0.2cm] inText)
                -| ([xshift=0.2cm] inNote.east)
                -| (inNote.east);
    \end{tikzpicture}%
}%
]{% Draw note in right margin
    \tikz[remember picture] \draw node[notestyle] (inNote) {#1};%
    \begin{tikzpicture}[remember picture, overlay]%
        \draw[connectstyle]
            ([yshift=-0.2cm] inText)
                -| ([xshift=-0.2cm] inNote.west)
                -| (inNote.west);
    \end{tikzpicture}%
}%
}%
%}}}
% END TODO-Notes support

\usepackage{acronym}
\usepackage{amsfonts}
\usepackage{amsmath}

\usepackage{graphics, epsfig, psfrag}

% listing definitions from cf

\newtheorem{definition}{Definition}
\newtheorem{proposition}{Proposition}
\newtheorem{lemma}{Lemma}
\newtheorem{theorem}{Theorem}

\begin{document}

\lstdefinelanguage{L}
  {morekeywords={Oracle,if,then,else,Finalize,return,Initialize,Encrypt,Decrypt,
  false,Hash,true,Tag,Verify,AuxVerify,Extract,and,or,AuxDecrypt,Append,for,do,
  Encryption,Decryption,GenerateCoins,in},
  sensitive=false,
  morecomment=[l]{//},
  morecomment=[s]{/*}{*/},
  morecomment=[s]{(*}{*)},
}
\lstset{mathescape=true,language=L,basicstyle=\small,frame=none}
\lstset{numbers=left, numberstyle=\tiny, stepnumber=1, numbersep=5pt}
\lstset{emph={GenerateCoins,Oracle,Finalize,Initialize,Encrypt,Hash,Decrypt,Verify,Tag},emphstyle={\bfseries\underbar}}



\begin{titlepage}
\large
\noindent
Bauhaus-Universität Weimar\\
Faculty of Media\\
Degree Program Computer Science and Media\\
\author{Michael Pannier}
\title{Can't touch this}
\vspace{20mm}
\begin{center}
    \huge{\bfseries{Can't touch this -\\
		A Prototype for Public Pointing Interaction}}
\end{center}
\vspace{15mm}
\begin{center}
    \huge{\bfseries{Master Thesis}}\\
\end{center}
\vspace{20mm}
Michael Frank Pannier
\hfill Registration Number 51755\\
born 19th December 1984 in Dessau\\
\newline
\newline
1st Supervisor: Prof. Dr. Eva Hornecker\\
2nd Supervisor: Prof. Dr. Sven Bertel\\
\vfil
\noindent
Date of Submission: 17th November 2014\\
\end{titlepage}

\frontmatter

\chapter*{}
\section*{Acknowledgement}

Great thanks goes to my parents Annemarie and Joachim and my brothers Justus and Jan, who have supported me during the past two years, morally and financially.
Furthermore, I would like to thank the owner of the Chair of Media Security professor Stefan Lucks, who has given me the opportunity to work in the
ever-growing reasearch area of cryptography. I owe my gratitude to my advisor Christian Forler, who, with his constructive
criticism and helpful remarks, helped guide this thesis to its proper destination. I also thank Christof Bräutigam, Ewan Fleischmann, Lars Harmsen,
Alexander Kümmel, Eik List, Thomas Knapke, Michael Pannier und Michael Völske, for their interminable support during the time
I spent working on this thesis, as well as Benno Stein for co-supervising it.


\vfill

\section*{Danksagung}

Mein größter Dank geht an meine Eltern Annemarie und Joachim und meine Brüder Justus und Jan, die mich während der letzten zwei Jahre moralisch und finanziell unterstützt haben.
Danken möchte ich dem Inhaber des Lehrstuhls für Mediensicherheit Professor Stefan Lucks, der mir die Möglichkeit gegeben hat, in dem
stetig wachsenden Forschungsgebiet der Kryptographie zu arbeiten. Meinem Betreuer Christian Forler gehört Dank, denn mit seiner konstruktiven
Kritik und seinen guten Anmerkungen hat er diese Arbeit zu einem guten Ziel geführt.
Weiter danke ich Christof Bräutigam, Ewan Fleischmann, Lars Harmsen, Alexander Kümmel, Eik List, Thomas Knapke, Michael Pannier und Michael Völske, die mir
während meiner Bearbeitungszeit stets mit Rat und Tat zur Seite standen sowie meinem Zweitbetreuer Professor Benno Stein.


\chapter{Abstract}
\label{abstract}

%\subsection*{New Version}

Museums tend to be perceived as old fashioned. At least, that is what some people assume and therefore not even consider having a look for themselves. Nevertheless, there are many modern and open minded ones, which are willing to experiment with new possibilities, to get rid of their dusted reputation and to evolve.
\\
So, I was called to do exactly that. -- Implement a novel informatory interaction system for a museum of pre- and protohistoric history, where precious artifacts are locked up behind thick glass. The challenge was not only to develop a working prototype, but also make it intuitive, low maintenance and robust enough for everyday use. The system I developed employs the natural behavior of visitors. It detects potential users and enables them to interact with the system via pointing-gestures. Moreover, it can easily been set up and altered by museum personnel.

%\subsection*{Annotations}
%
%\begin{itemize}
	%\item Exciting summary
	%\item Create interest 
%\end{itemize}
%
%\subsection*{Old version}
%
%Three dimensional (3D) graphics are a common sight in modern media, while two dimensional techniques are widely used for interaction. In virtual reality, several 3D devices are used to navigate and manipulate the virtual contents. Nevertheless, they are often not easy to use or error prone. At the same time, home entertainment systems (i.e. Kinect) can be operated with simple hand gestures. Hence, a novel interaction prototype has been developed as an interactive museum information (IMI)-system. Here, users are tracked with an ASUS Xtion-motion sensor. Gestures can be analyzed using the OpenNI framework. By simply pointing at it, a user then describes ones interest in a specific exhibit and the software will provide further information regarding the exhibit. The IMI-System is a low cost and maintenance system. Thus, the museum's staff defines and edits the objects of interest and the corresponding information themselves.


\tableofcontents
\listoffigures
%keine seitenzahl auf dem deckblatt
%\pagestyle{empty}

\chapter*{Abbrevations}
\begin{acronym}[\hspace{2cm}]
  %\acro{Abkuerzung}{ausgeschrieben}
	\acro{IMI}{Interactive Museum Installation}
	\acro{MS}{Microsoft}
	\acro{RFID}{Radio-Frequency Identification}
	\acro{FSD}{Functional Specification Document}
	\acro{MIT}{Massachusetts Institute of Technology}
	\acro{SDMS}{Spacial Data-Management System}
	\acro{WYSIWYG}{''What you see is what you get''}
	\acro{GUI}{Graphical User Interface}
	\acro{SUI}{Single-User Interface}
	\acro{MUI}{Multi-User Interface}		
	\acro{HCI}{Human Computer-Interaction}
	\acro{TUI}{Tangible User Interface}
	\acro{VR}{Virtual Reality}
	\acro{3D}{three-dimensional}
	\acro{HMD}{head-mounted display}
	\acro{DOF}{degrees of freedom}
	\acro{AR}{Augmented Reality}
	\acro{SDK}{Software Development Kit}
	\acro{CAVE}{Cave Automatic Virtual Environment}
	\acro{2D}{two-dimensional}
	\acro{BCI}{Brain-Computer Interface}
	\acro{MVT}{Museumsverband Thüringen}
	\acro{HDD}{Hard Disk Drive}
	\acro{PDLC}{Polymer Dispersed Liquid Crystal}
	\acro{IR}{infra-red}
	\acro{FUBI}{Full Body Interaction}
	\acro{UI}{User Interface}
	\acro{wpm}{words per minute}
	\acro{cpm}{characters per minute}
	\acro{IV}{Independent Variable}
	\acro{SD}{Standard Deviation}
	\acro{ID}{Identificator}
	\acro{AOA}{Area of Affinity}
	\acro{LOS}{Length of Stay}
	\acro{n/s}{not specified}
	\acro{SUS}{Standard Usability Scale}
\end{acronym}


\mainmatter

\chapter{Introduction}
\label{introduction}

%\begin{itemize}
	%\item Liste
	%\begin{itemize}
		%\item mit
			%\begin{itemize}
				%\item Unterpunkten
			%\end{itemize}
	%\end{itemize}
%\end{itemize}
%
%\begin{enumerate}
	%\item Aufzählung
	%\begin{enumerate}
		%\item mit
			%\begin{enumerate}
				%\item Unterpunkten
			%\end{enumerate}
	%\end{enumerate}
%\end{enumerate}
%
%\begin{description}
	%\item[Begriff] Erklärung von Begriff
%\end{description}

\paragraph{Introduction - Annotations}

\begin{itemize}
	\item Short overview, about what has been build
	\item Summary
	\\
	\item System of libraries for pointing interaction
	\item Information system (Information On Demand)
	\item 'Uncharted territory' $\to$ technical focus
	\item Template solution / 'just a proof of concept'
  \\
	\item Motivation
	\item Working within the confines of museums respectively public installations
	\\
	\item \ac{IMI}
	\item blub
	\item \ac{IMI}
\end{itemize}

%------------------------------------------------------------------------------------------

\paragraph{Related Work - Annotations}

\begin{itemize}
	\item Backgrounds
	\begin{itemize}
		\item Historical
		\item Technical
	\end{itemize}
	\item Application areas
	\item Not to much detail
	\item Only in respect to the thesis' topic
\end{itemize}

%------------------------------------------------------------------------------------------

\section{Museums}
\label{related_work_museums}

\paragraph{Annotations}

\begin{itemize}
	\item Historical evolution
	\begin{itemize}
		\item Museums are believed to be old fashioned
		\item Mostly willing to experiment (Examples)
		\begin{itemize}
			\item Dioramas
			\item ...
			\item Animatronics
			\item Robotics
		\end{itemize}
	\end{itemize}
\end{itemize}

%------------------------------------------------------------------------------------------

\subsection*{Old version}

Museums, much like libraries, are foremost seen as a place of knowledge and its preservation. Hence, visitors behave in a very reserved manner. Whereas this may apply for a library, museums are willing to involve people instead of merely providing information. Many Museums therfore employ guides, who give tours and tell visitors about the exhibits. In addition to their factual knowledge, they can also provide anecdotes and other information needed to bond with a certain topic. Apart of instructive and teaching staff, museums have tried many other ways to involve their visitors more. One of those is employing technology. With time technology evolved, and so did technological augmentations in museums.
\\
It may have started with simple mechanics, which moved some models, and later included basic electronics, which illuminated particular exhibits. Microchips and computers became more and more popular and affordable. So, the next step was immanent. There were info-terminals (...) Yet another chapter was opened, when the internet and wireless communication were introduced. Burgard et al. build an autonomous tour-guide robot called RHINO. It was able to navigate through the museum freely without bumping into visitors. RHINO could be used as a tour-guide for present visitors as well as for visitors on the internet, for it had a simple build-in and a web interface [Bur98]. RHINO was deployed at the “Deutsches Museum Bonn” in 1998.
\\
In 2002, a group from the University of Limmerick made a survey in the Hunt Museum. The
museum is owned and run by the Hunt family, whose tradition it was from the beginning to involve the visitors. Therefore, they had so-called cabinets of curiosity [Cio02], special compartments within the exhibition, where additional exhibits were hidden. For example, one had to open drawers in order to find a collection of plates. Via this exploration, the visitors became involved. Inspired by their observations, Ciolfi et al. implemented a completely new and interactive part of the exhibition in 2005. Two new rooms were  introduced. First, there was a study room with three interactive devices for getting further information about certain exhibits. They were disguised as a chest, a painting and a desk. The second room, the room of opinion, was plain white with plinths, on which visitors could record their interpretations of intended function of certain exhibits. In  order to manage all the data, a third and hidden room was used to host all the data-servers [Cio05].
\\
Something about [Hor06].

%------------------------------------------------------------------------------------------

\section{Public single-user interfaces}
\label{related_work_single}

\paragraph{Annotations}

\begin{itemize}
	\item Human behavior concerning public interfaces
	\begin{itemize}
		\item self-service at train-stations
		\item public interfaces, such as Tobias Fischer's \textit{SMS-Schleuder für Fassaden}
		\item Intuitive usage vs. inhibition
	\end{itemize}
\end{itemize}

\section{Tangible Interfaces}
\label{related_work_tangible}

\paragraph{Annotations}

\begin{itemize}
	\item Technologies for input / interaction
	\item Hands-free
	\item Gestural interaction (Kinect)
\end{itemize}

%------------------------------------------------------------------------------------------

\section{Virtual Reality}
\label{related_work_vr}

\paragraph{Annotations}

\begin{itemize}
	\item Input
	\begin{itemize}
		\item Metaphors and devices
		\begin{itemize}
			\item Navigation and selection in 3d space
			\item Possibilities
			\item Difficulties
			\item Constraints
		\end{itemize}
	\end{itemize}
\end{itemize}



\bibliographystyle{alpha} % plain = [1], alpha = [DKS01]
\bibliography{../quellen/references}
\chapter*{Affidavit}
\subsubsection{Eidesstattliche Erklärung}

Hiermit versichere ich, dass ich die Masterarbeit selbstständig verfasst
und keine anderen als die angegebenen Quellen und Hilfsmittel benutzt habe,
alle Ausführungen, die anderen Schriften wörtlich oder sinngemäß entnommen
wurden, kenntlich gemacht sind und die Arbeit in gleicher oder ähnlicher
Fassung noch nicht Bestandteil einer Studien- oder Prüfungsleistung war.

\subsubsection{Affidavit}
I hereby declare that this master thesis has been written only by the undersigned and without any assistance from third parties.
Furthermore, I confirm that no sources have been used in the preparation of this thesis other than those indicated in the thesis itself, as well
as that the thesis has not yet been handled in neither in this nor in equal form  at any other official comission.

\vspace{1.5cm}
\line(1,0){140}\\
Jakob Wenzel


\appendix
%\chapter{Appendix}
\chapter{Observation and Interview Sheet}
\label{appendix_box}

\begin{figure}[H]%
\includegraphics[width=\columnwidth]{../pics/jewelbox.eps}%
\caption*{Image of the jewel box placed by the feet of the princess inside the Ha\ss leben-showcase.}%
%\label{fig:biatch} 
\end{figure}

%-----------------------------------------------------------------------------

%\chapter{Observation and Interview Form}
%\label{appendix_form}

\begin{figure}[H]%
\includegraphics[width=0.75\columnwidth]{../pics/interview-a.eps}%
\caption*{First page of the observation and interview form of the pre- and main study.}%
%\label{fig:interview1} 
\end{figure}

\begin{figure}[H]%
\includegraphics[width=0.75\columnwidth]{../pics/interview-b.eps}%
\caption*{Second page of the observation and interview form of the pre- and main study.}%
%\label{fig:interview2} 
\end{figure}

%-----------------------------------------------------------------------------

%\chapter{SUS-Questionnaire}
%\label{appendix_sus}

\begin{figure}[H]%
\includegraphics[width=0.75\columnwidth]{../pics/sus-quest.eps}%
\caption*{SUS-questionnaire that was handed out to participants after the interview.}%
%\label{fig:sus-quest} 
\end{figure}


\end{document}
