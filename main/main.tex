% vim: foldmethod=marker

%% Dokumentenklasse (Koma Script) ---------------------------------------
\documentclass[%%{{{
	%draft, % Entwurfsstadium, Achtung: links funktionieren nicht
	final, % fertiges Dokument
	11pt, % Grundschriftgroesse (Standard)
	normalheadings, % keine grossen Ueberschriften wie es Standard waere
%	ngerman, % wird an andere Pakete weitergereicht
	a4paper,
%	BCOR5mm, % Bindekorrektur fuer Rand auf der Innenseite
%	DIV11, % Seitengroesse (siehe Koma Skript Dokumentation !)
	1.1headlines, % Zeilenanzahl der Kopfzeilen
	pagesize, % Schreibt die Papiergroesse in die Datei.
%	twoside, % Seitenraender fuer zweiseitiges Layout
	oneside, % kein zweis. Layout, besser fuer das lesen am Bildschirm
%	openright, % Kapitel beginnen immer auf der rechten Seite
	titlepage, % Titel als einzelne Seite ('titlepage' Umgebung)
%	parindent, % Absaetze eingerueckt (Standard)
	halfparskip, % Absaetze getrennt durch halbe Leerzeile, keine Einrueckung
	headsepline, % Linie unter Kolumnentitel
	nochapterprefix, % keine Ausgabe von 'Kapitel:'
	bibtotoc, % Bibliographie ins TOC
	tocindent, % eingerueckte Gliederung
	listsindent, % eingereuckte LOT, LOF
	pointlessnumbers, % ueberschriftnummerierung ohne Punkt, siehe DUDEN
%	fleqn, % Formeln werden linksbuendig angezeigt
]{scrbook} % moegl. Klassen: scrartcl, scrreprt, scrbook%}}}
% -----------------------------------------------------------------------

%@proceedings{DBLP:conf/fse/2004,
%   editor    = {Bimal K. Roy and Willi Meier},
%   title     = {Fast Software Encryption, 11th International Workshop, FSE
%                2004, Delhi, India, February 5-7, 2004, Revised Papers},
%   booktitle = {FSE},
%   publisher = {Springer},
%   series    = {Lecture Notes in Computer Science},
%   volume    = {3017},
%   year      = {2004},
%   isbn      = {3-540-22171-9},
%   bibsource = {DBLP, http://dblp.uni-trier.de}
%}

% für lstlisting verzeichnis


% ------------------------------------------------------------------------
% LaTeX - Preambel  ******************************************************
% ------------------------------------------------------------------------
% von: Matthias Pospiech
% ========================================================================
%%&TODO: mathpackages
% ~~~~~~~~~~~~~~~~~~~~~~~~~~~~~~~~~~~~~~~~~~~~~~~~~~~~~~~~~~~~~~~~~~~~~~~~
% Einige Pakete muessen unbedingt vor allen anderen geladen werden
% ~~~~~~~~~~~~~~~~~~~~~~~~~~~~~~~~~~~~~~~~~~~~~~~~~~~~~~~~~~~~~~~~~~~~~~~~
%
\usepackage{xspace} % Define commands that don't eat spaces.
\usepackage{ifpdf} %\ifpdf \else \fi
\usepackage{calc} % Calculation with LaTeX
\usepackage[english]{babel} % Languagesetting / change to ngerman if needed
%\usepackage[table]{xcolor} % Farben
\usepackage[usenames,dvipsnames]{color} % Farben
\usepackage[]{graphicx} % Bilder
\usepackage{epstopdf} %% If an eps image is detected, epstopdf is automatically called to convert it to pdf format.
\usepackage[]{amsmath} % Amsmath - Mathematik Basispaket
\usepackage{ragged2e} % Besserer Flatternsatz (Linksbuendig, statt Blocksatz)
\usepackage{array}

% ~~~~~~~~~~~~~~~~~~~~~~~~~~~~~~~~~~~~~~~~~~~~~~~~~~~~~~~~~~~~~~~~~~~~~~~~
% Fonts Fonts Fonts
% ~~~~~~~~~~~~~~~~~~~~~~~~~~~~~~~~~~~~~~~~~~~~~~~~~~~~~~~~~~~~~~~~~~~~~~~~

\usepackage[T1]{fontenc} % T1 Schrift Encoding (notwendig für die meisten Type 1 Schriften)
\usepackage{textcomp}	 % Zusätzliche Symbole (Text Companion font extension)

%% - Latin Modern
\usepackage{lmodern}
%% -------------------
%
%% - Times, Helvetica, Courier (Word Standard...)
%\usepackage{mathptmx}
%\usepackage[scaled=.90]{helvet}
%\usepackage{courier}
%% -------------------
%%
%% - Palantino , Helvetica, Courier
%\usepackage{mathpazo}
%\usepackage[scaled=.95]{helvet}
%\usepackage{courier}
%% -------------------
%
%% - Bera Schriften
%\usepackage{bera}
%% -------------------
%
%% - Charter, Bera Sans
%\usepackage{charter}
\linespread{1.2}
%\renewcommand{\sfdefault}{fvs}



% ~~~~~~~~~~~~~~~~~~~~~~~~~~~~~~~~~~~~~~~~~~~~~~~~~~~~~~~~~~~~~~~~~~~~~~~~
% Math Packages
% ~~~~~~~~~~~~~~~~~~~~~~~~~~~~~~~~~~~~~~~~~~~~~~~~~~~~~~~~~~~~~~~~~~~~~~~~

\usepackage[fixamsmath,disallowspaces]{mathtools} % Erweitert amsmath und behebt einige Bugs
%%%\usepackage{fixmath}
%%%\usepackage[all,warning]{onlyamsmath} % Warnt bei Benutzung von Befehlen die mit amsmath inkompatibel sind.
\usepackage{icomma} % Erlaubt die Benutzung von Kommas im Mathematikmodus

% ~~~~~~~~~~~~~~~~~~~~~~~~~~~~~~~~~~~~~~~~~~~~~~~~~~~~~~~~~~~~~~~~~~~~~~~~
% Symbole
% ~~~~~~~~~~~~~~~~~~~~~~~~~~~~~~~~~~~~~~~~~~~~~~~~~~~~~~~~~~~~~~~~~~~~~~~~
\usepackage{amssymb}
%\usepackage{wasysym}
%\usepackage{marvosym}
%\usepackage{pifont}

% ~~~~~~~~~~~~~~~~~~~~~~~~~~~~~~~~~~~~~~~~~~~~~~~~~~~~~~~~~~~~~~~~~~~~~~~~
% Tables (Tabular)
% ~~~~~~~~~~~~~~~~~~~~~~~~~~~~~~~~~~~~~~~~~~~~~~~~~~~~~~~~~~~~~~~~~~~~~~~~

\usepackage{booktabs}
\usepackage{tabularx} % tabularx nach hyperref laden

% ~~~~~~~~~~~~~~~~~~~~~~~~~~~~~~~~~~~~~~~~~~~~~~~~~~~~~~~~~~~~~~~~~~~~~~~~
% text related packages
% ~~~~~~~~~~~~~~~~~~~~~~~~~~~~~~~~~~~~~~~~~~~~~~~~~~~~~~~~~~~~~~~~~~~~~~~~

\usepackage{url} % Setzen von URLs. In Verbindung mit hyperref sind diese auch aktive Links.
\usepackage[stable, ragged,  multiple]{footmisc} % Fussnoten 'perpage' entfernt
\usepackage[english]{varioref} % Intelligente Querverweise /change to ngerman if needed
\usepackage{enumitem} % Listen

\usepackage{chngcntr}
\counterwithout{footnote}{chapter} % Fussnoten unabhängig vom chapter hochzaehlen

% ~~~~~~~~~~~~~~~~~~~~~~~~~~~~~~~~~~~~~~~~~~~~~~~~~~~~~~~~~~~~~~~~~~~~~~~~
% Pakete zum Zitieren
% ~~~~~~~~~~~~~~~~~~~~~~~~~~~~~~~~~~~~~~~~~~~~~~~~~~~~~~~~~~~~~~~~~~~~~~~~

\usepackage[babel, english=british, german=quotes, french=guillemets]{csquotes} % clever quotations
\SetBlockThreshold{2} % Anzahl von Zeilen
\newenvironment{myquote}%
	{\begin{quote}\small}%
	{\end{quote}}%
\SetBlockEnvironment{myquote}

% Zitate =================================================================
\usepackage[%
	square,	% for square brackets;
	comma,	% to use commas as separaters;
	numbers,	% for numerical citations;
	sort,		% orders multiple citations into the sequence in which they appear in the list of references;
	sort&compress,    % as sort but in addition multiple numerical citations
]{natbib}

%%% Bibliography styles created with custombib
%%% Doc: ftp://tug.ctan.org/pub/tex-archive/macros/latex/contrib/custom-bib/makebst.pdf
%%%\bibliographystyle{bib/bst/AlphaDINFirstName}
%\bibstyle{plainnat}

% ~~~~~~~~~~~~~~~~~~~~~~~~~~~~~~~~~~~~~~~~~~~~~~~~~~~~~~~~~~~~~~~~~~~~~~~~
% PDF related packages
% ~~~~~~~~~~~~~~~~~~~~~~~~~~~~~~~~~~~~~~~~~~~~~~~~~~~~~~~~~~~~~~~~~~~~~~~~
\usepackage{pdfpages} % Include pages from external PDF documents in LaTeX documents

% ~~~~~~~~~~~~~~~~~~~~~~~~~~~~~~~~~~~~~~~~~~~~~~~~~~~~~~~~~~~~~~~~~~~~~~~~
% figures and placement
% ~~~~~~~~~~~~~~~~~~~~~~~~~~~~~~~~~~~~~~~~~~~~~~~~~~~~~~~~~~~~~~~~~~~~~~~~

%% Bilder und Graphiken ==================================================

\usepackage{float}             % Stellt die Option [H] fuer Floats zur Verfgung
\usepackage{flafter}          % Floats immer erst nach der Referenz setzen
\usepackage{subfig} % Layout wird weiter unten festgelegt !
\usepackage{wrapfig}	        % Bilder von Text Umfliessen lassen

% Make float placement easier
\renewcommand{\floatpagefraction}{.75} % vorher: .5
\renewcommand{\textfraction}{.1}       % vorher: .2
\renewcommand{\topfraction}{.8}        % vorher: .7
\renewcommand{\bottomfraction}{.5}     % vorher: .3
\setcounter{topnumber}{3}              % vorher: 2
\setcounter{bottomnumber}{2}           % vorher: 1
\setcounter{totalnumber}{5}            % vorher: 3

% ~~~~~~~~~~~~~~~~~~~~~~~~~~~~~~~~~~~~~~~~~~~~~~~~~~~~~~~~~~~~~~~~~~~~~~~~
% science packages
% ~~~~~~~~~~~~~~~~~~~~~~~~~~~~~~~~~~~~~~~~~~~~~~~~~~~~~~~~~~~~~~~~~~~~~~~~

\usepackage{units}

% ~~~~~~~~~~~~~~~~~~~~~~~~~~~~~~~~~~~~~~~~~~~~~~~~~~~~~~~~~~~~~~~~~~~~~~~~
% layout packages
% ~~~~~~~~~~~~~~~~~~~~~~~~~~~~~~~~~~~~~~~~~~~~~~~~~~~~~~~~~~~~~~~~~~~~~~~~

%% Zeilenabstand =========================================================
%
%%% Doc: ftp://tug.ctan.org/pub/tex-archive/macros/latex/contrib/setspace/setspace.sty
\usepackage{setspace}
%\doublespace	        % 2-facher Abstand
%\onehalfspacing        % 1,5-facher Abstand
% hereafter load 'typearea' again

%% Seitenlayout ==========================================================
%
% Layout mit 'typearea'
\typearea[current]{last}
\raggedbottom     % Variable Seitenhoehen zulassen


%% Kopf und Fusszeilen====================================================
%%% Doc: ftp://tug.ctan.org/pub/tex-archive/macros/latex/contrib/koma-script/scrguide.pdf
\usepackage[%
   automark,         % automatische Aktualisierung der Kolumnentitel
   nouppercase,      % Grossbuchstaben verhindern
]{scrpage2}

\pagestyle{scrheadings} % Seite mit Headern
%\pagestyle{scrplain} % Seiten ohne Header
%\pagestyle{empty} % Seiten ohne Header

% loescht voreingestellte Stile
\clearscrheadfoot
%\clearscrheadings
\clearscrplain
%
%\ohead{\pagemark} % Oben aussen: Seitenzahlen
\cfoot{\pagemark} % unten mitte...
%\ihead{\headmark} % Oben innen: Kapitel und Section
\chead{\headmark} % Oben mitte: Kapitel und Section

% Angezeigte Abschnitte im Header
\automark[section]{chapter} %[rechts]{links}
%
\setheadsepline{.4pt}[\color{black}] % Linie zwischen Kopf und Textk�rper

%% Fussnoten =============================================================
% Keine hochgestellten Ziffern in der Fussnote (KOMA-Script-spezifisch):
\deffootnote{1.5em}{1em}{\makebox[1.5em][l]{\thefootnotemark}}
\addtolength{\skip\footins}{\baselineskip} % Abstand Text <-> Fussnote
\setlength{\dimen\footins}{10\baselineskip} % Beschraenkt den Platz von Fussnoten auf 10 Zeilen
\interfootnotelinepenalty=10000 % Verhindert das Fortsetzen von
                                % Fussnoten auf der gegenüberligenden Seite

%% Schriften (Sections )==================================================

% -- Koma Schriften --
\newcommand\SectionFontStyle{\sffamily}
\setkomafont{chapter}{\huge\SectionFontStyle}    % Chapter
\setkomafont{sectioning}{\SectionFontStyle} %  % Titelzeilen % \bfseries
%\setkomafont{pagenumber}{\bfseries\SectionFontStyle}             % Seitenzahl
\setkomafont{pagenumber}{\normalfont\SectionFontStyle}             % Seitenzahl nicht fett...
\setkomafont{pagehead}{\itshape\small\sffamily}        % Kopfzeile, beeinflusst aber auch fusszeile...
\setkomafont{descriptionlabel}{\itshape}        % Kopfzeile
%
\renewcommand*{\raggedsection}{\raggedright} % Titelzeile linksbuendig, haengend
%

%% Captions (Schrift, Aussehen) ==========================================

%%% Doc: ftp://tug.ctan.org/pub/tex-archive/macros/latex/contrib/caption/caption.pdf
\usepackage{caption}
% Aussehen der Captions
\captionsetup{
   margin = 10pt,
   font = {small,rm},
   labelfont = {small,bf},
   format = plain, % default oder 'hang'
   indention = 0em,  % Einruecken der Beschriftung
   labelsep = colon, %period, space, quad, newline
   justification = RaggedRight, % justified, centering
   singlelinecheck = true, % false (true=bei einer Zeile immer zentrieren)
   position = bottom %top
}
%%% Bugfix Workaround
\DeclareCaptionOption{parskip}[]{}
\DeclareCaptionOption{parindent}[]{}

%\DeclareGrphicsExtensions{.pdf,.png,.jpg}

% Aussehen der Captions fuer subfigures (subfig-Paket)
\captionsetup[subfloat]{%
   margin = 10pt,
   font = {small,rm},
   labelfont = {small,bf},
   format = plain, % default oder 'hang'
   indention = 0em,  % Einruecken der Beschriftung
   labelsep = space, %period, space, quad, newline
   justification = RaggedRight, % justified, centering
   singlelinecheck = true, % false (true=bei einer Zeile immer zentrieren)
   position = bottom, %top
   labelformat = parens % simple, empty % Wie die Bezeichnung gesetzt wird
 }

%% Inhaltsverzeichnis (Schrift, Aussehen) sowie weitere Verzeichnisse ====

\setcounter{secnumdepth}{2}    % Abbildungsnummerierung mit groesserer Tiefe
\setcounter{tocdepth}{2}		 % Inhaltsverzeichnis mit groesserer Tiefe
%

% Auszufuehrende Befehle  ------------------------------------------------

\listfiles
%------------------------------------------------------------------------


\newcolumntype{b}{>{\global\let\currentrowstyle\relax}}
\newcolumntype{n}{>{\currentrowstyle}}
\newcommand{\rowstyle}[1]{\gdef\currentrowstyle{#1}%
  #1\ignorespaces
}



% Silbentrennung
\input{preambel/Hyphenation}

%für quellcode
\usepackage{listings}
\lstset{
    language=[Sharp]C,
    basicstyle=\ttfamily\small,
    frame=single,       % einfacher rahmen um quellcode 
    %frameround=fttt,   % runder rahmen für rahmentyp frame
    %frame=trBL         % komplexer rahmen um quellcode
    %frame=lines,        % rahmen nur unten und oben
    %numbers=left,
    %numberstyle=tiny
    %backgroundcolor=\color{lightgray}
    %keywordstyle=\color{orange}\bfseries,
    %keywordstyle=\color{RoyalBlue}\bfseries,
    keywordstyle=\color{Black}\bfseries,
    commentstyle=\color{darkgray},
    stringstyle=\color{red}
}

%
% WORKAROUND, damit lstlistoflistings funktioniert:
% Quelle: http://www.komascript.de/node/477
%
\makeatletter% --> De-TeX-FAQ
\renewcommand*{\lstlistoflistings}{%
\begingroup
\if@twocolumn
\@restonecoltrue\onecolumn
\else
\@restonecolfalse
\fi
\lol@heading
\setlength{\parskip}{\z@}%
\setlength{\parindent}{\z@}%
\setlength{\parfillskip}{\z@ \@plus 1fil}%
\@starttoc{lol}%
\if@restonecol\twocolumn\fi
\endgroup
}
\makeatother% --> \makeatletter
%für tabellen
\usepackage{multirow}
%\newcolumntype{C}[1]{>{\centering}m{#1}}
%\usepackage{tabularx}


% für TODO-Notes
\usepackage{color}
\usepackage{tikz}

% für mathematische symbole
\usepackage{amsmath}
\usepackage{amssymb}
\usepackage{bm}

% für Einbinden von .pdf's
\usepackage{pdfpages}

% wegen deutschen Umlauten
\usepackage[utf8]{inputenc}

% für Seitenlayout
\pagestyle{useheadings}

\usepackage{amsthm}
% für Zeilennummern
\usepackage{lineno}
% Einschalten der ZN
%%%\linenumbers
% Setzen der ZN
%%%\modulolinenumbers[1]

\usepackage{graphicx}

% BEGIN TODO-Notes support
%{{{
\makeatletter \newcommand \listoftodos{\section*{Todo list} \@starttoc{tdo}}
\newcommand\l@todo[2]
    {\par\noindent \textit{#2}, \parbox{10cm}{#1}\par} \makeatother

\definecolor{orange}{rgb}{1,0.5,0}
\tikzstyle{notestyle} = [draw=black, fill=orange, text width = 2.5cm]
\tikzstyle{notestyleleft} = [notestyle]
\tikzstyle{connectstyle} = [draw = orange, thick]
% Command for inserting a todo item
\newcommand{\todo}[1]{%
% Add to todo list
\addcontentsline{tdo}{todo}{\protect{#1}}%
%
\begin{tikzpicture}[remember picture, baseline=-0.75ex]%
    \node [coordinate] (inText) {};
\end{tikzpicture}%
%
% Make the margin par
\marginpar[%
{% Draw note in left margin
    \tikz[remember picture] \draw node[notestyleleft] (inNote) {#1};%
    \begin{tikzpicture}[remember picture, overlay]%
        \draw[connectstyle]
            ([yshift=-0.2cm] inText)
                -| ([xshift=0.2cm] inNote.east)
                -| (inNote.east);
    \end{tikzpicture}%
}%
]{% Draw note in right margin
    \tikz[remember picture] \draw node[notestyle] (inNote) {#1};%
    \begin{tikzpicture}[remember picture, overlay]%
        \draw[connectstyle]
            ([yshift=-0.2cm] inText)
                -| ([xshift=-0.2cm] inNote.west)
                -| (inNote.west);
    \end{tikzpicture}%
}%
}%
%}}}
% END TODO-Notes support

\usepackage{acronym}
\usepackage{amsfonts}
\usepackage{amsmath}

\usepackage{graphics, epsfig, psfrag}

% listing definitions from cf

\newtheorem{definition}{Definition}
\newtheorem{proposition}{Proposition}
\newtheorem{lemma}{Lemma}
\newtheorem{theorem}{Theorem}

\begin{document}

\lstdefinelanguage{L}
  {morekeywords={Oracle,if,then,else,Finalize,return,Initialize,Encrypt,Decrypt,
  false,Hash,true,Tag,Verify,AuxVerify,Extract,and,or,AuxDecrypt,Append,for,do,
  Encryption,Decryption,GenerateCoins,in},
  sensitive=false,
  morecomment=[l]{//},
  morecomment=[s]{/*}{*/},
  morecomment=[s]{(*}{*)},
}
\lstset{mathescape=true,language=L,basicstyle=\small,frame=none}
\lstset{numbers=left, numberstyle=\tiny, stepnumber=1, numbersep=5pt}
\lstset{emph={GenerateCoins,Oracle,Finalize,Initialize,Encrypt,Hash,Decrypt,Verify,Tag},emphstyle={\bfseries\underbar}}



\begin{titlepage}
\large
\noindent
Bauhaus-Universität Weimar\\
Faculty of Media\\
Degree Program Computer Science and Media\\
\author{Michael Pannier}
\title{Can't touch this}
\vspace{20mm}
\begin{center}
    \huge{\bfseries{Can't Touch This -\\
		A Prototype for Public Pointing Interaction}}
\end{center}
\vspace{15mm}
\begin{center}
    \huge{\bfseries{Master Thesis}}\\
\end{center}
\vspace{20mm}
Michael Frank Pannier
\hfill Registration Number 51755\\
born 19th December 1984 in Dessau\\
\newline
\newline
1st Supervisor: Prof. Dr. Eva Hornecker\\
2nd Supervisor: Prof. Dr. Sven Bertel\\
\vfil
\noindent
Date of Submission: 9th January 2015\\
\end{titlepage}

\frontmatter
%
\chapter*{}
\section*{Acknowledgement}

Great thanks goes to my parents Annemarie and Joachim and my brothers Justus and Jan, who have supported me during the past two years, morally and financially.
Furthermore, I would like to thank the owner of the Chair of Media Security professor Stefan Lucks, who has given me the opportunity to work in the
ever-growing reasearch area of cryptography. I owe my gratitude to my advisor Christian Forler, who, with his constructive
criticism and helpful remarks, helped guide this thesis to its proper destination. I also thank Christof Bräutigam, Ewan Fleischmann, Lars Harmsen,
Alexander Kümmel, Eik List, Thomas Knapke, Michael Pannier und Michael Völske, for their interminable support during the time
I spent working on this thesis, as well as Benno Stein for co-supervising it.


\vfill

\section*{Danksagung}

Mein größter Dank geht an meine Eltern Annemarie und Joachim und meine Brüder Justus und Jan, die mich während der letzten zwei Jahre moralisch und finanziell unterstützt haben.
Danken möchte ich dem Inhaber des Lehrstuhls für Mediensicherheit Professor Stefan Lucks, der mir die Möglichkeit gegeben hat, in dem
stetig wachsenden Forschungsgebiet der Kryptographie zu arbeiten. Meinem Betreuer Christian Forler gehört Dank, denn mit seiner konstruktiven
Kritik und seinen guten Anmerkungen hat er diese Arbeit zu einem guten Ziel geführt.
Weiter danke ich Christof Bräutigam, Ewan Fleischmann, Lars Harmsen, Alexander Kümmel, Eik List, Thomas Knapke, Michael Pannier und Michael Völske, die mir
während meiner Bearbeitungszeit stets mit Rat und Tat zur Seite standen sowie meinem Zweitbetreuer Professor Benno Stein.



\tableofcontents
\listoffigures
\listoftables
%keine seitenzahl auf dem deckblatt
%\pagestyle{empty}

\chapter*{Abbrevations}
\begin{acronym}[\hspace{2cm}]
  %\acro{Abkuerzung}{ausgeschrieben}
	\acro{IMI}{Interactive Museum Installation}
	\acro{MS}{Microsoft}
	\acro{RFID}{Radio-Frequency Identification}
	\acro{FSD}{Functional Specification Document}
	\acro{MIT}{Massachusetts Institute of Technology}
	\acro{SDMS}{Spacial Data-Management System}
	\acro{WYSIWYG}{''What you see is what you get''}
	\acro{GUI}{Graphical User Interface}
	\acro{SUI}{Single-User Interface}
	\acro{MUI}{Multi-User Interface}		
	\acro{HCI}{Human Computer-Interaction}
	\acro{TUI}{Tangible User Interface}
	\acro{VR}{Virtual Reality}
	\acro{3D}{three-dimensional}
	\acro{HMD}{head-mounted display}
	\acro{DOF}{degrees of freedom}
	\acro{AR}{Augmented Reality}
	\acro{SDK}{Software Development Kit}
	\acro{2D}{two-dimensional}
	\acro{BCI}{Brain-Computer Interface}
	\acro{MVT}{Museumsverband Thüringen}
	\acro{HDD}{Hard Disk Drive}
	\acro{PDLC}{Polymer Dispersed Liquid Crystal}
	\acro{IR}{infra-red}
	\acro{FUBI}{Full Body Interaction Framework}
\end{acronym}


\mainmatter

\chapter{Abstract}
\label{abstract}

\paragraph{Annotations}

\begin{itemize}
	\item Exciting summary
	\item Create interest 
\end{itemize}
\chapter{Introduction}
\label{introduction}

%\textbf{Worum geht's hier überhaupt??? Und eine Masterarbeit fängt man nicht damit an 'Mein Auftrag war' sondern man erklärt das Ziel dahinter, warum ist das interessant, was ist das Forschungsinteresse...}

In this work, I describe the development of an interactive museum installation (IMI). The system presents a novel way to augment public displays with a system that is intuitive to use, easy to maintain, and inexpensive. Natural interaction without additionally required devices on the users end lowers inhibition and frustration. Simultaneously, awareness for the displayed contents is raised.

\begin{figure}[H]%
\includegraphics[width=\columnwidth]{../pics/entry.eps}%
\caption{Interactive Museum Installation inside the Haßleben-showcase at the Museum für Ur- und Frühgeschichte Thüringens.}%
\label{fig:entry} 
\end{figure}

Before the system depicted in Figure \ref{fig:entry} could be developed, a collaboration with a local museum had to be established. Because the installation would be based on in- and output modalities that actually make sense in a museum, visitors and staff had to be observed and interviewed. After looking at several suitable museum candidates in Weimar. I chose the one with the most promise in fitting properties as well as institutional openness for my purposes. Together, we conceived some ideas for possible installations. Not all of them were applicable an some were too far off my expertise. Nevertheless, there were two concepts for augmenting the \textit{gravesite of Haßleben-showcase}, that we were very interested in and excited about.

The first concept, \textit{''Interaction with Tangibles''}, directly addressed visitors' haptic perception. Therefore, it was planned to use \ac{MS} Gadgeteer-hardware\footnote{\ac{MS} Gadgeteer is a modular system of various hardware-components distrubuted by GHI Electronics. It resembles Arduino- and other microcontrollers.} as embedded components of tangible devices. A number of reproductions could be placed outside the showcase. Each interactive tangible could then be manipulated or placed on a pedestal to gain information about its corresponding exhibit. Here, certain exhibits could have been photogrammetically scanned in three dimensions. After that, the digital model could be scaled to a handy size and otherwisely modified. Ultimately, the tangible could be printed or casted. Such an object could then be enhanced by using \ac{RFID}-technology\footnote{RFID-transponders or -tags do not require any batteries, are cheap and robust. In addition, their range is very limited, which allows several tags on one tangible.}. In order to make it interactive, it would be fitted with such a \ac{RFID}-tag. There is a \ac{RFID}-module for Gadgeteer, which would have allowed identification of each tangible. The corresponding information could then be provided by any medium compatible with Gadgeteer.
\\
A different approach was based on an assumption of natural behavior of visitors. After a meeting at the museum, a second concept of \textit{''Interaction by Pointing''} emerged. Later the underlying assumption was confirmed by the observation of visitors' behavior around showcases. Visitors do not only talk about exhibits, but they also point at certain exhibits during interaction with each other. Therefore, a device should be build or utilized for users to point with, enabling them to select a certain exhibit inside the showcase. Additional information about the point of interest would then be displayed in an appropriate manner. 
\\
While all involved understood these concepts were fairly comprehensible, their technical realizations were unclear at first. Throughout further investigations, the work turned from testing various modes of input to a more technical approach. Both ways of input revealed different challenges along the way.

%\textbf{Add a paragraph describing the final system with a sketch - this will make things much clearer for the reader!}

%\begin{center}
	%\textit{''Developing an innovative museum installation for and with a local museum.''}
%\end{center}
%Previous specifications
%\begin{itemize}
	%\item interactive museum installation
	%\item Gadgeteer (embedded hardware)
	%\item observations and interviews
	%\item input and output modalities that make sense in a museum
	%\item prototype
	%\item interactive design process
%\end{itemize}

%-----------------------------------------------------------------------------

Throughout the following chapters I document my proceedings during the development of the aforementioned system. Chapter \ref{motivation} gives background information about the fields of study which are included in my work. Thus, there is a brief outline about the progression of technologies employed by museums, behavior of users around public interfaces and with tangibles. In addition, a brief overview of virtual reality-techniques is given.
\\
Afterwards, I present my goals for the development of this system. Before I come to explain the technical principles and evaluation of my implementations, I give a short review of my partnering process. Thus, Chapter \ref{partnering} deals with finding a suitable museum for a collaboration. 
\\
Chapter \ref{conception} explains the whole development-process of the system's functionality. It begins with possible system designs and explains their possibilities and constraints. In the end of Chapter \ref{conception}, the final concept is shown along with necessary obligations such as an \ac{FSD} and the contract between me, the university and the museum.
\\
Chapter \ref{implementation} addresses the implementation of the system's functionality. Therefore, all libraries and softwares are explained in more detail.
\\
Experimental lab-installations and the final museum-installation are described in Chapter \ref{installation}. Therefore, measurements, hardware specifications, and other influential criteria are presented in detail.
\\
The final installation is evaluated in Chapter \ref{evaluation}, where visitors were observed and interviewed before and after alterations by the system. Chapter \ref{discussion} then deals with the discussion of the evaluation's findings.
\\
In the end, I discuss potential future work, which could improve, extend, and follow my system. In Chapter \ref{future_work}, I would also like to mention reactions and suggestions I encountered along my work.

%\paragraph{Introduction - Annotations}
%
%\begin{itemize}
	%\item Short overview, about what has been build
	%\item Summary
	%\\
	%\item System of libraries for pointing interaction
	%\item Information system (Information On Demand)
	%\item 'Uncharted territory' $\to$ technical focus
	%\item Template solution / 'just a proof of concept'
  %\\
	%\item Motivation
	%\item Working within the confines of museums respectively public installations
%\end{itemize}
\chapter{Partnering process}
\label{partnering}

The very first step after having the idea of introducing a new way for information to be retrieved in public places was to find a partner to realize it with. In order to find the most promising and suitable cooperation, appropriate properties would have to be defined and considered for each institution before partnering with any of them. Afterward, a suitable exhibit and an agreement on a design for the installation would be found.

%------------------------------------------------------------------------------
\section{Requirement analysis}
\label{partnering_requirement}

To determine an ideal partner for a cooperation, a mutual beneficial system of needs and demands had to be established. Therefore, each party's needs and offerings were identified. As Table \ref{tab:museums_needs_demand} shows, three major criteria were determined. Possible cooperations would be based on those criteria. In addition, special characteristics would be considered as well.

\begin{table}[h]
	\centering
	\begin{tabular}{ bl !{\vrule width 1pt} nl nl}
		\rowstyle{\bfseries}
												& Museum 										& Me \\
		\toprule
		\textbf{Needs} 			& Improvement / Innovation	& Access to a public space \\
												&														&	with exhibits and visitors \\		
												& New group of visitors			& Authentic content \\
												& Publicity	/ Awareness			& Potential test subjects \\
		\hline
		\textbf{Offerings}	& A public space						& Technological expertise \\ 
												& Factual expertise		 			& Development and testing \\ 
												& Resources									& Motivation \\ 
	\end{tabular}
	\caption{Needs and Demand.}
	\label{tab:museums_needs_demand}
\end{table}

Museums want to get people interested in their respective topics. Thus, reaching more people and raising awareness is one of their main interests. A good way to attract new groups of visitors is to offer something unique and innovative. Although there are companies offering services like guide- or information-systems, they are either cosmetic, expensive or high-maintenance. On the other hand, a museum has valuable offerings. Usually, they have a budget for renovation and improvements. The staff is highly skilled and experienced concerning the exhibits and visitors' behavior around them. Finally, a museum offers a public space, where a system can be tested under natural conditions.
\\ 
The Bauhaus-Universität and specifically the chair for \ac{HCI} as well as myself wanted the final system to work in a real-life environment, but not as a lab-study alone. Hence, we needed access to a public place in order to reach a broad variety of people. Those would be unbiased toward the nature of interaction and content as well. Meanwhile, we could provide our knowledge of interaction design and the suitability of contemplable technologies. And lastly, I was highly motivated to develop a working system.
\\
After finding a cooperation partner, a \ac{FSD} would be made, which includes the system's properties ordered by necessity. In addition, a contract between all parties would be drawn up to register each party's contributions and obligations.

%\paragraph{Annotations}
%
%\begin{itemize}
	%\item 'What do we have to offer?'
	%\begin{itemize}
		%\item Expertise
		%\item Time
		%\item Motivation
	%\end{itemize}
	%\item 'What do we need?'
	%\begin{itemize}
		%\item A museum
		%\item Access
		%\item Public (for evaluation)
	%\end{itemize}
	%\item 'What should the museum be offering?'
	%\begin{itemize}
		%\item Location ('A museum')
		%\item Staff's expertise
		%\item Hardware
		%\item Access (for evaluation)
	%\end{itemize}
	%\item 'What does the museum want?' \textit{better: need}
	%\begin{itemize}
		%\item A working Improvement of their exhibition
	%\end{itemize}
	%\item Pflichtenheft
	%\item Contract
	%\item Further Cooperation
%\end{itemize}

%-----------------------------------------------------------------------------

\section{Potential partner museums}
\label{partnering_investigation}

According to \ac{MVT}~\cite{ThueringerMuseumsverbandOrte} there are more than 50  museums in Weimar within a few kilometers distance from the town. Table \ref{tab:museums_amounts} only shows museums registered at the \ac{MVT} and the three towns with the most of them. Other towns have between one and six registered museums. Further, it is most likely that there are more museums than those in this list. It provides a good starting point, though.

\begin{table}[h]
	\centering
	\begin{tabular}{ bl !{\vrule width 1pt} nr }
		\rowstyle{\bfseries}
		Town		& Museums \\
		\toprule
		Weimar	& 26 \\ 
		Erfurt 	& 12 \\ 
		Jena 		& 12 \\ 
	\end{tabular}
	\caption{Museums in and around Weimar.}
	\label{tab:museums_amounts}
\end{table}

Regarding the high amount of museums in Weimar alone, it seemed promising to start looking for a suitable cooperation partner right here. Since 26 museums are too many to investigate thoroughly, a preselection had to be made. In the first step, the focus was on flexibility. This meant, only a small administrative apparatus could guarantee fast decisions and less organizational meetings with boards and other decision makers. Hence, all the \textit{Klassikstiftung}'s museums were crossed of the list, narrowing it down to only 10 remaining candidates. Next, and after some further research, museums with less interesting topics or inconvenient concepts were withdrawn. This included the tiny \textit{umbrella museum} and \textit{Weimar Haus}, a place glutted with animatronics. Afterward, the list of candidates was down to five (see Table \ref{tab:museums_finalists}). A personal visit to each of these museums was indispensable now.

\begin{table}[h]
	\centering
	\begin{tabular}{ bl }
		\rowstyle{\bfseries}
		Museum \\
		\toprule
		Deutsches Bienenmuseum \\
		Kirms-Krakow-Haus \\
		Museum für Ur- und Frühgeschichte Thüringens \\ 
		Palais Schardt \\
		Pavillon Presse \\
	\end{tabular}
	\caption{Remaining cooperation candidates.}
	\label{tab:museums_finalists}
\end{table}

Gathering impressions in person was a process of three stages. In the first stage, I would visit a museum and noted its technical and pedagogical equipment. This was directly followed by the next stage, an informal introduction to some of the staff  containing a chat about my plans and the respective person's attitude towards them. The final stage was a formal introduction-meeting between my professor, me and the administrative staff of each museum, that had expressed serious interest. This serious interest wasn't shown by the Kirms-Krakow-Haus and the Pavillon Presse. Hence, the aforementioned meeting only took place at the Deutsche Bienenmuseum, Museum für Ur- und Frühgeschichte Thüringens and Palais Schardt. We introduced ourselves at each venue, because a discussion about what might be done was more efficient directly on site.  

%\paragraph{Annotations}
%
%\begin{itemize}
	%\item Project process: Partnering
	%\\
	%\item Preselection of possible partners
	%\item Criteria
	%\begin{itemize}
		%\item Proximity
		%\begin{itemize}
			%\item Thüringer Museumsverband~\cite{ThueringerMuseumsverbandOrte} (Weimar, Jena, Erfurt, Apolda have 50 museums)
			%\item Weimar alone has 26, and probably more than that
		%\end{itemize}
		%\item Flexibility
		%\begin{itemize}
			%\item Little administrative apparatus $\to$ no Klassikstiftung
			%\item Making (faster) decisions, due to less administration (Gremien)
			%\item Direct connection to chairs
			%\item Willingness for cooperation
		%\end{itemize}
		%\item Open-mindedness
		%\begin{itemize}
			%\item Many different and realizable topics
			%\item Excitable for and capable of new Ideas
			%\item Willingness for change
		%\end{itemize}
		%\item Attractiveness
		%\begin{itemize}
			%\item Topic(s)
			%\item Realizability of the interaction-idea(s)
			%\item Guts
		%\end{itemize}
	%\end{itemize}
	%\item 'Supply and demand'
	%\begin{itemize}
		%\item Necessities to realize the idea
		%\item Who provides what $\to$
	%\end{itemize}
%\end{itemize}

%-----------------------------------------------------------------------------

%\paragraph{Annotations}
%
%\begin{itemize}
	%\item Visit preselected museums
	%\begin{itemize}
		%\item Taking notes
		%\item Taking photos
	%\end{itemize}
	%\item Getting an Overview $\to$ (Im)Possibilities
	%\begin{itemize}
		%\item Some Criteria for realizability
		%\begin{itemize}
			%\item Atmosphere (outdated vs. innovative tendencies)
			%\item Space for an installation
			%\item Number of other visitors
		%\end{itemize}
	%\end{itemize}
	%\item Establish a first contact
	%\begin{itemize}
		%\item Talk to staff
		%\item Make an appointment with executives (board)
	%\end{itemize}
%\end{itemize}

%-----------------------------------------------------------------------------

\section{Decision for a partner museum}
\label{partnering_decision}

A formal introduction-meeting went as follows: First, I explained some of my previous projects, related installations in other museums and the general intent of the professor's chair. Next, the staff explained their museum's concept and which subject area they would like to emphasize. After that, we discussed potential concepts. Those ranged from augmentations of existing exhibits to completely new installations.

\paragraph{Deutsches Bienenmuseum} 

The museum is run by the beekeepers association of Thuringia. The staff we encountered was very skilled with the craft of beekeeping, but less professional concerning museum education and design. They listened to my remarks and we had an inspiring discussion about potential topics and their feasibility. Unfortunately, the association's chairman and we could not agree on a specific project. Also, because bees hibernate, visitor attendances are seasonal and also fluctuant. Hence, the Deuschte Bienenmuseum was out of the picture.

\textbf{F I G U R E}
%\begin{figure}%
%\includegraphics[width=\columnwidth]{filename}%
%\caption{.}%
%\label{fig:bienenmuseum} %
%\end{figure}

\paragraph{Palais Schardt} 

This venue is owned by a family, which exhibits multiple collections of art and crafts as well as the building itself. In addition, they operate a cafe and use the adjacent hall for events. The husband is a restorer by trade and gives talks about the building and its historic significance, while his wife handles planning and the cafe. Events are regular and the cafe supplies casual customers and visitors. Both were very interested in a cooperation and had some ideas for installations. But the monument protection of the building and minor financial issues complicated feasibility. Therefore, Palais Schardt also had to go. 

\textbf{F I G U R E}
%\begin{figure}%
%\includegraphics[width=\columnwidth]{filename}%
%\caption{.}%
%\label{fig:schardt_pavillon} %
%\end{figure}

\paragraph{Museum für Ur- und Frühgeschichte Thüringens} 

Since the state office for preservation of historical monuments and archeology of Thuringia is the bearer of the museum, all personnel is very competent at their field of work. In addition, the museum employs special staff, that maintains the exhibition, gives tours and is present for arising topical questions during opening hours. Classes of 5th and 6th grade visit regularly for field trips as well as visitors from all age groups. The exhibition was already altered by several media installations. Moreover, the director was very enthusiastic from the first meeting on and had several ideas, of which exhibits to emphasize.

\textbf{F I G U R E}
%\begin{figure}%
%\includegraphics[width=\columnwidth]{filename}%
%\caption{.}%
%\label{fig:muft} %
%\end{figure}

Summarizing, the Deutsche Bienenmuseum and Palais Schardt were deemed less interesting and lacking feasibility. The Museum für Ur- und Frühgeschichte Thüringens was chosen as the cooperation partner, because it checked the most boxes of the previous Requirement Analysis (see Chapter \ref{partnering_requirement}), while the others lacked at least once in the \textit{Needs}- or \textit{Offerings}-category. It was the most professional and ambitious candidate with promising resources and conditions.

%\paragraph{Annotations}
%
%\begin{itemize}
	%\item Offical introduction at the museum
	%\begin{itemize}
		%\item Personal
		%\begin{itemize}
			%\item Projects
			%\begin{itemize}
				%\item Perceiving AR (Psychophysiologie und Wahrnehmung - Huckauf)
				%\item pEYEwrite (Psychophysiologie und Wahrnehmung - Huckauf)
				%\item Schlender (Usability - Bertel)
				%\item Neural Control (Vernetzte Systeme - Schatter)
				%\item KickFlickable Interfaces (HCI - Hornecker)
			%\end{itemize}
			%\item Bachelor Thesis (VR - Fröhlich)
			%\item Skills
			%\begin{itemize}
				%\item Interdisciplinary work experience (Schlender, KickFlickable Interfaces)
				%\item Multiple programming Languages and their (dis-)advantages for a project (Schlender) $\to$ Chapter \ref{conception} Conception
				%\item Interface Design 
				%\item User Experience
				%\item Rapid Prototyping
			%\end{itemize}
		%\end{itemize}
		%\item Present requirements see \ref{museums_requirement}
	%\end{itemize}
	%\item Brainstorming
	%\begin{itemize}
		%\item Museum-staff: 'Emphases'
		%\item Me: 'Possible solutions'
	%\end{itemize}
%\end{itemize}

%-----------------------------------------------------------------------------

%\subsection*{Old version}
%
%In order to finding a museum to cooperate with several steps had to be made. They included getting an overview of all museums in Weimar, finding several candidates for that cooperation, scouting those candidates and getting in contact with the most promising of them, and, finally, discussing possible concepts within their exhibitions.
%\\
%The first step was to find out about all the museums in Weimar and close by. So I looked them up on the website of Museumsverband Thüringen~\cite{ThueringerMuseumsverbandOrte} see Table \ref{tab:museums_amounts}, where there is a list of all members with links to further information. This list included museums in Weimar, Jena and Erfurt. Some were run privately, others by a foundation or a club, and a few by a public owner. Since there was a total of fifty museums and half of them in Weimar alone, there had to be a preselection.
%
%\begin{table}[]
	%\centering
	%\begin{tabular}{ bp{5em} nr }
		%\rowstyle{\bfseries}
		%Location & Amount \\
		%\toprule
		%Weimar & 26 \\ 
		%Erfurt & 12 \\ 
		%Jena & 12 \\ 
		%Apolda & 1 \\ 
		%\bottomrule
		%Total & 51 \\ 
	%\end{tabular}
	%\caption{Museums in and around Weimar.}
	%\label{tab:museums_amounts}
%\end{table}
%
%Hence, as the following step, only museums in Weimar were chosen. In addition, the museums run by Klassikstiftung were taken out of consideration, for the foundation seemingly being too big and too inflexible concerning innovation in their historic premises. Some very small museums were struck off the list as well. This left four candidates remaining. They were Pavillon Presse, Pallais Schardt, Bienenmuseum and Museum für Ur- und Frühgeschichte Thüringens.
%\\
%The next step was to get some first hand experience of each of the aforementioned museums. So, I went to visit all of them. During the visit I took notes and pictures of the exhibitions. Afterward, I talked to some staff members, explained what I was about to do, and arranged an appointment for an official introduction later on.
%\\
%The first visit was to the Bienenmuseum. It is run by a club of beekeepers and displays exhibits of beekeeping throughout the ages and several cultures. The exhibition is mainly conventional with vitrines and open exhibits. Moreover, they offer workshops, in which attendees learn more about bees in general, 'making' honey and even dipping our pouring candles.
%\\
%Pallais Schardt was the second visit. It is the historical home of the Schardt family, a very influential family at the court of Sachsen-Weimar. This place is owned by the Brinkmann family and run aside a cafe with traditional pastries from that particular era. Mr. Brinkmann is giving tours around the premisses and explains the building's significance in close contact to historical events. In addition, the saloon and other rooms can be rented for festivities.
%\\
%Right next to Pallais Schardt is Pavillon Presse. It used to be a printery and now accomodates printing presses and equipment from all ages. The museum is privately run by a foundation and volunteers. This museum was struck of the list immediately after the visit, for being to capricious to work with.
%\\
%The final visit was to the Museum für Ur- und Frühgeschichte Thüringens. There, artifacts from fossils, which a millions of years old, to medieval times are exihibited. The museum was overhauled in 1999 and thus, has a modern touch already. It is owned and run by the Thüringisches Landesamt für Denkmalpflege und Archäologie.
%
%\begin{figure}[H] % [H] -> Here!
	%\centering
	%\includegraphics[width=1.0\textwidth]{../pics/Original.png}
	%\caption{Original display of the Haßleben grave.}
	%\label{fig:museums_original}
%\end{figure}
%
%After those field trips, I fashioned a presentation, in which I would introduce myself and previous projects I participated in. Later on in the meeting, I would show pictures of the museums in Limmerick and Vienna and explained the work, which had been done there. Finally, I prepared a short presentation of the Microsoft Gadgeteer-system and some of its capabilities. Following my presentation, the attending museum-staff, my professor end I discussed possible deployment scenarios. During the brainstorming the museum-officials named exhibits, which could or rather should receive more attention, whilst me and my professor suggested fitting solutions or explained further technological possibilities.
%\\
%At Pallais Schardt, the owners were very interested in technology, but they could not imagine how and where to make use of it. The best thing we could come up with was a guided tour. Thus, I was invited to one of their soirees with classical music and a tour of the house, in order to making up my own mind. Although it was very interesting, nothing ground-breaking arose.
%\\
%At Museum für Ur- und Frühgeschichte, the director was very fascinated by the demo and immediately came up with several exhibits, which seemed fitting to him. Yet, his optimism had to be reined a little. Some of the tasks he had in mind were unfortunately not realizable with the tools I had in hand.
%\\
%At Bienenmuseum, there were two main topics. First, social interaction of bees. For instance, bees dance to communicate the direction of plenty resources. Second, bees' perception of their environment. Bees see in another spectrum than we do and they can smell a lot better than us. In the end of our meeting, we were discussing about a virtual bee hive. This installation would be able
%to simulate the behavior of a bee colony according to some certain inputs, which could be made by visitors.
%\\
%The final decisions were made after working out several key criteria for the best possible cooperation. Those were common criteria every museum could or could not meet and special criteria, which could also tip the scales. Three common criteria were identified. First of all, the amount and age of visitors was very important. Since the prototype had to be evaluated, a sufficient number of potential test subjects with a certain grade of affinity for technology would be needed. Second and not much less important, was the size and quality of the staff. If there was no expert of the museum's subject, who was able to work together with me, the project would be a fail. The third criteria was plainly budget. At some point, additional electronics and/or other equipment would be necessary. The special criteria more or less had an influence on the aforementioned main criteria. For example, monument protection, seasons, and motivation were some of them. First, Bienenmuseum had to go, because the club's chairman was not very fond of our discussions. Furthermore, the staff was not particularly professional and seemed to run the museum more as a hobby. The fact, that the museum has a large variety of visitors was a big plus, which was neutralized with the other fact, that bees are seasonal, and so are the according numbers of visitors. This makes an evaluation rather difficult, for not providing a constant number of test subjects.
%\\
%Finally, Pallais Schardt was struck from the list. Although, its owner was a restorer by trade, very approachable, and there were lots of events at the cafe, it had some corresponding cons as well. The building ant its historic role was very interesting, yet is a landmark. Thus, it must not be altered in any form, which might prove hard later on. The many people visiting the cafe are mostly 50 years an older. Hence, their abilities to understand and use technology as intended could be too much a risk during evaluation. Sadly, it is just a cafe and not a museum.
%\\
%The last item on the list is the Museum für Ur- und Frühgeschichte Thüringens. The major con was the planned exhibition, which leaves not much space for alterations. But, it is controlled by regional authorities. Hence, there is a budget for innovation projects. Moreover, the staff at the museum is interested in innovation and highly qualified in their field of expertise.

\chapter{Conception}
\label{conception}

After the \textit{Museum für Ur- und Frühgeschichte Thürigens} was chosen as a partner, all previous ideas had to be analyzed more thoroughly with feasibility in mind. Thus, impractical, and too complex or too simple ideas were eliminated in two rounds of review. At first, vague ideas were either improved or discarded. Hence, a screen displaying only information about a fossilized fireplace was eliminated. The idea of a system for digitizing stone carvings was considered too complex to realize and therefore discarded as well. Afterwards, some of the museum's staff and I looked at the contents, that could be provided for the remaining candidates. This left us with two remaining possibilities, that were promising enough from an educational as well as a technical standpoint. The first one was the reproduction of the \textit{Fürstengrab von Haßleben}, which contains replicas and original artifacts from a 1700 year old grave of a Teutonic princess. A close second was a workshop, which should have shown how archeologists and restorers work behind the scenes of a museum. Here, the latter consisted of too many single parts and a lot of questions remained unanswered.

%\begin{figure}[H] % [H] -> Here!
	%\centering
	%\includegraphics[scale = 0.7]{../pics/Original.eps}
	%\caption{Fürstengrab von Haßleben-showcase prior to installation.}
	%\label{fig:conception_grave}
%\end{figure}

According to the aforementioned review, the Fürstengrab von Haßleben was most promising and therefore chosen in the end. It contains many special relics from ordinary, Teutonic pottery to rare, Roman coins and jewelry. There are original artifacts and replicas on display inside the showcase, which I am collectively referring to as \textit{exhibits} throughout this work. Some of these exhibits are shown in Figure \ref{fig:conception_grave}a, b and c. The apparent eclecticism is, what makes the grave so special though. It is a sublime showcase for thriving trade and cultural exchange between Teutons and Romans as far east as Thuringia. Further, it proves how Teutons began adapting roman traditions, such as burials. In order to emphasize this insight, an interactive system was to be developed. Unfortunately, the showcase is located on the second floor. Thus, it does not get the attention it deserves. People are often tired after having visited the first floor. Hence, the museum's staff asked for an installation that would reactivate the visitors' attention.

%---------------------------------------------------------------------------- 

\section{System Preconditions}
\label{conception_system}

The system was to be developed and tested by me, and the museum-staff is responsible for its future maintenance. The full range of visitors' backgrounds cannot be foreseen. Some visitors might not have the proper technical experiences to operate contemporary interfaces. Consequently, it was crucial to design the system with that in mind. It had to be operable by absolute lay persons, who have no prior experience concerning information technologies. Hence, the interface had to be as intuitive and natural as possible. Four major points had to be considered.
\\
First, established and abstract input devices, such as keyboard and mouse, had to be replaced by something more natural. In order to be intuitive, the interaction was designed to capture and use the natural behavior of visitors. Outputs, on the other hand, had to be as discreet and as conservative as possible to not disturb or interfere with the exhibition. Thus, invasive technologies such as speakers and animatronics were excluded by the museum from the beginning. This consideration only left visual and haptic channels for output. The third point was, that daily operations at the museum were not to be compromised. So, it was not possible to develop the prototype inside the Haßleben-showcase itself and a full-size mockup had to be build somewhere else. Furthermore, the showcase and its precious exhibits had to be protected from any possible decay and nothing was to be rearranged. Thus, I measured the showcase and acquired a room in which a mockup could be placed for the prototype's implementation and testing\footnote{For a further description of the lab-setup see Chapter \ref{setup_development}}. Finally, the system's components, in- and output devices, had to be robust enough to cope with daily use. Moreover, they should also stay in their intended place. This meant that they had to be somehow attaching to the showcase.

In summary, the requirements for the final system were narrowing down the possibilities right from the beginning. Hence, we came up with several ideas and followed up on all of them, until one promised to be the most feasible.

%\paragraph{Annotations}
%
%\begin{itemize}
	%\item User perspective
	%\begin{itemize}
		%\item Visitor
		%\item Curator / staff
	%\end{itemize}
	%\item System view
	%\\
	%\item Development of ideas according to the plan
	%\begin{itemize}
		%\item Method of elimination
		%\item Feasibility
		%\begin{itemize}
			%\item Effort
			%\item Cost
		%\end{itemize}
	%\end{itemize}
%\end{itemize}

%-----------------------------------------------------------------------------

%\paragraph{Annotations}
%
%\begin{itemize}
	%\item Possibilities of hard- and software
	%\item Capabilities of a single programmer (me)
%\end{itemize}

%-----------------------------------------------------------------------------

\section{Concept Development}
\label{conception_constraints}

Developing the system, we followed two initial approaches. They were supposed to lead us to an intuitive, easy to use interface, which would be very naturally operable. The first concept featured the development of tangibles. Interactive objects would be placed outside the showcase and visitors would be able to interact with them. Haptic feedback would enable visitors to experience the exhibits in an unusual way. By touching replicas of otherwise locked up exhibits a deeper involvement is highly likely. Meanwhile, the other concept was based on gestural interaction. With this concept, visitors are enabled to interact with the showcase by pointing. This approach was based on the natural behavior of visitors. Like the previous approach an uncommon experience should raise visitors' involvement and attention.

\paragraph{Tangibles} 

The early idea behind this work was to work with \ac{MS} Gadgeteer to develop a tangible interface for and with a museum. Thus, we first thought about how to include those Gadgeteer-modules. Therefore, I built the demo device shown in Figure \ref{?}, which was based on a \textit{FEZ Spider Starter Kit}~\cite{SpiderKitGHI}. In addition, it utilized an \ac{RFID}-reader~\cite{RFIDreaderGHI} and a potentiometer~\cite{PotentiometerGHI}. The RFID-transponders were attached to an old 2,5" \ac{HDD} and a wireless mouse. When the \ac{RFID}-tags were recognized, an image of the object was displayed on the screen. By turning the potentiometer's knob the angle of view changed accordingly. This gave an impression of the possibilities of the hardware. Unfortunately, we only had two \ac{RFID}-tags that had the size of a credit card. After some research though, I found some tags for the correct frequency band and in sizes from a grain of rice over credit cards to key chains~\cite{RFIDtransponder}. Hence, including \ac{RFID}-tags in tangibles was feasible. 
\\
The shape and size of the tangibles were still up for debate. Another point was, whether the hardware would be placed inside or outside the tangibles. This decision dictates the shape and size of the tangibles and therefore the interaction. If it would be placed inside, the tangibles would have to be big. They would have turned out at approximately the size of a box of milk. Such an \textit{active tangible} would be handy and a whole system could be concentrated in one device. On the other hand, they would be prone to damage and maybe even theft. Hence, the tangibles would have to be tough and in some way attached to the showcase. In addition, batteries would have to be either charged or changed. This would take a certain amount of maintenance.
\\
With the hardware outside the tangibles and hidden in a pedestal in front of the showcase, the tangibles could be smaller. Moreover, \textit{passive tangibles} grant more flexibility concerning the shape as well. As described earlier (see Chapter \ref{motivation_interfaces} in ~\cite{TangibleUI}), the tangibles could have different features depending on certain properties. In this case, several \ac{RFID}-tags could be placed in each tangible. Depending to their \textit{constrained} collocation on the \ac{RFID}-reader, different reactions of the system could be triggered. In contrast to Ullmer at al., 2005, though, this affordance would be hidden and thus less obvious. The tangibles would have to be attached to the pedestal as well, although they would be less expensive to replace.

Both approaches had their advantages and disadvantages and none of them was concrete enough to make a decision. Thus, we continued to specify the concepts depending on their strengths and weaknesses. We did this, by anticipating probable relations between the exhibits inside showcase and the behavior of visitors behind the glass. There are several things visitors tend to do, if they are interested in an exhibit. They would like to inspect it up close. First, this means they would like to touch an exhibit and feel it. Second, they want to see it in more detail and from different angles. Next and induced by restrictions, visitors talk about an exhibit or  request further information. This could be anything from its age to where and how it was found.
\\
An active tangible could provide nearly all of those qualities in one package. It could - like the demo device - be fitted with a display and an \ac{RFID}-reader. The corresponding \ac{RFID}-tags could then be placed close to the device in order to trigger a particular output. Those outputs could be saved either on the device itself or provided by a server. The question of how to trigger different reactions was to be answered next. The device could either be placed on a pedestal equipped with \ac{RFID}-tags or the tags had to be brought to the reader in any other way. As mentioned earlier, an active tangible would be sizable and it would have to be related to the showcase's topic as well. Hence, it would be reasonable to combine those two criteria and fabricate enlarged reproductions of exhibits from the showcase. In order to fit the whole hardware, an active tangible would have to have a simple shape. This unfortunately excluded several of the more interesting exhibits, such as coins, a golden ring and other jewelry. Some options remained though. There was a skull, pottery and the metal remains of two jewelry boxes.
\\
The passive tangibles did not appear to cause this much consideration. Any exhibit could have been \ac{3D} scanned\footnote{The scans could have been done in the labs of the chair of Computer Vision and Engineering at Bauhaus-Universität Weimar.}, turned into a digital model, appropriately altered to fit an \ac{RFID}-tag and then printed or milled out. The printed or milled reproduction could be used as a positive to produce casting molds, afterwards. Thus, replacing damaged or otherwise lost tangibles would be more cost-efficient. In addition, it could be done by the museum-staff themselves. One or more \ac{RFID}-readers could be placed in a pedestal in front of the showcase. Depending on the \ac{RFID}-reader and a tangible's tag, the system would display the corresponding output.
\\
During those considerations, a third possibility came up. A hybrid approach that combined both principles was possible as well. The reproduction of a jewelry box could be turned into an active tangible and passive tangibles could be put inside to trigger an output. The \ac{RFID}-reader would be placed underneath the box's floor and the display in the lid. In order to provide different types of content, we thought about also producing two different types boxes. A more or less \textit{authentic reconstruction} made of wood and metal fittings could provide authentic information about a passive tangible's cultural background. Meanwhile, the other box could be constructed of transparent material, which would allow the user to see the hardware. This \textit{futuristic reconstruction} could provide statistical content for the same passive tangible.

However, the main problem remained with all approaches. Some kind of pedestal would have to be built and placed outside the showcase to hold the active and/or passive tangibles. Although passive tangibles would have been more cost-efficient to replace than active ones, maintenance was rated too high. Furthermore, if the pedestal was not to obscure the showcase, it would have been too low\footnote{The height of the showcase floor is about 65cm. For more details see Chapter \ref{installation}.} to grant satisfactory access for any visitor.

%-----------------------------------------------------------------------------

\paragraph{Pointing}

%As a result of the earlier drawbacks, we tried to minimize the objects outside the showcase. Hence, the display should be put inside.

The alternate concept took a completely different approach. It was more related to \ac{VR} and the interaction in \ac{3D} environments, where users are pointing at an object to select it~\cite{VRObjectSelectionCnG}. The underlying idea was to develop an information system that would be based on pointing-based interaction. A user points at an exhibit inside the showcase, the system recognizes the gesture, calculates the intended target and displays the corresponding information.
\\
Since the display should not interfere with the exhibits or occlude them, we had to make decisions about the position, size and type of the display. In order to not occlude exhibits, the display should not be placed in front or above the exhibits. Directly behind the glass panel would also have been problematical. It should have been placed along the visitors' viewing direction as they already would be looking into the showcase. This way, it would still imply coherence through visual proximity. A monitor on the one hand, and a projector on the other were two possible technologies to choose from. Both came with their own challenges. While a projector would have been easier to conceal than a monitor, a monitor would produce less heat and noise. Because most of the visitors approach the showcase from the long side and tend to stay there for most of the time, the display should be visible from this direction. This meant placing the projection plane or display on the opposing wall. Another solution for a projector came up during this consideration. A \ac{PDLC} switchable film~\cite{PDLC} could have been placed on the glass panel. Whenever the system was activated, the film and projector could have been activated as well\footnote{A \ac{PDLC} switchable film can be switched between a transparent and an opaque state. In its opaque state, it can be very well be used as a projection surface~\cite{PDLC}.}. Unfortunately, this solution would have been too expensive and difficult to install. A projection in the other direction was also disregarded, because the cost and heat issues caused by a projector were considered to high. Heat produced by a projector causes issues regarding the artifacts' conservation and is a safety risk for the sealed showcase. Therefore, we decided to install an LED-screen. It should be placed inside the showcase close to the exhibits.

\textit{Object selection in \ac{VR}-environments}~\cite{VRObjectSelectionCnG} and the \textit{SMSlingshot}~\cite{SMSlingshot}, nearly always use a \textit{pointing device} of some sort. With such a device, a potential user could directly point at the original exhibits within the showcase and trigger the corresponding reaction of the system - displaying related information. As described in detail in Chapter \ref{evaluation_pre}, I observed interactions between visitors and the showcase as well as between each other. During the pre-study, it turned out that visitors often pointed at the particular exhibits they were talking about. The interface could be designed to emulate this natural interaction between visitors and incorporate of the natural behavior.
\\
The first intention was to rebuild the SMSlingshot with Gadgeteer-hardware. The tangible was equipped with a microcontroller, a small display, a keyboard, a green laser, a wireless transmitter and of course batteries. A PC was used to put all the information together and render the output. Therefore, it had a camera to track the point a user was aiming at and a corresponding transmitter to receive the fired messages~\cite{SMSlingshot}. All those modules could be provided by Gadgeteer except the laser. A laser could have been controlled with a \textit{Breakout module}~\cite{BreakoutGHI} and a relay. However, shooting a laser into the showcase was a delicate issue. Hence, this solution had to be revisited, because for safety reasons it was not feasible. There could have been injuries of visitors' eyes or some of the precious exhibits might have reacted to the laser's energy in a corrosive way. We did not want to take those risks, but we were very keen on the idea of pointing interaction. Thus, I looked for other tracking methods. We could have used a tracking system similar to the aforementioned ones used in \ac{VR}. Those systems are expensive to install and maintain, though. Moreover, a proper compatibility with Gadgeteer was doubtful. So, I started looking for alternatives to Gadgeteer, too. Two established systems immediately came to mind. First, the \textit{Nintento Wii}, which uses a wireless device with pointing capabilities and additional inputs. Second, the \textit{\ac{MS} Kinect}, which is able to recognize free-hand gestures and might not require any device. Both are comparably inexpensive to acquire, have experienced support and communities and use less dangerous \ac{IR} light.
\\
The decision between the two was made according to the same criteria as mentioned above. Pointing with no device should be a more intuitive way to interact with the exhibition and other visitors than any handheld device. Furthermore, the restraint to use the system should be reduced. No tangible or pedestal would have to be created and attached to the showcase, which decreased cost for maintenance. Hence, the \ac{MS} Kinect was chosen.
\\
There is a Kinect for \ac{MS} Windows along with a special \ac{SDK} for \ac{MS} Visual Studio. As it turned out, the hardware inside the \ac{MS} Kinect was developed by \textit{PrimeSense} and is also used by the \textit{ASUS Xtion PRO}. This \ac{3D}-sensor is less expensive and smaller, which allows to be less intrusive inside the showcase. Besides, we already had some of them at the faculty, which meant that I could start developing right away. Another change was the decision for an open source \ac{SDK} called \textit{OpenNI}\footnote{OpenNI was co-founded by PrimeSense, a hardware developer that produces \ac{3D} sensing hardware. In November 2013 PrimeSense was bought by Apple, whereupon OpenNI was shut down.}, which in combination with its add-on \textit{NiTE} enabled me to use \textit{skeleton tracking}. This was critical for my approach, because I needed to have a \ac{3D} vector in order to be able to calculate where a user was pointing. Skeleton tracking would deliver the joints of a tracked person. Hence, I was able to retrieve the directions a limb is oriented in. If this vector was extended, I was able to calculate its possible intersection with an exhibit. More about used software and the exact calculations can be found in the next chapter.

The last topic that needed addressing was \textit{feedback}. Since there would be no haptic or acoustic feedback, and no \textit{glowing dot} produced by a laser either, future users would need another visual feedback in order to be able to see where they were pointing and determine how to correct that. Once more, Gadgeteer could have provided a solution. Our first idea was to replace the laser's dot by a spotlight. The system would calculate the position a user was pointing at and transmit it to a Gadgeteer-system. It would then move a special highlight to this position within the showcase. Only two actuators would be sufficient. The maintenance of this kind of installation could become very complicated though, because the system would have to be installed on the ceiling of the showcase. Actuators need to be calibrated regularly and mechanical gearing will wear out. Hence, this realization concept was dismissed. Nevertheless, the principle should remain the same. Thus, the aforementioned position would now be shown on an overview of the showcase on the display.

%\paragraph{Annotations}
%
%\begin{itemize}
	%\item Technical
	%\item From the museums perspective
	%\begin{itemize}
		%\item Size
		%\item Cost
		%\item Inclusion
	%\end{itemize}
	%\item Limitations of hard- and software
	%\item Capabilities of a single programmer (me)
%\end{itemize}

%-----------------------------------------------------------------------------

\section{Final Concept}
\label{conception_final}

%\textbf{Soll ich hier nochmal das Gesamtkonzept zusammenfassen oder ist es besser, das Organisatorische noch etwas mehr zu vertiefen und dafür das fertige System im Implementation-Kapitel genau zu erklären? -- Dann muss ich hier nochmal die Überschrift ändern.}

The final system consists of a \textit{depth sensor}, a \textit{PC} and a \textit{display}. All of the hardware is placed inside the showcase. In addition, an active tangible to remotely activate and deactivate the system should be developed as well. It was only intended to be a feasibility study, which determines if and how active Gadgeteer-tangibles might be incorporated into the system, later. Suitable components were recommended by me and provided by the museum after to mutual agreement.

The system requires two pieces of software. The first software is of an administrative nature and allows the museum staff to define and maintain the whole exhibition. The second software is presenting the exhibition to the visitors. Previously defined exhibits are selectable.
\\
The exhibition can be defined by museum-staff themselves. Therefore, an exhibition plane has to be defined and validated first. After that all the exhibits' positions on the plane can be defined and validated. Those positions can be defined in the same way users later interact with the system, by pointing. To exclude a certain inaccuracy when defining a position, it would have to be defined from different angles and validated afterwards. The whole process will be described in Chapter \ref{installation} and the technical execution in Chapter \ref{installation_tech}. Furthermore, the corresponding contents such as explanatory texts and detailed images are provided by the staff. Contents and positions can be changed, removed from or reloaded into the exhibition.
\\
When one or more visitors enter the area in front of the showcase the system recognizes them and reacts in an inviting fashion. A defined interaction space enables the user to interact with the system by pointing at an exhibit. No devices outside the showcase are needed. 

\paragraph{Functional Specification Document (\ac{FSD})} The final concept all parties agreed on was written down by me in an \ac{FSD} and responsibilities were covered by a contract. The document states, which features of the final system must, should and must not be implemented and working.
\\
The necessary features or \textit{must-criteria} where that the the system would have to have separate modes for administration and presentation of an exhibition. The visual feedback of the interaction would be provided by the display. Visitors would be automatically recognized by the system, but only one user at a time would be able interact with it. The whole system would be maintainable by the museum's staff and will start and shut down automatically.
\\
Preferable features \textit{should} be realized, but would not be mandatory. Thus, there should be a system's manual. For guided tours, it should be further possible to switch the system into a 'blind' mode, where it does not react to people. Extensive exhibits should have a slide show. The system should be operable with either the left or right hand. In addition, statistics about the system's use should be logged for later analysis.
\\
There were also criteria that were not requested, and therefore \textit{must not} be implemented. Any free-hand gestures other than pointing must not be recognized by the system. Further, the lighting inside the showcase must not be controlled or influenced by the installation. Feedback has to be only visual and not auditive or haptic. Hence, speakers or tangibles must not come to use. 
\\
Furthermore, the \ac{FSD} describes system requirements concerning hard- and firmwares, data formats and other organizational parameters. 

In addition to the \ac{FSD}, a contract was drawn up by the university's layer's office. It sorted responsibilities and was later signed by the museum's director, my professor and me. Both documents can be found in the appendix.

%\paragraph{Annotations}
%
%\begin{itemize}
	%\item 'Pflichtenheft'-criteria
	%\begin{itemize}
		%\item Must
		%\begin{itemize}
			%\item 
		%\end{itemize}
		%\item Should
		%\begin{itemize}
			%\item 
		%\end{itemize}
		%\item Could
		%\begin{itemize}
			%\item 
		%\end{itemize}
		%\item See appendix
	%\end{itemize}
	%\item Contract 
		%\begin{itemize}
			%\item MUFT, BUW and me
			%\item Avoid misconceptions
			%\item Commitments / Obligations
			%\item Responsibilities
			%\item Boundaries
			%\item Legal stuff
			%\item See appendix
		%\end{itemize}
%\end{itemize}

\chapter{Implementation}
\label{implementation}

The \ac{IMI}-system consists of two main parts. First, the hardware part involves the physical tracking and computing of its data in the background. Second, the software part, which includes the \ac{IMI}-libraries and -softwares utilizing them.

%\paragraph{Annotations}
%
%\begin{itemize}
	%\item Explanation of functionalities
	%\item Diagrams
	%\begin{itemize}
		%\item Classes
		%\item Sequences
	%\end{itemize}
	%\item Sketches
%\end{itemize}

%-----------------------------------------------------------------------------

\section{Testing}
\label{conception_testing}

Before anything could be installed or evaluated, a reliable system had to be developed. Therefore, I researched suitable environments for an extensible system. Because most \ac{SDK}s for PrimeSense's hardware are implemented in C++ or C$\#$ and Gadgeteer uses Microsoft's .NET framework and C$\#$, the final system should be implemented in C$\#$. Thus, an \ac{SDK} written in C$\#$ was to be found. After having tried several open source frameworks, the \ac{FUBI} developed at Universität Augsburg proved to fit our needs best. \ac{FUBI} came with a C$\#$-wrapper, which incorporated all functionality of OpenNI and NiTE that was necessary to achieve our goals. Moreover, its leading developer, \textit{Dipl.-Inf. Felix Kistler}, kindly explained how to incorporate the new approach to \ac{FUBI}.  

\paragraph{Annotations}

\begin{itemize}
	\item Test of pointing accuracy
	\begin{enumerate}
		\item One centered Point I
		\begin{itemize}
			\item Only Pointing
			\item \textit{Images and sketches}
			\item \textit{Data and Statistics}
			\item results and conclusion
			\item See appendix
		\end{itemize}
		\item One centered Point II
		\begin{itemize}
			\item Pointing, Aiming and Combined
			\item \textit{Images and sketches}
			\item \textit{Data and Statistics}
			\item results and conlusion
			\item See appendix
		\end{itemize}
		\item Four Points on each corner of the plane
		\begin{itemize}
			\item Classification of combined values
			\item \textit{Images and sketches}
			\item \textit{Data and Statistics}
			\item results and conlusion
			\item See appendix
		\end{itemize}
	\end{enumerate}
	\item Development of algorithms for eye-hand mismatch (elbow/hand + head/hand)
	\begin{itemize}
		\item Description of Eye-Hand Mismatch [ref]
		\item \textit{Sketches of classification}
	\end{itemize}
	\item Test of algorithm's accuracy
	\begin{itemize}
		\item Target = '90 percent of all values within a 10cm radius of mean value'
		\item Differentiation between real and virtual point
		\item Necessity of 1:1-mapping of real and virtual point
	\end{itemize}
\end{itemize}

%------------------------------------------------------------------------------

\section{Interactive Museum Installation - Libraries}
\label{implementation_libraries}

\paragraph{Annotations}

\begin{itemize}
	\item 'What are the libraries?'
	\begin{itemize}
		\item Overview
		\item Structure of Exhibition and Exhibits
	\end{itemize}
	\item 'What does each one do?'
	\begin{itemize}
		\item Modularity
		\item Config-files (XML)
		\item Particular methods (Lotfußpunkte, Ebenenschnittpunkt, DataLogger etc.)
	\end{itemize}
\end{itemize}



\section{Interactive Museum Installation - Administration-software}
\label{implementation_administration}

\paragraph{Annotations}

\begin{itemize}
	\item 'What is the administration-software?'
	\begin{itemize}
		\item Define and edit exhibitions
		\begin{itemize}
			\item ExhibitionPlane
			\item Define, load and remove Exhibits
			\item Define and change UserPosition
			\item Edit dwelltimes
			\item Load Background(s)
		\end{itemize}
		\item Define and edit exhibits
		\begin{itemize}
			\item Define and change Position
			\item Load and remove Images
			\item Write and load Description (up to 310 charcters)
		\end{itemize}
	\end{itemize}
	\item 'What does it do?'
	\begin{itemize}
		\item \textit{Sequences}
		\item Paper-mockup
		\item Create (re-)loadable Config-files
	\end{itemize}
\end{itemize}



\section{Interactive Museum Installation - Presentation-software}
\label{implementation_presentation}

\paragraph{Annotations}

\begin{itemize}
	\item 'What is the presentation-software?'
	\begin{itemize}
		\item Display information of previously defined interactive exhibits
		\item Overview-map of ExhibitionPlane
		\item Feedback of exhibits' positions and pointing position
		\item Description (Readability, Sehwinkel) and Images as slide show
	\end{itemize}
	\item 'What does it do?'
	\begin{itemize}
		\item Check for Exhibition
		\item Pre-calculate Lookup for exhibit-selection (saves processing power)
		\item Recognize visitors
		\item Identify user by predefined UserPosition 
	\end{itemize}
\end{itemize}



\section{Interactive Museum Installation - Presentation-remote}
\label{implementation_remote}

\paragraph{Annotations}

\begin{itemize}
	\item 'What is the presentation-remote?'
	\begin{itemize}
		\item Microsoft Gadgeteer-Device
		\item Bluetooth / WiFi-connection to PC
		\item For lecturers in order to explain exhibits themselves
	\end{itemize}
	\item 'What does it do?'
	\begin{itemize}
		\item Automatically connect to Presentation-software
		\item Toggle Presentation-software's blindness
	\end{itemize}
\end{itemize}



\section{Interactive Museum Installation - Statistics-tool}
\label{implementation_tool}

\paragraph{Annotations}

\begin{itemize}
	\item 'What is the statistics-tool and what does it do?'
	\begin{itemize}
		\item Small tool to evaluate logged user-data
		\item Statistics, such as average length of stay/session, exhibits chosen and how many transitions 
	\end{itemize}
\end{itemize}

\chapter{Setups and Hardware}
\label{installation}

Three installations were build. One lab-setup for development, one makeshift setup was placed in the faculties lobby, and the final one was installed inside the showcase in Museum für Ur- und Frühgeschichte Thüringens. The various setups differed more or less in dimensions and were run with different hardwares. Early tests were conducted with the lab-setup. The lobby-setup was used for a stress-test during an open door-event at the faculty, whereas the final evaluation took place in the museum. Then, only the presentation-software was tested.

%\paragraph{Annotations}
%
%\begin{itemize}
	%\item Current State
	%\begin{itemize}
		%\item Comparing Lab- and Summaery-setup
		%\item Documentation of system's installation
	%\end{itemize}
%\end{itemize}

%-----------------------------------------------------------------------------

\section{Lab-setup}
\label{installation_lab}

A special lab had to be found and equipped with all necessary Hardware. The Hardware was lend to me by multiple sources of the faculty, while the museum's carpenter made a pedestal consisting of a surface and feet. The surface is made out of four 9mm-press boards. The feet seemed to unstable and thus were replaced with one desk rack for each board.

%-----------------------------------------------------------------------------

\section{Lobby-setup}
\label{installation_lobby}

After some technical difficulties with the museum-setup, the first test under aggravated conditions was conducted during \textit{Summ\ae{}ry}\footnote{Summ\ae{}ry is an open door-event at the faculty of media, where all chairs present their work throughout the faculty-buildings.}. Therefore, I build a makeshift setup in the facultie's lobby. It consisted of three tables forming the exhibition plane and a bar table, on which the computer and a tripod with the sensor on top were positioned. There were three targets - a candy bar, a stack of coins, and a stack of fliers - lying on the plane (\textit{see Figure}).
\\

%-----------------------------------------------------------------------------

\section{Final museum-setup}
\label{installation_museum}

\begin{itemize}
	\item Automatic boot at 8:30am [Bios]
	\item Runnging
	\item Logfiles for each \textit{Session-Event}
	\begin{itemize}
		\item Start Session: User in interaction zone (Exhibition.UserPosition +/- Threshold from SessionHandler := 250mm)
		\item New Target: User pointing at a target
		\item Target Selected: Dwelltime (Exhibition.SelectionTime := 700ms) starts slide show for selected target
		\item End Session: User leaves interaction zone
	\end{itemize}
	\item Automatic shutdown at 4:45pm [Software]
\end{itemize}
\chapter{Evaluation}
\label{evaluation}

The \ac{IMI}-system in its final configuration was developed to be a reliable for every day use and easy to maintain. Although tests in the controlled environment of the lab were promising, it had to proof itself in a realistic scenario. Therefore, the \ac{IMI}-system was installed inside the Haßleben-showcase. Afterwards, the \ac{IMI}-exhibition about the showcase was defined by the museum staff and me. The staff was responsible for descriptive texts, detailed images and the overview sketch for proper feedback. Meanwhile, I assisted during the definition of the exhibition plane and the exhibits' positions.
\\
However, in order to determine, whether or not the \ac{IMI}-system raised awareness for the topic of the showcase a pre-study had to be made. Therefore, the behavior of visitors around the un-augmented showcase had to be observed. In addition, their awareness of the showcase's contents had to be found out as well. Upon this baseline, behavior of visitors with the \ac{IMI}-system present could be evaluated as well.
\\
A true insight could only be gained by examining how the \ac{IMI}-system would be accepted by visitors over a longer period of time. Furthermore, they should not be influenced by as less unusual circumstances as possible.

The observation and questioning of visitors mostly leads to qualitative data. This data has to be evaluated differently than quantitative data gained from experimental research. Thus, I employed methods of \textit{grounded theory}. In experimental research, a hypothesis leads to a study that produces data, which is then used to either accept or reject the hypothesis. Meanwhile, grounded theory also collects data from a study. This data is then used to shape a theory~\cite{GroundedTheory}. A comparison of the two different evaluation methods can be seen in Figure \ref{fig:experiment_vs_grounded}.

\textbf{F I G U R E}
%\begin{figure}%
%\includegraphics[width=\columnwidth]{filename}%
%\caption{.}%
%\label{fig:new_target} %flow chart von updateTarget 
%\end{figure}
 
The observations of visitors around the Haßleben-showcase already had a purpose, to gain a basic understanding of visitors' interaction. In contrast to grounded theory, we already had the basic theory that \textbf{interaction with exhibits of a showcase raises the awareness about it and its contents}. This means, that casual engagement with a topic increases the knowledge about it.
\\
Related observations have been made with an interactive installation at the Science Museum in London. Visitors were invited by the installation to interact with it. Therefore, a user had to perform different gestures to trigger an animation. Feedback was given in textual and verbal form~\cite{Engagement}. After analyzing their observations with the methods of grounded theory, Haywood an Cairns among other things concluded that
\textit{
\begin{quote}
''engagement with the exhibit does have parallels with what is needed for successful learning, and this was not previously known.''\textnormal{~\cite{Engagement}}
\end{quote}}


%-----------------------------------------------------------------------------

\section{Pre-Study}
\label{evaluation_pre}

The Haßleben-showcase is at the beginning of the second room on the second floor of the museum. Visitors frontally approach it when the come through the door. There are several related showcases in the room. Among them is a coffin in the middle of the room. In the following room, a the topic changes. There is a pottery oven in the corner and a bench on front of it. From this bench, I observed visitors in the previous room with the Haßleben-showcase in it. To disguise myself, I had one of the museum's audio guides\footnote{The Museum für Ur- und Frühgeschichte Thüringens offers audio guides for free. Visitors only have to leave a deposit. The audio guide is an app installed on an iPod Touch. It provides brief information about certain showcases and exhibits. They are identified by a sticker with the number of the corresponding audio track on it. The audio guide has German and English versions of each track.} with me.  

The interaction of visitors with the showcase itself and amongst each other was observed and noted. Further, I noted the size of a group of visitors along with their age, which in some cases had to be estimated. When visitors left the \textit{Haßleben-room}, they were asked about the showcase. The intention was to gain information about their grade of awareness considering the Haßleben-showcase. Therefore, I conducted a \textit{semi-structured interview} with each group of visitors leaving the Haßleben-room. \textbf{engagement}
\\
I hypothesized that the awareness considering a showcase can be graded into the following three stages:

\begin{enumerate}
	\item Awareness of its \textbf{mere existence}.
	\item Awareness of its \textbf{general composition}.
	\item Awareness of its \textbf{specific composition}.
\end{enumerate} 
 
Hence, my questions were intended to grade each group of visitors with respect to those stages. In the end of the interview visitors were asked about their visiting habits concerning museum. The questions were:

\begin{itemize}
	\item ''Can you remember the grave of the princess of Haßleben?''
	\item ''What can you remember? -- What objects were on display?''
	\item ''What is, in your opinion, shown in the image?''\footnote{An image of a \textit{jewel box} positioned by the feet of the princess from the Haßleben-showcase was shown to the visitors.}
	\item ''What would you change (positive or negative)?''
	\item ''What were you especially interested in? What would you like to know more about?''
	\item ''Did you read the grave's explanatory text?''
	\item ''On what occasions do you usually visit museums and how often?''
\end{itemize}

All observations and answers were noted in a protocol-sheet, which can be seen whole in the Appendix of this work. 

Ergebnisse der Protokolle:
\\- Alter (min, max, avg)
\\- Gruppengrößen (min, max, avg)
\\- Interaktion untereinander
\\- Interaktion mit Vitrine
\\- einzelne Antworten vorstellen

%-----------------------------------------------------------------------------

\section{Study}
\label{evaluation_study}

\paragraph{Annotations}

\begin{itemize}
	\item Pre- and postcondition of exhibition
	\item Survey of visitors' behavior prior to system's installation and afterwards
	\begin{itemize}
		\item Interaction between visitors
		\item Interaction with display
		\item \ac{LOS} SUS-test
		\item Interviews
		\item Evaluation-Forms
	\end{itemize}
\end{itemize}

Bimodale Verteilung der 1. Stichprobe gegen modale Verteilung der 2. Stichprobe

%-----------------------------------------------------------------------------

\section{Post-Study}
\label{evaluation_post}

\chapter{Discussion}
\label{discussion}

The pre- and main study were conducted during a period of five days. They both compare the awareness of visitors concerning the Haßleben-showcase and its exhibits. Therefore, visitors were observed during their time around the showcase and interviewed afterwards. The questions aimed at the participants \textit{stages of awareness} introduced in the previous Chapter.
\\
Moreover, a post-study was evaluated to gain an idea of how the \ac{IMI}-system was used by visitors when they did not feel monitored. Furthermore, the stored data can give an insight into necessary improvements of the current \ac{IMI}-exhibition of the Haßleben-showcase.

\paragraph{Samples and Comparability} During the pre-study 53 visitors participated in the interviews. They were distributed over 19 groups. That is an average group size of about 2.8 visitors per group. Their average age was 24.33 years. The properties of average age and group size of pre-study and main study are approximately the same. In the main study, 58 participants from 32 groups were participating. The average group size of 1.8 was considerable lower. However, the age was nearly identical. The average age of participants from the main study was 25 and hence only 6 months above the value of the pre-study.
\\
Nevertheless, the studies can not be treated as a between-subjects test. Reason for this restriction is the distribution of the participants' ages. Both samples do have the same average age, yet the distribution of the ages varies. While the sample of the main study is normally distributed, the sample of the pre-study is not. Here, the distribution is bimodal. This means that both might have a similar average age, but the reason for this fact is different. Hence, the samples are not statistically comparable when it comes to that particular criteria. The distributions of the pre-study and main study can be seen in Figure \ref{fig:discussion_age-distribution}.

\textbf{F I G U R E}
%\begin{figure}%
%\includegraphics[width=\columnwidth]{filename}%
%\caption{.}%
%\label{fig:discussion_age-distribution} %Graph der transitions zwischen den 9 targets
%\end{figure}

The majority of all casual visitors and invited participants of both studies were on an identical level of knowledge about the Haßleben-showcase. Thus, their engagement with the \ac{IMI}-system can be seen as impartial. There were sufficiently less casual visitors among the sample of the main study and, therefore, more technical experienced participants. Those pre-recruited participants might have been less restrained in using novel technologies. Their technical expertise, however, was of little use, because no devices had to be operated and the \ac{GUI} of the presentation software had not been shown to anyone prior to the main study. Hence, all participants, casual and invited, had to rely on their physical capabilities alone. Furthermore, only a few had prior knowledge of the contents of the \ac{IMI}-exhibition inside the Haßleben-showcase. Consequently, the answers of both samples can be seen as equally impartial, while the interaction of several pre-recruited participants is certainly more experienced.

%-----------------------------------------------------------------------------

\paragraph{Observed Interactions}

Interaktion bzw. Umgang mit Haßleben-showcase
\\
Interaktion bzw. Umgang untereinander

%-----------------------------------------------------------------------------

\paragraph{Interviews}

''Can you remember the grave of the princess of Haßleben?''
\\
''What can you remember? -- What objects were on display?''
\\
''What is, in your opinion, shown in the image?''
\\
''What would you change (positive or negative)?''
\\
''What were you especially interested in? What would you like to know more about?''
\\
''Did you read the grave's explanatory text?''
\\
''On what occasions do you usually visit museums and how often?''

%-----------------------------------------------------------------------------

\paragraph{Usability}

What does the outcome of the \ac{SUS}-questionnaire mean?

%-----------------------------------------------------------------------------

\paragraph{Conclusion}

The \ac{IMI}-system reliably works in a real world environment and on a daily basis. Hence, the developed system complies to our initial ambitions.

Interaction by free-hand pointing gestures is as intuitive as estimated. Visitors that were observed did not show great shyness or restraint to use the system. Because the \ac{IMI}-system is an augmentation and not a fundamental part of the showcase, the exhibition is not disturbed. With low cost and little effort, the \ac{IMI}-system is able to augment a showcase of an exhibition or other presentable setups.

%\paragraph{Annotations}
%
%\begin{itemize}
	%\item Conclusions
	%\begin{itemize}
		%\item Comparison to Conception
		%\item Comparison to 'Pflichtenheft' see \textit{Ref: Appendix}
	%\end{itemize}
	%\item Anecdotes
	%\begin{itemize}
		%\item Very short short-time memory $\to$ Instruction-sticker
		%\item Misconception of screen an a simple video and no interaction
		%\item Inhibitional factors (shyness, frustration, being watched)
	%\end{itemize}
%\end{itemize}
\chapter{Future Work}
\label{future_work}

During the development and implementation of free-hand pointing gestures as input for a public interface, new and interesting perspectives on this style of interaction appeared. The \ac{IMI}-system as such reliably works and has become an established part of the Museum für Ur- und Frühgeschichte Thüringens. My initial theory has been proved by the pre- and main study thus far. The presence of an interactive installation increases the engagement of visitors with the Haßleben-showcase and their awareness of the topic. Whether the interaction is successful or not did not seem to matter that much. The attention it creates provokes engagement with visitors, because they look at the exhibits more carefully once they are pointing at them. 
\\
Nevertheless, there is room for improvements and further research. Talking to staff of the museum and the university, fellow students and participants of the studies revealed many exciting ways to extend or improve the \ac{IMI}-system. 

While developing the basic functionality of free-hand pointing interaction, some problems were encountered and overcome with sufficient success for the \ac{IMI}-system to work properly. Yet, these issues present opportunities to improve the interaction.
\\
The first technical issue is the \textit{angular error}, which is explained in Chapter \ref{installation_tech}. The pointing position of a user is prone to error, which is directly related to the angle of impact of the pointing vector onto the exhibition plane. To minimize this another aspect of improper pointing was utilized as a counter measure. \textit{Eye-hand missmatch} makes a user point and aim at two different positions on the exhibition plane. The \ac{IMI}-system computes an average position of the two. To calculate the position, it is assumed, that either of the vectors is dominant and the combined position is biased in favor of this vector's pointing position. The dominant vector, however, is determined by each of its axes absolute value in comparison to the other vectors values. Future research might find a more reliable way of this determination. Subjects could be observed more closely while pointing and physiological aspects could be taken into consideration as well. As Figure \ref{fig:dominant_pointing} depicts, pointing with the right arm results in a drift of the pointing position to the left, whereas the aiming vector tends to go to the right.
\begin{figure}[H]%
\includegraphics[width=\columnwidth]{../pics/pivotpoints.eps}%
\caption{Leaver effects of free-hand pointing gestures.}%
\label{fig:dominant_pointing} %Draufsicht ideogram aus principle 
\end{figure}
As the elbow moves outwards, the pointing position moves to the left. Here, the hand acts as a pivot point. The same applies for the head and hand. Only if the hand moves outward, the aiming position moves with it. This time, the head is the pivot point. Hence, each joint can act as a pivot point. Recognizing and compensating those effects can result in a more reliable computation of a pointing position. 

\textit{Kernel functions} were not entirely investigated. The current \ac{IMI}-system uses a basic triangular function. The linear characteristics of affinity might be a problem in target selection. Dominant and submissive properties are modulated by the maximum value and the radius of the kernel. Non-linear functions could improve on these properties.
\\
Towards the end of the implementation functionalities of the \ac{MS} XNA Framework were used to calculate intersections. The XNA Framework defines basic geometrical shapes like planes, spheres and boxes~\cite{MSXNA}. Those shapes could be used to define new kernel functions. Furthermore, \ac{IMI}-exhibits could be defined in \ac{3D} space with a bounding sphere as a kernel around it. Hence, the exhibition plane might be obsolete. This would present many of new possibilities for public interaction with free-hand pointing gestures.

Discussions with fellow students and staff of the museum and Faculty of Media brought up the question of combining the \ac{IMI}-system with \textit{tangibles} and \textit{mobile devices}. A possible inclusion of tangibles is introduced in Chapter \ref{conception_constraints} under 'Tangibles' and in Chapter \ref{implementation_presentation} by the concept of the \textit{presentation remote}.
\\
Mobile devices could also be addressed by wireless communication like bluetooth or WiFi. For instance, the audio guide at the Museum für Ur- und Frühgeschichte Thüringens is based on an iPod Touch. These devices could be utilized by another \ac{IMI}-application to display the specific information of an \ac{IMI}-exhibit in addition to the main screen of the \ac{IMI}-system inside a showcase.

The studies confirmed the request for the aforementioned improvements and led to further possible alterations and upgrades of the \ac{IMI}-system. The two most frequently mentioned aspects of the \ac{IMI}-system that need revisiting is the feedback and the readability of the presentation software. Participants of the main study communicated that the visual feedback presented on the display was helpful, but to inconvenient. They further suggested to present the feedback directly on the exhibition plane as it was initially proposed in Chapter \ref{conception_constraints}. Moreover, participants perceived the position and size of the display as hindering, because they had to switch their focus of attention between the intended target and the visual feedback on the display. The size of the display can also be seen as a reason for some of the readability issues. Hence, a bigger display that is closer to the actual \ac{IMI}-exhibits could get rid of those problems. The second readability issue is lack of time. Participants could not finish reading the text and looking at the images. Thus, another layout for the presentation of the \ac{IMI}-exhibits should be considered. The parallelism of text and images is too confusing and users either get frustrated or have to start the presentation all over again. This leads to another proposition from several participants of the main study. They requested additional gestures as commands
\\
One implicit feature participants wished for was to allow for mobility of a user. The pre-determined interaction space restricts the visibility of the exhibits. It was not clear, that a user only had to stand in the interaction space for the selection process. When the presentation of an \ac{IMI}-exhibit was running, there was no need to stay on the footsteps. Since there is one display, only a single input can be processed by the \ac{IMI}-system. It can not handle a multitude of users pointing at different \ac{IMI}-exhibits. Currently, there has to be a mechanism to identify one user from a group of visitors, who is in charge of the interaction. Nevertheless, the location of the footsteps could be used to mark a certain user, who is then able to move around the \ac{IMI}-exhibition and interact with the \ac{IMI}-system. The mark could then be reassigned once another user steps inside the interaction space.
\\
An issue that was already mentioned during the pre-study was the improvable suitability for children. The exhibition plane is to high for small children to see all the exhibits properly. Furthermore, the recognition of a user only works for a certain height due to the definition of the user position of an \ac{IMI}-exhibition. If the location of a child's hip is too much below the hip location of the defining admin, the child can not be recognized as a user. A number of parent asked for a step to provide a raised view angle. This solution could also solve the recognition issue.

Certain adjustments to improve the interaction could be made right away. As mentioned in Chapters \ref{installation_testing} and \ref{setup_development}, minor modifications were done right away during the tests of the technical principles and different setups. In addition to that, the trembling of the feedback position in the navigational view of the presentation software was reduced by buffering. In succession to the main study the Haßleben-showcase was equipped with additional spot lights. They especially highlight areas where \ac{IMI}-exhibits are positioned.   

Finally, the \ac{IMI}-system is a novel way of interacting in public spaces. The Haßleben-showcase at the Museum für Ur- und Frühgeschichte Thüringens is an example of how the presence of a natural walk-up-and-use interface influences the perception of an ordinary showcase. The awareness about its topic and contents is raised through natural engagement. 
\\
The \ac{IMI}-system requires certain improvement and further testing. Yet it has successfully proved itself as a prototype for a public interface that needs no more input than a pointing user. And to the question, if the immediacy, control, and expressiveness of recent touch-based natural interfaces can be applied to 3D problems~\cite{ForewordCnG}?\\
-Yes, they can!

%\paragraph*{Annotations}
%
%\begin{itemize}
	%\item My work in relation to situation described in chapters \ref{introduction} and \ref{related_work}
	%\item Outlook of possible further developments or optimizations of the system
	%\begin{itemize}
		%\item Multi-user
		%\item Mobile devices
		%\item Audio
		%\item 3-dimensional positioning of objects and users
		%\item different possibilities of feedback 
	%\end{itemize}
%\end{itemize}



\bibliographystyle{alpha} % plain = [1], alpha = [DKS01]
\bibliography{../quellen/references}
\chapter*{Affidavit}

\subsubsection{Affidavit}
I hereby declare that this master thesis has been written only by the undersigned and without any assistance from third parties.
Furthermore, I confirm that no sources have been used in the preparation of this thesis other than those indicated in the thesis itself, as well
as that the thesis has not yet been handled in neither in this nor in equal form  at any other official commission.

\vspace{1.5cm}
\line(1,0){140}\\
Michael Pannier

\appendix
%\chapter{Appendix}
\chapter{Observation and Interview Sheet}
\label{appendix_box}

\begin{figure}[H]%
\includegraphics[width=\columnwidth]{../pics/jewelbox.eps}%
\caption*{Image of the jewel box placed by the feet of the princess inside the Ha\ss leben-showcase.}%
%\label{fig:biatch} 
\end{figure}

%-----------------------------------------------------------------------------

%\chapter{Observation and Interview Form}
%\label{appendix_form}

\begin{figure}[H]%
\includegraphics[width=0.75\columnwidth]{../pics/interview-a.eps}%
\caption*{First page of the observation and interview form of the pre- and main study.}%
%\label{fig:interview1} 
\end{figure}

\begin{figure}[H]%
\includegraphics[width=0.75\columnwidth]{../pics/interview-b.eps}%
\caption*{Second page of the observation and interview form of the pre- and main study.}%
%\label{fig:interview2} 
\end{figure}

%-----------------------------------------------------------------------------

%\chapter{SUS-Questionnaire}
%\label{appendix_sus}

\begin{figure}[H]%
\includegraphics[width=0.75\columnwidth]{../pics/sus-quest.eps}%
\caption*{SUS-questionnaire that was handed out to participants after the interview.}%
%\label{fig:sus-quest} 
\end{figure}


\end{document}
