% ------------------------------------------------------------------------
% LaTeX - Preambel  ******************************************************
% ------------------------------------------------------------------------
% von: Matthias Pospiech
% ========================================================================
%%&TODO: mathpackages
% ~~~~~~~~~~~~~~~~~~~~~~~~~~~~~~~~~~~~~~~~~~~~~~~~~~~~~~~~~~~~~~~~~~~~~~~~
% Einige Pakete muessen unbedingt vor allen anderen geladen werden
% ~~~~~~~~~~~~~~~~~~~~~~~~~~~~~~~~~~~~~~~~~~~~~~~~~~~~~~~~~~~~~~~~~~~~~~~~
%
\usepackage{xspace} % Define commands that don't eat spaces.
\usepackage{ifpdf} %\ifpdf \else \fi
\usepackage{calc} % Calculation with LaTeX
\usepackage[english]{babel} % Languagesetting / change to ngerman if needed
%\usepackage[table]{xcolor} % Farben
\usepackage[usenames,dvipsnames]{color} % Farben
\usepackage[]{graphicx} % Bilder
\usepackage{epstopdf} %% If an eps image is detected, epstopdf is automatically called to convert it to pdf format.
\usepackage[]{amsmath} % Amsmath - Mathematik Basispaket
\usepackage{ragged2e} % Besserer Flatternsatz (Linksbuendig, statt Blocksatz)
\usepackage{array}

% ~~~~~~~~~~~~~~~~~~~~~~~~~~~~~~~~~~~~~~~~~~~~~~~~~~~~~~~~~~~~~~~~~~~~~~~~
% Fonts Fonts Fonts
% ~~~~~~~~~~~~~~~~~~~~~~~~~~~~~~~~~~~~~~~~~~~~~~~~~~~~~~~~~~~~~~~~~~~~~~~~

\usepackage[T1]{fontenc} % T1 Schrift Encoding (notwendig für die meisten Type 1 Schriften)
\usepackage{textcomp}	 % Zusätzliche Symbole (Text Companion font extension)

%% - Latin Modern
\usepackage{lmodern}
%% -------------------
%
%% - Times, Helvetica, Courier (Word Standard...)
%\usepackage{mathptmx}
%\usepackage[scaled=.90]{helvet}
%\usepackage{courier}
%% -------------------
%%
%% - Palantino , Helvetica, Courier
%\usepackage{mathpazo}
%\usepackage[scaled=.95]{helvet}
%\usepackage{courier}
%% -------------------
%
%% - Bera Schriften
%\usepackage{bera}
%% -------------------
%
%% - Charter, Bera Sans
%\usepackage{charter}
\linespread{1.2}
%\renewcommand{\sfdefault}{fvs}



% ~~~~~~~~~~~~~~~~~~~~~~~~~~~~~~~~~~~~~~~~~~~~~~~~~~~~~~~~~~~~~~~~~~~~~~~~
% Math Packages
% ~~~~~~~~~~~~~~~~~~~~~~~~~~~~~~~~~~~~~~~~~~~~~~~~~~~~~~~~~~~~~~~~~~~~~~~~

\usepackage[fixamsmath,disallowspaces]{mathtools} % Erweitert amsmath und behebt einige Bugs
%%%\usepackage{fixmath}
%%%\usepackage[all,warning]{onlyamsmath} % Warnt bei Benutzung von Befehlen die mit amsmath inkompatibel sind.
\usepackage{icomma} % Erlaubt die Benutzung von Kommas im Mathematikmodus

% ~~~~~~~~~~~~~~~~~~~~~~~~~~~~~~~~~~~~~~~~~~~~~~~~~~~~~~~~~~~~~~~~~~~~~~~~
% Symbole
% ~~~~~~~~~~~~~~~~~~~~~~~~~~~~~~~~~~~~~~~~~~~~~~~~~~~~~~~~~~~~~~~~~~~~~~~~
\usepackage{amssymb}
%\usepackage{wasysym}
%\usepackage{marvosym}
%\usepackage{pifont}

% ~~~~~~~~~~~~~~~~~~~~~~~~~~~~~~~~~~~~~~~~~~~~~~~~~~~~~~~~~~~~~~~~~~~~~~~~
% Tables (Tabular)
% ~~~~~~~~~~~~~~~~~~~~~~~~~~~~~~~~~~~~~~~~~~~~~~~~~~~~~~~~~~~~~~~~~~~~~~~~

\usepackage{booktabs}
\usepackage{tabularx} % tabularx nach hyperref laden

% ~~~~~~~~~~~~~~~~~~~~~~~~~~~~~~~~~~~~~~~~~~~~~~~~~~~~~~~~~~~~~~~~~~~~~~~~
% text related packages
% ~~~~~~~~~~~~~~~~~~~~~~~~~~~~~~~~~~~~~~~~~~~~~~~~~~~~~~~~~~~~~~~~~~~~~~~~

\usepackage{url} % Setzen von URLs. In Verbindung mit hyperref sind diese auch aktive Links.
\usepackage[stable, ragged,  multiple]{footmisc} % Fussnoten 'perpage' entfernt
\usepackage[english]{varioref} % Intelligente Querverweise /change to ngerman if needed
\usepackage{enumitem} % Listen

\usepackage{chngcntr}
\counterwithout{footnote}{chapter} % Fussnoten unabhängig vom chapter hochzaehlen

% ~~~~~~~~~~~~~~~~~~~~~~~~~~~~~~~~~~~~~~~~~~~~~~~~~~~~~~~~~~~~~~~~~~~~~~~~
% Pakete zum Zitieren
% ~~~~~~~~~~~~~~~~~~~~~~~~~~~~~~~~~~~~~~~~~~~~~~~~~~~~~~~~~~~~~~~~~~~~~~~~

\usepackage[babel, english=british, german=quotes, french=guillemets]{csquotes} % clever quotations
\SetBlockThreshold{2} % Anzahl von Zeilen
\newenvironment{myquote}%
	{\begin{quote}\small}%
	{\end{quote}}%
\SetBlockEnvironment{myquote}

% Zitate =================================================================
\usepackage[%
	square,	% for square brackets;
	comma,	% to use commas as separaters;
	numbers,	% for numerical citations;
	sort,		% orders multiple citations into the sequence in which they appear in the list of references;
	sort&compress,    % as sort but in addition multiple numerical citations
]{natbib}

%%% Bibliography styles created with custombib
%%% Doc: ftp://tug.ctan.org/pub/tex-archive/macros/latex/contrib/custom-bib/makebst.pdf
%%%\bibliographystyle{bib/bst/AlphaDINFirstName}
%\bibstyle{plainnat}

% ~~~~~~~~~~~~~~~~~~~~~~~~~~~~~~~~~~~~~~~~~~~~~~~~~~~~~~~~~~~~~~~~~~~~~~~~
% PDF related packages
% ~~~~~~~~~~~~~~~~~~~~~~~~~~~~~~~~~~~~~~~~~~~~~~~~~~~~~~~~~~~~~~~~~~~~~~~~
\usepackage{pdfpages} % Include pages from external PDF documents in LaTeX documents

% ~~~~~~~~~~~~~~~~~~~~~~~~~~~~~~~~~~~~~~~~~~~~~~~~~~~~~~~~~~~~~~~~~~~~~~~~
% figures and placement
% ~~~~~~~~~~~~~~~~~~~~~~~~~~~~~~~~~~~~~~~~~~~~~~~~~~~~~~~~~~~~~~~~~~~~~~~~

%% Bilder und Graphiken ==================================================

\usepackage{float}             % Stellt die Option [H] fuer Floats zur Verfgung
\usepackage{flafter}          % Floats immer erst nach der Referenz setzen
\usepackage{subfig} % Layout wird weiter unten festgelegt !
\usepackage{wrapfig}	        % Bilder von Text Umfliessen lassen

% Make float placement easier
\renewcommand{\floatpagefraction}{.75} % vorher: .5
\renewcommand{\textfraction}{.1}       % vorher: .2
\renewcommand{\topfraction}{.8}        % vorher: .7
\renewcommand{\bottomfraction}{.5}     % vorher: .3
\setcounter{topnumber}{3}              % vorher: 2
\setcounter{bottomnumber}{2}           % vorher: 1
\setcounter{totalnumber}{5}            % vorher: 3

% ~~~~~~~~~~~~~~~~~~~~~~~~~~~~~~~~~~~~~~~~~~~~~~~~~~~~~~~~~~~~~~~~~~~~~~~~
% science packages
% ~~~~~~~~~~~~~~~~~~~~~~~~~~~~~~~~~~~~~~~~~~~~~~~~~~~~~~~~~~~~~~~~~~~~~~~~

\usepackage{units}

% ~~~~~~~~~~~~~~~~~~~~~~~~~~~~~~~~~~~~~~~~~~~~~~~~~~~~~~~~~~~~~~~~~~~~~~~~
% layout packages
% ~~~~~~~~~~~~~~~~~~~~~~~~~~~~~~~~~~~~~~~~~~~~~~~~~~~~~~~~~~~~~~~~~~~~~~~~

%% Zeilenabstand =========================================================
%
%%% Doc: ftp://tug.ctan.org/pub/tex-archive/macros/latex/contrib/setspace/setspace.sty
\usepackage{setspace}
%\doublespace	        % 2-facher Abstand
%\onehalfspacing        % 1,5-facher Abstand
% hereafter load 'typearea' again

%% Seitenlayout ==========================================================
%
% Layout mit 'typearea'
\typearea[current]{last}
\raggedbottom     % Variable Seitenhoehen zulassen


%% Kopf und Fusszeilen====================================================
%%% Doc: ftp://tug.ctan.org/pub/tex-archive/macros/latex/contrib/koma-script/scrguide.pdf
\usepackage[%
   automark,         % automatische Aktualisierung der Kolumnentitel
   nouppercase,      % Grossbuchstaben verhindern
]{scrpage2}

\pagestyle{scrheadings} % Seite mit Headern
%\pagestyle{scrplain} % Seiten ohne Header
%\pagestyle{empty} % Seiten ohne Header

% loescht voreingestellte Stile
\clearscrheadfoot
%\clearscrheadings
\clearscrplain
%
%\ohead{\pagemark} % Oben aussen: Seitenzahlen
\cfoot{\pagemark} % unten mitte...
%\ihead{\headmark} % Oben innen: Kapitel und Section
\chead{\headmark} % Oben mitte: Kapitel und Section

% Angezeigte Abschnitte im Header
\automark[section]{chapter} %[rechts]{links}
%
\setheadsepline{.4pt}[\color{black}] % Linie zwischen Kopf und Textk�rper

%% Fussnoten =============================================================
% Keine hochgestellten Ziffern in der Fussnote (KOMA-Script-spezifisch):
\deffootnote{1.5em}{1em}{\makebox[1.5em][l]{\thefootnotemark}}
\addtolength{\skip\footins}{\baselineskip} % Abstand Text <-> Fussnote
\setlength{\dimen\footins}{10\baselineskip} % Beschraenkt den Platz von Fussnoten auf 10 Zeilen
\interfootnotelinepenalty=10000 % Verhindert das Fortsetzen von
                                % Fussnoten auf der gegenüberligenden Seite

%% Schriften (Sections )==================================================

% -- Koma Schriften --
\newcommand\SectionFontStyle{\sffamily}
\setkomafont{chapter}{\huge\SectionFontStyle}    % Chapter
\setkomafont{sectioning}{\SectionFontStyle} %  % Titelzeilen % \bfseries
%\setkomafont{pagenumber}{\bfseries\SectionFontStyle}             % Seitenzahl
\setkomafont{pagenumber}{\normalfont\SectionFontStyle}             % Seitenzahl nicht fett...
\setkomafont{pagehead}{\itshape\small\sffamily}        % Kopfzeile, beeinflusst aber auch fusszeile...
\setkomafont{descriptionlabel}{\itshape}        % Kopfzeile
%
\renewcommand*{\raggedsection}{\raggedright} % Titelzeile linksbuendig, haengend
%

%% Captions (Schrift, Aussehen) ==========================================

%%% Doc: ftp://tug.ctan.org/pub/tex-archive/macros/latex/contrib/caption/caption.pdf
\usepackage{caption}
% Aussehen der Captions
\captionsetup{
   margin = 10pt,
   font = {small,rm},
   labelfont = {small,bf},
   format = plain, % default oder 'hang'
   indention = 0em,  % Einruecken der Beschriftung
   labelsep = colon, %period, space, quad, newline
   justification = RaggedRight, % justified, centering
   singlelinecheck = true, % false (true=bei einer Zeile immer zentrieren)
   position = bottom %top
}
%%% Bugfix Workaround
\DeclareCaptionOption{parskip}[]{}
\DeclareCaptionOption{parindent}[]{}

%\DeclareGrphicsExtensions{.pdf,.png,.jpg}

% Aussehen der Captions fuer subfigures (subfig-Paket)
\captionsetup[subfloat]{%
   margin = 10pt,
   font = {small,rm},
   labelfont = {small,bf},
   format = plain, % default oder 'hang'
   indention = 0em,  % Einruecken der Beschriftung
   labelsep = space, %period, space, quad, newline
   justification = RaggedRight, % justified, centering
   singlelinecheck = true, % false (true=bei einer Zeile immer zentrieren)
   position = bottom, %top
   labelformat = parens % simple, empty % Wie die Bezeichnung gesetzt wird
 }

%% Inhaltsverzeichnis (Schrift, Aussehen) sowie weitere Verzeichnisse ====

\setcounter{secnumdepth}{2}    % Abbildungsnummerierung mit groesserer Tiefe
\setcounter{tocdepth}{2}		 % Inhaltsverzeichnis mit groesserer Tiefe
%

% Auszufuehrende Befehle  ------------------------------------------------

\listfiles
%------------------------------------------------------------------------


\newcolumntype{b}{>{\global\let\currentrowstyle\relax}}
\newcolumntype{n}{>{\currentrowstyle}}
\newcommand{\rowstyle}[1]{\gdef\currentrowstyle{#1}%
  #1\ignorespaces
}


